\section{Related Work}

What is vandalism: \cite{wiki:vandalism}, \cite{Viegas2004},
\cite{Priedhorsky2007}
The official definition for vandalism on the Wikipedia defines it as
``a \textit{deliberate} attempt to compromise the integrity
of Wikipedia''~\cite{wiki:vandalism}.
This definition is open to interpretation, so it can't be cleanly
formulated.
Proxies for this definition abound:
\begin{itemize}
\item \textbf{revert} -- test earlier revisions for textual equality.
\item \textbf{comment revert} -- the comment indicates that the edit
    is a revert.
\item \textbf{rollback} -- the comment indicates that it is a
    rollback; rollbacks can only be performed by editors with
    administrative capabilities.
\item \textbf{edit quality} -- compare textually against future
    revisions to discover how the edit fares.
\item \textbf{manual annotation} -- we know vandalism when we see it;
    bad because people reasonably disagree about what rises to the
    level of vandalism.
\end{itemize}

Chin\etal focus on NLP techniques, but used supervised active
learning (lookup that phrase) to obtain new annotations that
provided valuable counter-examples to their machine learning
algorithm~\cite{Chin2010}.

Smets\etal experiment with Naive Bayes applied to a Bag-of-Words
model, but also pursue a compression-based
classifier~\cite{Smets2008}.
Itakura and Clarke follow up with a different compression-based
classifier, and focus separately on insertion spam and
text replacement spam\~cite{Itakura2009}.

Belani~\cite{Belani2010} also implements a Naive Bayes classifier,
but evaluates the performance using several different techniques.


\begin{tabular}{| r | c | c | c | c | c | c |}
\hline
Work & Metadata & Text & Language & Reputation & Precision & Recall \\
\hline
Potthast \etal, 2008~\cite{Potthast2008}
    & 3 & 8 & 4 & 1
    & 86\% & --- \\
Smets\etal, 2008~\cite{Smets2008}
Chin \etal, 2010~\cite{Chin2010}
    & 0 & 3 & 9 & 0
    & ~81\% & --- \\
\hline
\end{tabular}


