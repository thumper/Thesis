\chapter{Abstract}
\begin{comment}
WikiTrust: From Quantity to Quality
by
B. Thomas Adler
\end{comment}

The Wikipedia was initially created as a sub-project to another
online encyclopedia, to promote collaboration between writers
before submitting their work to a peer review process.
Ironically, the criticism most widely levied against the Wikipedia
is the lack of accountability
for authors, and the potential to misinform readers.
There is a large community around the Wikipedia project which actively
fixes errors as they are discovered, but an unending
stream of vandals and spammers chip
away at the good will of volunteers who
maintain the project for the collective good.
We suggest that a vandalism detection system
can be used to focus
the volunteer effort on edits more likely to be a problem,
making more efficient use of the project's human resources.

We develop measures to quantify the effort of authors
(based solely on the revision history of the Wikipedia),
and show how these might be combined into an author
reputation system.
We desire that author reputation be correlated with the
stability of the text they contribute ---
low reputation should be a predictor of author
contributions which will be edited or deleted.

We propose automated methods for evaluating the quality of
authors, and (separately) the text they create.
By relying on the history of edits, we can draw inferences about
the quality of the text that has come earlier in the history.
Instead of measuring ``truth,'' our reputation ideas
measure the ``group consensus'' in a piece of text.
As the article text stabilizes over time, we conclude that
it has reached a form which most members of the community can
reasonably agree on.
As group collaboration increases in prominence on the Internet,
we feel that this research will open the door on new applications
and quality measures.

