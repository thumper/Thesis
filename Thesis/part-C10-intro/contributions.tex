\subsection{Contributions of this Work}

We propose research into two ideas that can be leveraged
into tools to assist well-intentioned users.
The first is a \intro{reputation system} for users, which measures
their contributions to the Wikipedia with respect to some
notion of the group consensus.
The second idea is a \intro{trust system} for text,
which models the amount of review each word receives from editors
to quantify the likelihood that the text will persist
into the future.
One of our goals is to preserve the current user experience,
as much as possible.


Our basic idea is to take advantage of the Wikipedia's version history
for articles to develop content-driven \intro{reputation systems}
and \intro{trust systems}~\cite{WikiMTWtrust06,McGuinness06,www07,WikiTrust2008}.
A content-driven reputation system is one which uses
the version history to measure the work performed by users;
users gain reputation when they make
contributions to the Wikipedia which are deemed valuable, and lose reputation
when they make contributions which are reverted.
A content-driven trust system uses the version history
to predict the quality of the text submitted, incorporating
information from the reputation system.
The trust system acts as a guide to users,
indicating recent edits that need more scrutiny.
The advantage of content-driven systems,
as opposed to manual rating systems, is that they do
not significantly alter the user experience.

We have implemented an initial prototype of our reputation and
trust algorithms~\cite{www07,WikiTrust2008}.
This system takes as input an XML dump of a Wikipedia project,
analyzes the evolution of articles to determine lasting
contributions by authors, and computes a reputation for
each author based on the size and age of their contributions.
At the same time, a trust value is computed for each word
based on the reputations of authors who edit
regions of text nearby.
Our evalutions for both algorithms are very encouraging:
low reputation authors predominantly have their contributions
reverted, and low trust text is much more likely to be
deleted on the next revision than high trust text.
Further details of the algorithms and evaluations
are discussed in Sections~\ref{sec-reputation}
and~\ref{sec-trust}.

The course of our continued investigation is to examine
other applications of reputation and trust systems in
the context of the Wikipedia:

