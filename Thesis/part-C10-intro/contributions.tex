\section{Contributions of this Work}

Our thesis is that the creation of an automated reputation system
for editors of the Wikipedia can be used to good effect in the creation
of a Wikipedia vandalism detection system.

We develop a quality measure to reflect the reaction of
the community to an edit.
This quality measure is consistent with the less general
``reverts are bad'' notion used in other
works~\cite{Smets2008,Itakura2009,Belani2010,West2010},
but is more nuanced in allowing a contribution to be massaged
but still be considered as adding some value to the project.
As part of realizing this quality measure, we introduce the
author attribution problem for collaborative works,
and a refined definition for edit distance applicable to
measuring contributions in a collaborative work.

Using the edit quality measure as a basis, we are able
to construct a \intro{reputation system} for authors.
We show that this reputation system has the useful property
of predicting the edit quality of future edits by the same
author.

With the availability of annotated corpora for the vandalism
detection problem, we also develop a machine learning solution based on our
WikiTrust technologies.
We extract features based on calculations made as part of the
reputation system and show that the resulting predictions perform
competitively against other vandalism detection solutions.

The programs resulting from this research are open source,
available under the BSD license from our project website:
\url{http://www.wikitrust.net/}.

