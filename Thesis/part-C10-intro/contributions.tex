\section{Contributions of this Work}

This dissertation takes up the question of how to help readers of the
Wikipedia understand the quality of an article.
Our approach is to examine the revision history of an article
and summarize the evolution of text through a reputation system
rating the authors of the text.

We develop a quality measure to reflect the reaction of
the community to an edit.
This quality measure is consistent with the less general
``reverts are bad'' notion used in other
works~\cite{Smets2008,Itakura2009,Belani2010,West2010},
but is more nuanced in allowing a contribution to be revised
and still be considered as adding some value to the project.
As part of constructing this quality measure, we consider the
author attribution problem for collaborative works with a revision
history, and a definition for edit distance applicable to
measuring contributions in the collaborative work.

Using the edit quality measure as a basis, we show how
to construct a \intro{reputation system} for authors.
We demonstrate that this reputation system has the useful property
of predicting the edit quality of future edits by the same
author.
With the availability of annotated corpora for the vandalism
detection problem, we also develop a machine learning solution based on our
WikiTrust technologies.
We extract features based on calculations made as part of the
reputation system and show that the resulting predictions perform
competitively against other vandalism detection solutions.

The programs resulting from this research are open source,
available under the BSD license from our project website:
\url{http://www.wikitrust.net/}.

