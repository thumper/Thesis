\section{Contributions of this Work}

Our thesis is that the task of Wikipedia vandalism detection
can be assisted by the creation of a reputation system
for editors.

We develop a quality measure to reflect the reaction of
the community to an edit.
This quality measure is consistent with less general
``reverts are bad'' notion used in other
works~\cite{Smets2008,Itakura2009,Belani2010,West2010},
but is more nuanced in allowing a contribution to be massaged
and be considered to have had some value to the project.
As part of realizing this quality measure, we introduce the
author attribution problem for collaborative works,
and a refined definition for edit distance to better capture
the notion of the size of a contribution.

Using the edit quality measure as a basis, we are able
to construct a \intro{reputation system} for authors.
We show that this reputation system has the useful property
of predicting the edit quality of future edits by the same
author.

The programs resulting from this research are open source,
available under the BSD license from our project website:
\url{http://www.wikitrust.net/}.

