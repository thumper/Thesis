\section{World Wide Collaboration}

The dot-com boom of the late 1990's brought the open source movement into
mainstream consciousness, bringing with it the mantra ``information
wants to be free''~\cite{wiki:Information-free}.
In the midst of this environment, the
Wikipedia\footnote{\url{http://www.wikipedia.org}}
first appeared.
The Wikipedia
is an online encyclopedia using an open model of group collaboration
where anyone can contribute: when an article is
displayed, any reader can click on an ``edit'' button to modify
the text as they see fit.
Thanks to this openness, the Wikipedia has grown to
over 3.4 million articles and as of September 2011,
is the seventh most visited site on the
web\footnote{According to Alexa traffic rankings, \url{http://www.alexa.com}}.
This large-scale group collaboration
has become known as \intro{crowdsourcing}.
By encouraging visitors to contribute their own content,
sites adopting this model\footnote{For example,
Flickr, YouTube, StackExchange and even Facebook.}
hope to grow rapidly as users build on each others' work~\cite{Taylor2007}.



The particularly open model of group collaboration that the Wikipedia embodies
in allowing anonymous contributions
also receives much criticism about the potential for
misinformation~\cite{Seigenthaler05,NewYorkTimes05a,Lehmann2006,Hickman2006,Davis2006,Stross2006,Schiff2006,BBC2007},
both intentional and accidental.
A study comparing the quality of the Wikipedia against that
of the Encyclop{\ae}dia Brittanica found that the number of errors
in the Wikipedia is very near that of the curated work~\cite{Giles2005},
but this achievement is not attained by mere chance.
As an online encyclopedia, accuracy of information is a major concern,
thus the Wikipedia community has developed a process
of eternal vigilance around screening edits: their volunteer \intro{RC
Patrol} scans all recent changes and reverts edits that they do not
consider suitable~\cite{wiki:RCPatrol}.
The important feature of the Wikipedia that enables this
process is that all past versions are kept for each article.
Users can easily roll back an article to a previous version,
undoing the contributions of other users.
A fundamental insight behind wiki development is that,
if well-intentioned and careful users outnumber ill-intentioned
or careless users in the community, then valuable content will predominate,
since the undesired contributions are easily undone~\cite{Leuf2001}.

Online communities have a typical lifecycle:
a small community develops and rallies around unifying principles;
then the community grows and attracts a more diverse group of members;
finally, the relative anonymity of a large community encourages
a small ``anti-social element.''
The problem for the Wikipedia is how to keep these bad actors at bay.
Obvious vandalism is easy to identify and revert,
but minor changes to factual information can be quite insidious.
For example, changing a date by a few days is difficult for anyone
to verify as a correction and not vandalism.
Most famously, an anonymous user created a new biography entry
within Wikipedia with the following
text~\cite{Seigenthaler05,NewYorkTimes05a,NewYorkTimes05b}:
\begin{quote}
``John Seigenthaler Sr.\ was the assistant to
Attorney General Robert Kennedy in the early 1960s.
For a short time, he was thought to have been directly involved
in the Kennedy assassinations of both John, and his brother, Bobby.
Nothing was ever proven.''
\end{quote}
This mixture of fact and fiction is very plausible given the
mythic nature of the Kennedy assassination, and it is impossible
for the RC Patrol to validate without checking reference materials.

