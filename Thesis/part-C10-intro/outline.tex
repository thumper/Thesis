\section{Outline}

This work is divided into two halves.
Part~\ref{part:foundations} covers the basic concepts we developed
before a reputation system could be constructed.
Part~\ref{part:applications} covers some applications that can
be built atop the basic concepts to create useful tools for
a collaborative community.

We start Part~\ref{part:foundations}
by defining terminology and notation in Chapter~\ref{ch:defs}.

We then introduce the author attribution problem in Chapter~\ref{ch:diff}.
In order to accord credit to authors for the work they do,
it is first necessary to identify their contribution properly.
We extend a traditional \intro{greedy text difference} algorithm to
account for the existance of multiple authors and the possibility
of restoring text from older revisions.

Chapter~\ref{ch:editquality} then looks at measuring the size of
an author contribution.
The well-known answer is to use edit distance~\cite{Levenshtein1966},
but we discover that the traditional formulation seems to fall short.
We refine the definition to build on the result computed by the
algorithm in Chapter~\ref{ch:diff}, and account for the rearranging
and substitution of text.
Finally, we conclude the foundational elements by developing a measure
of edit quality in Chapter~\ref{ch:editquality}.
We construct this measure to estimate the approval or disapproval of
later authors on the same article.
By aggregating the judgement of multiple later authors, we arrive at
a value which represents the community's assessment of an edit.

\mynote{Need outline for Part II.}

