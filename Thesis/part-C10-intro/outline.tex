\section{Outline}

We start by defining terminology and notation in Chapter~\ref{ch:defs}.
We then introduce the author attribution problem in Chapter~\ref{ch:diff}.
To accord credit to authors for the work they do,
it is first necessary to identify their contribution properly.
We extend a greedy text difference algorithm~\cite{Reichenberger1991,Burns1997} to
account for the existence of multiple authors and the possibility
of restoring text from older revisions.

Chapter~\ref{ch:editquality} then looks at measuring the size of
an author contribution.
The well-known answer is to use edit distance~\cite{Levenshtein1966},
but we discover that the traditional formulation must be adjusted
to take into account the rearranging of text that some editors contribute.
We refine the definition to build on the result computed by the
algorithm in Chapter~\ref{ch:diff}, and account for the rearranging
and substitution of text.
Finally, we conclude the foundational elements by developing a measure
of edit quality in Chapter~\ref{ch:editquality}.
We construct this measure to estimate the approval or disapproval of
later authors on the same article.
By aggregating the judgement of multiple later authors, we arrive at
a value that represents the community's assessment of an edit.

In Chapter~\ref{ch:contrib} we propose a simple model to combine
quality measures into a measure of an author's \textit{contribution}.
Chapter~\ref{ch:reputation} constructs a reputation system
out of the contribution measure, and evaluates the performance
via several different metrics.
Chapter~\ref{ch:vandalism} uses the reputation system as a feature
for machine learning in the vandalism detection problem, and compares
the performance to other solutions for the same problem.

In Chapter~\ref{ch:conclusion}, we summarize the results of this work
and present some directions for future work.

