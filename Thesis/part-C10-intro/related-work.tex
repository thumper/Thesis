\subsection{Related Work}

As the Wikipedia becomes a standard resource for the Internet
public, there is rising interest in quality measures.
A series of incidents show that the Wikipedia can be manipulated,
despite the ``many eyes'' reviewing the site:
a prank biography~\cite{Seigenthaler05,NewYorkTimes05a,NewYorkTimes05b},
congressional aides adjusting political
biographies~\cite{TheSun06,NewZelandHerald06,BBC06},
a user pretending to be a professor~\cite{BBC07},
and a slew of other self-interested parties making
inappropriate edits~\cite{Wired07,Wikiscanner07,NPR08}.
Articles have been written questioning the
general credibility of the Wikipedia~\cite{NewYorkTimes06,TheNewYorker06},
and a scientific study addressing the question
has been published~\cite{Giles05}.


The idea of assigning trust to specific sections of text of Wikipedia
articles as a guide to readers has been previously proposed in the scientific
literature~ \cite{WikiMTWtrust06,Cr06,McGuinness06}, as well as in white
papers~ \cite{King07} and blogs~\cite{PaoloMassa07}; these papers also contain
the idea of using text background color to visualize trust values.
% Luca: changed this section, to make the difference better stand
% out.
The work most closely related to our trust coloring is~\cite{McGuinness06},
where the idea of computing text trust from the
revision history was introduced.
The methods are different from ours:
% That work introduced two ideas that proved influential for our
% approach:
% that the trust of a piece of text could be computed from the
% reputation of the original author, and the reputations of the authors
% who subsequently revised the article, and that the quality of a trust
% labeling could be evaluated via its ability to predict text
% stability.
in~\cite{McGuinness06}, the analysis is performed at the granularity
level of sentences.
% ; all sentences introduced in the same revision form
% a {\em fragment,} and share the same trust.
% The trust values of the fragments are computed using a Bayesian
% network, which takes as input at every version the previous trust of
% the fragment, the amount of surviving fragment text, and the
% reputation of the version's author.
Flagging individual text changes is not a goal of the algorithm: text
insertions are not displayed at word level, and fragment reorderings
and deletions are not flagged via the trust labels.
Deleted text is not tracked: if text is deleted, and then re-inserted,
it is counted as new.
Among other things, this creates an incentive to vandalism:
blanking an article suffices to reset its entire trust assignment.
% To validate the trust assignment, \cite{McGuinness06} computes the
% correlation between the trust of a fragment, and the probability that
% the fragment appears in the most recent version of the article.
% We refine this criterion into one of our evaluation criteria, namely,
% the predictive power of trust with respect to word longevity.
In~\cite{WikiMTWtrust06}, the trust of authors and fragments is
computed on the basis of the author-to-fragment and
fragment-of-article graphs; the authors claim in~\cite{McGuinness06}
that the subsequent approach of using the revision history yielded
superior results.
% The white paper \cite{King07} focuses on the user interface aspects of
% displaying information related to trust and author contributions.
% Related work that relies on an analysis of revision information to
% infer trust has been performed in the context of software, where logs
% are mined in order to find revision patterns that point to possible
% software defects and weak points (see, e.g., \cite{Livshits05}).

Other studies of Wikipedia quality have focused on trust as
article-level, rather than word-level, information.
These studies can be used to answer the question of whether an
article is of good quality, or reliable overall, but cannot be used to
locate within an article the portions of text which deserve the most
scrutiny, as our approach can.
In~\cite{WikiTrust06}, which inspired~\cite{McGuinness06}, the
revision history of a Wikipedia article is used to compute a trust
value for the entire article.
In~\cite{Emigh05b,Mingus07}, metrics derived via natural language processing
are used to classify articles according to their quality.
In~\cite{Lih04}, the number of edits and unique editors are used to
estimate article quality.
The use of revert times for quality estimation has been proposed
in~\cite{Viegas04}, where a visualization of the Wikipedia editing
process is presented; an approach based on edit frequency and dynamics
is discussed in~\cite{WilkinsonHuberman07}.
There is a fast-growing body of literature reporting on
statistical studies of the evolution of Wikipedia content,
including~\cite{Viegas04,Voss05,Ortega07}; we refer to~\cite{Ortega07} for an
insightful overview of this line of work.
The history flow of text contributed by Wikipedia authors has
been studied with flow visualization methods in~\cite{Viegas04}.
% the results have been used to analyze a number of interesting patterns
% in the content evolution of Wikipedia articles.
% Work on mining software revision logs (see, e.g., \cite{Livshits05})
% is similar in its emphasis of in-depth analysis of revision logs; the
% aim, however, is to find revision patterns and indicators that point
% to software defects, rather than to develop a notion of author
% reputation.

In our work on the Wikipedia, we keep the notions of reputation and
trust distinct, as they refer to different entities: reputation to
authors, and trust to content.
This distinction is not always present in other work; when reputation
and trust are applied to the same entity, the computed values of
reputation are directly intended to suggest the trust that can be
placed in an entity.
Reputation systems have been the topic of much research; a general
overview can be found in~\cite{ResnickZFK00}, where several classes of
approaches are described and compared.
Systems that infer reputation from content analysis have a long
history.
The PageRank algorithm~\cite{PageRank98}, as well as the
hubs-and-authorities approach of~\cite{Kleinberg99}, infer the
reputation of hyper-linked documents from the link structure; this idea
spurred a very large amount of work on ranking of Web pages, the
fundamental problem for Web search engines.
Reputation systems based on content analysis have also been proposed
for the filtering of email~\cite{GolbeckEmail04}, for electronic
marketplaces~\cite{Zacharia99,YuSingh00,Wilensky06},
and for grid computing~\cite{PatelTeacyJennings05}, among other
applications.

The notion of information trust is extremely broad, with entire
conferences dedicated to it~\cite{itrust}; for a general overview, see
e.g.~\cite{Grandison00}.
Much work on the notion of trust has been done in the context of
social networks; see
e.g.~\cite{Dellarocas03,Kamvar03,Guha04,Golbeck05}.
These notions of trust are generally based on user feedback, rather
than on algorithmic content analysis.
Our approach is also related to {\em collaborative filtering,} where
feedback from diverse participants or information sources is
aggregated into a single notion of information trust
(see, e.g.,~\cite{Tapestry,Filtering95,Upendra95,Breeseetal-UAI98}).
% In particular, regarding collaboration for content creation,
% a reputation system for automating paper review has been considered in
% \cite{ReviewerRating01}: reviewers who in the past predicted the
% correct outcome of paper acceptance gain reputation, and their
% reputation is used to influence future acceptance/rejection decisions.
% In our system, authors whose edits go in the general direction of page
% change gain reputation, and their reputation lends trust to the text
% they add or leave in place (see Section~\ref{sec-base-system}); the
% principles are clearly related, even though the underlying techniques
% are different.

