
\chapter{Definitions}
\label{ch:defs}

The following terminology and notation
will be used throughout this work.

\renewcommand{\labelitemi}{}
\begin{itemize}
\item \articles \ -- the set of main Wikipedia \intro{articles}.
    We use this term interchangeably with the term \intro{pages},
    as each article appears as a single web page on the Wikipedia site.
    For this work, we only consider articles which are in
    the NS0 (Main) namespace to be in \articles.
    Talk and User pages, for example, are in other namespaces.
\item \users \ -- the set of registered Wikipedia \intro{users},
    plus a single anonymous user.
    MediaWiki stores IP address information for anonymous users,
    but due to the ambiguity of tracking users this way,
    we map all anonymous use to the single anonymous user.
\item \revisions{\article{}} \ -- each article $\article{} \in \articles$,
    has a history of versions
    describing how that article has evolved over time.
    We denote the chronologically ordered list
    of all $n > 0$ revisions of article \article{} by
    \begin{equation}
	\revisions{\article{}} =
		\version{1}, \version{2}, \ldots, \version{n}
    \end{equation}
    Version \version{1} is the first instantiation of the article,
    and version \version{n} is the most recent.
    A common pattern among Wikipedia editors is to save checkpoints
    of their edits, so that there are several consecutive revisions by
    the same author.
    We \intro{filter} revisions to keep only the last in a sequence of
    consecutive revisions by the same author.
    Thus, throughout this work, we assume that two consecutive revisions
    of an article never have the same author.
    \mynote{Take this out?}
    Sometimes for convenience, we take version \version{0} to be
    an empty revision which precedes all other revisions.
\item \revisions{} \ -- the chronologically ordered list of all
    revisions, across all articles in \articles.
    That is,
    \begin{equation}
    \revisions{} = \mathrm{sort}_\mathrm{time}\left[
	\bigcup_{\article{} \in \articles} \revisions{\article{}}
	\right]
    \end{equation}

\item \version{i} -- each article, $\article{} \in \articles$,
    has a history of versions
    describing how that article has evolved over time.
    The article to which the versions belong will generally be
    obvious from the context, but where it is not we will
    write $\version{i}^{\article{}}$ to indicate that we
    are referring to version \version{i} of article \article{}.

\item \revauthor{\version{i}} -- every revision of an article has
    a user associated with it with the edit.
    This function gives the user associated with \version{i} of
    an article.
    For anonymous users (who are distinguished by IP address in
    MediaWiki), we map each to the single anonymous user in \users.

\item \match{\version{i}, r, \prevrevs{\version{i}}} -- a function
    that locates the best matching text for the $r^{th}$ word of
    \version{i} in the text of previous revisions.
    See Equation~\ref{eq:bestmatch} and the accompanying text
    for further description and development.
        \mynote{Include $a$ in notation?  Take out $v$ part?}

\item \dist{a}{i,j} -- is the edit distance between revisions
    \version{i} and \version{j} of article $a \in \articles$.
    \mynote{Put in reference to chapter section.}

\item \tsurv{a}{i,j} -- for article $a \in \articles$, measures the
    amount of text which was introduced in revision \version{i}
    that survives to \version{j}.

\item \judges{a}{n}{i} -- a function which returns the set of
    \intro{judging} revisions associated with revision \version{i}
    of article $a \in \articles$.
    A judging revision for revision \version{i} is a revision \version{j}
    such that $j > i$ and
    $\revauthor{\version{i}} \not= \revauthor{\version{j}}$.
    The set is limited to at most $n$ revisions which are nearest
    to \version{i} in time.

\end{itemize}
\renewcommand{\labelitemi}{$\bullet$}


is obtained by author \editor{i} performing an
edit $\edit{i}: \version{i-1} \goesto \version{i}$.
We refer to the change set corresponding to
$\edit{i}: \version{i-1} \goesto \version{i}$
as the \intro{edit} performed at $\edit{i}$; the edit consists of the text
insertions, deletions, displacements, and replacements that led from
$\version{i-1}$ to $\version{i}$.

When editing versioned documents, authors commonly save several
versions in a short time frame, in order to avoid losing their work
due to computer or network problems.
To ensure that such behavior does not affect reputations, we
\textit{filter} the versions, keeping only the \textit{last} of consecutive
versions by the same author; we assume therefore that
$\editor{i} \neq \editor{i+1}$,
where $1 \leq i < n$.

Every version $\version{i}$, for $0 \leq i \leq n$, consists of a sequence
$[w^i_1, \ldots, w^i_{m_i}]$ of words, where $m_i$ is the number of
words of $v_i$; we have $m_0 = 0$.
For us, a \intro{word} is a whitespace-delimited sequence of
characters in the Wiki markup language that produces a Wikipedia article:
we work at the level of such markup language, rather than at the level
of the HTML produced by the wiki engine.

