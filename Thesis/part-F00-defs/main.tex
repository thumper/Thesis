
\chapter{Definitions}
\label{ch:defs}

The following terminology and notation
will be used throughout this work.

\renewcommand{\labelitemi}{}
\begin{itemize}
\item \articles \ -- the set of main Wikipedia \intro{articles}.
    We use this term interchangeably with the term \intro{pages},
    as each article appears as a single web page on the Wikipedia site.
    For this work, we only consider articles which are in
    the NS0 (Main) namespace to be in \articles.
    Talk and User pages, for example, are in other namespaces.
\item \users \ -- the set of registered Wikipedia \intro{users},
    plus a single anonymous user.
    MediaWiki stores IP address information for anonymous users,
    but due to the ambiguity of tracking users this way,
    we map all anonymous use to the single anonymous user.
\item \version{i} -- each article, $\article{} \in \articles$,
    has a history of versions
    describing how that article has evolved over time.
    We assume that we have $n > 0$ versions
    $\version{0}, \version{1}, \version{2}, \ldots, \version{n}$
    of an article;
    version \version{0} is empty,
    and version \version{n} is the most recent.
    The article to which the versions belong will generally be
    obvious from the context, but where it is not we will
    write $\version{i}^{\article{}}$ to indicate that we
    are referring to version \version{i} of article \article{}.
\end{itemize}
\renewcommand{\labelitemi}{$\bullet$}

The Wikipedia is made up of a collection of \intro{articles};
we use this term interchangably with the term \intro{pages},
as each article appears as a single web page on the Wikipedia site.
Each page, \page{}, has a history of versions describing how that
page has evolved over time.


is obtained by author \editor{i} performing an
edit $\edit{i}: \version{i-1} \goesto \version{i}$.
We refer to the change set corresponding to
$\edit{i}: \version{i-1} \goesto \version{i}$
as the \intro{edit} performed at $\edit{i}$; the edit consists of the text
insertions, deletions, displacements, and replacements that led from
$\version{i-1}$ to $\version{i}$.

When editing versioned documents, authors commonly save several
versions in a short time frame, in order to avoid losing their work
due to computer or network problems.
To ensure that such behavior does not affect reputations, we
\textit{filter} the versions, keeping only the \textit{last} of consecutive
versions by the same author; we assume therefore that
$\editor{i} \neq \editor{i+1}$,
where $1 \leq i < n$.

Every version $\version{i}$, for $0 \leq i \leq n$, consists of a sequence
$[w^i_1, \ldots, w^i_{m_i}]$ of words, where $m_i$ is the number of
words of $v_i$; we have $m_0 = 0$.
For us, a \intro{word} is a whitespace-delimited sequence of
characters in the Wiki markup language that produces a Wikipedia article:
we work at the level of such markup language, rather than at the level
of the HTML produced by the wiki engine.

