
\chapter{Definitions}
\label{ch:defs}

The following terminology and notation
will be used throughout this dissertation.

The Wikipedia is made up of a collection of \intro{articles};
we use this term interchangably with the term \intro{pages},
as each article appears as a single web page on the Wikipedia site.
Each page, \page{}, has a history of versions describing how that
page has evolved over time.
We assume that we have $n > 0$ versions
$\version{0}, \version{1}, \version{2}, \ldots, \version{n}$
of a page;
version \version{0} is empty, version \version{n} is the most recent,
and version \version{i}, for $1 \leq i \leq n$,
is obtained by author \editor{i} performing an
edit $\edit{i}: \version{i-1} \goesto \version{i}$.
We refer to the change set corresponding to
$\edit{i}: \version{i-1} \goesto \version{i}$
as the \intro{edit} performed at $\edit{i}$; the edit consists of the text
insertions, deletions, displacements, and replacements that led from
$\version{i-1}$ to $\version{i}$.

When editing versioned documents, authors commonly save several
versions in a short time frame, in order to avoid losing their work
due to computer or network problems.
To ensure that such behavior does not affect reputations, we
\textit{filter} the versions, keeping only the \textit{last} of consecutive
versions by the same author; we assume therefore that
$\editor{i} \neq \editor{i+1}$,
where $1 \leq i < n$.

Every version $\version{i}$, for $0 \leq i \leq n$, consists of a sequence
$[w^i_1, \ldots, w^i_{m_i}]$ of words, where $m_i$ is the number of
words of $v_i$; we have $m_0 = 0$.
For us, a \intro{word} is a whitespace-delimited sequence of
characters in the Wiki markup language that produces a Wikipedia article:
we work at the level of such markup language, rather than at the level
of the HTML produced by the wiki engine.

