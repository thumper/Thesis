
A first iteration for a content-driven reputation system
might start with the simplest of rules: the more content
an author contributes, the more reputation they gain.
To measure the content added by an author during some edit,
we compare the resulting version to the version immediately before,
using a basic difference algorithm~\cite{Myers86,TichyEditDist,BurnsLong97}.
The \intro{diff algorithm} will return a list of insertions
and deletions that transform the previous version into
the new version, and we can attribute the insertions
as contributions by the new author.

The underlying basis for both our reputation and trust systems
is an algorithm which attributes each word to some user.
Insertions of new words, and deletions of old words,
are easy to trace, but words that are copied are a problem.
For example, consider this section of text:
%
\begin{quote}
Since 27 November 2001, the economist Anders Fogh Rasmussen
has been Prime Minister to Denmark~\cite{wiki:Denmark-Fogh}.
\textit{As Prime Minister to Denmark, the economist Anders Fogh Rasmussen
leads the government with the consent of Queen Margrethe II.}
\end{quote}
%
The italicized text has been added, but it primarily repeats
what appears in the previous sentence.
Clearly, a model of attribution which simply tracks insertions
and deletions is too simple, as the author of the italicized
text would receive credit for words that are not his own.
In a similar vein,
regular maintainers may accidentally or purposefully receive
false credit.
A common phenonemon in the Wikipedia is the emergence
of a \intro{caretaker} for an article, a user who takes
special interest in the article and devotes much time
to revising and adding text to it.
On occasion, however, the caretaker will take exception to
anyone else editing the article, and completely revert all other edits.
At this point, the caretaker could then reintroduce
the same text, thereby receiving credit for the addition.

Instead, a better model of attribution would give credit
to the author who originally created the text,
regardless of edit wars.
For example, the WikiTravel site creates a tree representing
the version history of an article: two consecutive versions
have a parent-child relationship in the tree, except when
the second version is identical to an earlier version;
versions that are identical are merged into a single node
in the tree (which is attributed to the author of the earliest
such version)~\cite{WikiTravelAuthorship}
(see~\cite{Sabel2007} for a similar variation on organizing revisions).
Using this revision tree, the WikiTravel site computes the
authors of the article as being the set of authors starting
from the most recent version and following parent links
to the root of the tree.
Unscrupulous authors can subvert this system as
well, but must put some effort into changing
the wording around a little.

Historians of computer science will note a relation
to \textit{transclusions}~\cite{Nelson81}.
Nelson's vision for hypertext included the notion
of micropayments to authors, which required detailed,
and manual, attributions of text.
Our solution automatically detects attribution
(equivalently, transclusions of portions from earlier revisions)
in the context of a revisioned document edited
by multiple authors.


