
\section{Introduction}
\label{sec:diff-intro}

The Wikipedia is not unique, either as a collaborative work or as
a versioned document, although it is uniquely the largest such work
in colloquial language.
Programming projects (especially open source projects)
are also collaborative and versioned --- which makes for
an interesting analogy because many of the same questions can be asked:
\begin{enumerate}
\item How fast did the project grow?
\item Who contributed how much to the project?
\item How much do the different components interact?
\item How do the components evolve over time?
\item How many components are there in the project?
\end{enumerate}
Of these questions, we thought to focus on determining who contributed
how much to the project; as a user contributes more, they gain experience
in the community mores and our belief is that their future
contributions will be more valuable.
That is, our notion was to give credit to users for the words that
they write.

In the field of software engineering, the question of ``who contributed
how much'' is generally answered by the metric
\textit{Source Lines of Code} (SLOC).\footnote{\url{http://en.wikipedia.org/wiki/Source_lines_of_code}}
Computing who added which lines of text is achieved
by applying a basic text difference
algorithm~\cite{Myers86,TichyEditDist,BurnsLong97} to two consecutive
revisions.
The output of the algorithm allows us to determine what text was added and
deleted at each edit, and attribute the insertions as contributions by the
author of the second revision.
In principle, the same technique works in the case of tracking authorship
on the Wikipedia
\ldots\quad
They say fools rush in, and we did.

There are a few problems with this initial model.
The most obvious is that \textit{lines} are too granular
a unit for tracking the evolution of English prose;
editors frequently come in and revise phrases or
even specific words, so it's necessary to compute
differences at the granularity of \textit{words}.
A bigger problem is that text isn't only inserted
or deleted; it can also be copied or moved around.
There are variations of the text difference algorithms that can
detect moved and copied text, but to whom should we attribute this text,
the author who created the original text, or the editor who rearranges it
into its final form?

To answer this last question, let us consider a fairly extreme example.
Suppose user Alice creates a new article and writes a page of text for it.
Vanessa is a vandal and decides to \intro{blank} the article,
\ie, she deletes all the text for the article, but not the article itself.
Robert is part of the RCPatrol, and he notices that the article has
been tampered with, so he restores all the text back to Alice's version.
Now who should receive credit for writing the text, Robert or Alice?

We call this the \intro{author attribution} problem,
and in the context of the Wikipedia it seems obvious that Robert's
work is valuable maintenance, but Alice is the true ``author'' of the text.
Thus, a better model of attribution would give credit
to the author who originally created the text,
regardless of edit wars.
For example, the WikiTravel site creates a tree representing
the version history of an article: two consecutive versions
have a parent-child relationship in the tree, except when
the second version is identical to an earlier version;
versions that are identical are merged into a single node
in the tree, which is attributed to the author of the earliest
such version~\cite{WikiTravelAuthorship}
(see~\cite{Ekstrand2009,Sabel2007} for similar variations
on organizing revisions).
Using this revision tree, the WikiTravel site computes the
authors of the article as being the set of authors starting
from the most recent version and following parent links
to the root of the tree.
(Note that WikiTravel computes a set of authors as a requirement
of the license agreement that applies to contributed content,
so the problem is more than an academic one.)


A second example is now easier to analyze for attribution:
%
\begin{quote}
Since 27 November 2001, the economist Anders Fogh Rasmussen
has been Prime Minister to Denmark~\cite{wiki:Denmark-Fogh}.
\textit{As Prime Minister to Denmark, the economist Anders Fogh Rasmussen
leads the government with the consent of Queen Margrethe II.}
\end{quote}
%
The italicized text has been added, but half the content repeats
what appears in the previous sentence.
Clearly, a model of attribution which simply tracks insertions
and deletions is too simple, as the author of the italicized
text would receive credit for words that are not his own.


\mynote{Fill in, what is our proposal?}


Historians of computer science will note a relation
to \textit{transclusions}~\cite{Nelson81}.
Nelson's vision for hypertext included the notion
of micropayments to authors, which required detailed,
and manual, attributions of text.
We propose automatically detecting attribution
(equivalently, transclusions of portions from earlier revisions)
in the context of a revisioned document edited
by multiple authors.


\mynote{Move to later section}
These examples highlight some guiding principles of author attribution that
we choose to adopt:
\begin{enumerate}
\item Common words (\eg, ``to,'' ``on,'' ``of the'') shouldn't be
    ascribed to a single author.
\item The user who initially inserted the text is the original author.
\item Added text should be compared against old revisions, in case it is
being reinstated from the past revision.
\end{enumerate}


