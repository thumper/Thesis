\section{Related Work}
\label{sec:diff-related}

Finding the difference between two strings is a long-studied problem,
known as the \intro{string-to-string correction} problem~\cite{Wagner1974}.
Initial work on this problem revolved around finding the
Longest Common Subsequence (LCS)~\cite{Gusfield1999}, which identifies the
sequence of symbols\footnote{Note that the symbols need not be sequential.}
in common between the two strings.
From there, it is easy to identify the optimal
set of insertions, deletions, and replacements to transform
one string to the other.
The well-known UNIX \texttt{diff} utility~\cite{Hunt1976} is based
on this algorithm.

Tichy advances the idea that finding the shortest edit script is a
useful solution to the string-to-string correction problem, and
introduces \intro{block moves} as a useful operation in edit
scripts~\cite{Tichy1984}.
This takes advantage of the fact that strings may include internal
replication of symbols.

Myers shows that LCS and shortest edit script are equivalent to finding
different paths in an edit graph~\cite{Myers1986}, and develops a
``greedy'' solution to the problem.
Burns and Long~\cite{Burns1997} extend~\cite{Reichenberger1991},
which combines the greedy technique with Tichy's solution,
obtaining improved performance characteristics.

Specific to the Wikipedia, Fong and Biuk-Aghai present a different
dimension of the string-to-string correction problem~\cite{Fong2010}.
In previous formulations, solutions are optimized for abstract
performance characteristics (\eg running time or
edit distance~\cite{Levenshtein1966}); but the Wikipedia computes text differences
to present for human understanding.
Fong and Biuk-Aghai apply the hierarchical differencing idea of
Neuwirth~\etal~\cite{Neuwirth1992} to the differencing algorithm used by the
WikiTrust project~\cite{Adler2007}, and additionally classify components
of the edit script according to common behaviors of Wikipedia editors.

Historians of computer science will note a relation
to \textit{transclusions}~\cite{Nelson81}.
Nelson's vision for hypertext included the notion
of micropayments to authors, which required detailed
and manual attributions of text.
We propose automatically detecting attribution
(equivalently, transclusions of portions from earlier revisions)
in the context of a revisioned document edited
by multiple authors.


