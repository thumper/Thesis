\section{Tracking Text Evolution}

By ``text evolution,'' we mean noting how text shifts
around and is added to or deleted from --- that is,
understanding how it is changed from version to version
by examining the differences.
This is exactly the output from algorithms solving the
string-to-string correction problem.

The simplest string-to-string correction algorithms
note only insertions and deletions, but we are also interested
in noting that text has been copied (or even just moved)
to another part of the article so that we can assign authorship
correctly.
In fact, there are several properties that we think are desirable
when determining how text is reorganized.
\begin{itemize}
\item When text is duplicated within an article, we prefer to assign
    authorship to the author of the original copy.
    This prevents a vandal from duplicating text and then deleting the
    original copy.
    To achieve this goal, we must use an algorithm that allows
    block moves of text, and allows the same source text to be
    matched multiple times in the target revision.
\item If the same text is found multiple times in both the previous
    and new revision, there are multiple ways to describe how the
    text has evolved.
    We would like to give preference to the most plausible explanation:
    that the text was not rearranged.
    To do so, we give prefer to match chunks of text that are in the
    same relative order in their respective document
    versions.\footnote{This might seem irrelevant, given that duplicated
    text is given the same author.  This property becomes important
    when computing the edit distance from one version to another,
    where text in the same relative position does not increase the
    edit distance.}
\item We don't want to over-reward the first author to use common
    words (\eg ``the,'' ``of'').
    To simplify our algorithm, we ignore matches that are less than
    four words in length.\footnote{A better solution would use
    $n$-gram frequencies to decide when a match was suitable.}
\item When text is deleted in one version and then restored in a later
    version, we prefer the original author.
    This prevents edit wars or vandalism from disrupting the
    authorship of text.
\item When looking for matches, we prefer to match against chunks
    of text which have been \textit{live} most recently.
    That is, when matching against text from multiple revisions,
    we prefer to match against the most recent revision that is
    a good fit.
    This property accounts for editors who review the recent history
    of an article and restore text that was deleted, as often happens
    when vandalism occurs.
\item We prefer longer matches.
\end{itemize}

