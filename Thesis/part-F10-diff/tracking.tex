\section{Tracking Text Authorship}
\label{sec:diff-tracking}

To answer the question of which author contributed what text to a
collaborative document, we present first the general procedure to
perform the calculation.
We assume, for this discussion, a single article $\article{} \in \articles$,
with $n > 0$ revisions given by
\begin{equation*}
\versions{\article{}} = [ \version{1}, \version{2}, \ldots, \version{n} ].
\end{equation*}
We define the content of version \version{i}
as being a sequence of $l_i \ge 0$ words for all $0 < i \le n$:
\begin{equation*}
\words{\version{i}} = [ w_1, w_2, \ldots, w_{l_i} ].
\end{equation*}


At an abstract level, we can discover the inductive step by examining
the first several versions.
Clearly, for \version{1}, we have the author of the edit,
\revauthor{\version{1}}, as the author of each individual word.
To track the authorship of words in \version{2}, there are two cases:
\begin{enumerate}
\item A sequence of words in \version{2} also exists in \version{1}.
	In this case, we retain the original authorship of the words,
	\revauthor{\version{1}}.
\item A sequence of words in \version{2} does not also exist in \version{1}.
	In this case, the sequence must have been inserted by
	\revauthor{\version{2}} and we assign authorship accordingly.
\end{enumerate}
Word authorship in \version{3} is similar to the situation in
\version{2}, with an additional case:
\begin{enumerate}
\item A sequence of words in \version{3} also exists in \version{2}.
	In this case, we retain the original authorship of the words
	that was determined for \version{2}.
\item A sequence of words in \version{3} does not also exist in \version{2},
	but does exist in \version{1}.
	Again, we retain the original authorship of the words, as it was
	determined for \version{1}.
\item A sequence of words in \version{3} does not exist in any previous
	revision.
	This sequence must have been inserted by \revauthor{\version{3}}.
\end{enumerate}

The general flavor of the computation is now clear, but to describe it
more precisely we need some additional definitions.
For a given revision, we need to know the ordered list of earlier
revisions:
\begin{equation*}
    \prevrevs{\version{i}} = \left[ \version{j} \colon  0 < j < i \right]
\end{equation*}
And given some word $w_r$ from \version{i}, we need to compute the
location of the best match from a list of earlier revisions:
\begin{equation}
    \match{\version{i}, r, \prevrevs{\version{i}}} =
    \begin{cases}
	(\version{k}, s) & \text{if there exists a best match $w_s$ from \version{k}.} \\
	\emptyset & \text{if there is no match.} \\
    \end{cases}
\label{eq:bestmatch}
\end{equation}
where $k < i$ since we are only interested in matches with earlier revisions;
we will better describe this calculation in the next section.

Now it is possible to define the author of each word as a recursive relation.
Recalling $\words{\version{n}} = \left[ w_1, w_2, \ldots, w_{l_n} \right]$, let $0 < j \le l_n$ in the following definition:
\begin{equation}
\txtauthor{\version{n}, j} =
    \begin{cases}
	\txtauthor{\version{k}, s}, & \text{ if }
	\match{\version{n}, j, \prevrevs{\version{n}}} = (\version{k}, s) \\
	\revauthor{\version{n}}, & \text{ if there is no best match text. } \\
    \end{cases}
\label{eq:txtauthor}
\end{equation}

