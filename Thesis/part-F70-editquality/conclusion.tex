\section{Conclusions}

We propose two measures of revision quality computed
from Wikipedia's revision history.
The measure \textit{text longevity} is based on an intuitive
model of computing the text added by authors at each revision
and detecting how much of that text remains within the article
in subsequent revisions; to account for the variation in
the amount of preserved text over the subsequent revisions,
we model the change as a geometrically decaying process
and compute the decay rate as a single value to describe
the variation.
The measure \textit{edit longevity} was developed to address
the reality that authors also delete and rearrange text,
and that these are valuable contributions to the Wikipedia.
Using edit distance~\cite{Levenshtein1966} to describe the
amount of \intro{effort} that an author puts into making a
revision to an article; we use this as the basis for computing
edit longevity by estimating the amount of effort by an
author that brings the article text closer to some future
version of the article.

We evaluate these two measures using the PAN-WVC-10 dataset, which is
manually annotated to indicate which revisions are vandalism and which
are well-intentioned edits, and treat each as a predictor of vandalism.
We find that edit longevity performs much better than text longevity,
but even text longevity does better than chance at predicting vandalism.
Overall, these results are encouraging for using edit longevity and text
longevity as signals for inferring the community feedback of an author's
edit.  Knowing the quality of edits, we can build an author reputation
system upon these signals; we describe such a system in
Chapter~\ref{ch:reputation}.


\subsection{Triangle Inequality}


Whatever weight we choose for insertions and deletions in the edit
distance computation, we know that the author took some
\textit{explicit} action to effect the insertion or deletion.
The same is not true for the Move operation: our use of
Tichy's block moves~\cite{Tichy1984} obscures the information
about what text stays in the same relative order in the
transformation from the old revision to the new revision.

\mynote{TODO: Continue this line of thinking and reference
Sankoff and discuss reverse triangle inequality.}

