\section{Conclusions}

We propose two measures of revision quality computed
from Wikipedia's revision history.
The measure \textit{text longevity} is based on an intuitive
model of computing the text added by authors at each revision
and detecting how much of that text remains within the article
in subsequent revisions; to account for the variation in
the amount of preserved text over the subsequent revisions,
we model the change as a geometrically decaying process
and compute the decay rate as a single value to describe
the variation.
The measure \textit{edit longevity} was developed to address
the reality that authors also delete and rearrange text,
and that these are valuable contributions to the Wikipedia.
Using edit distance~\cite{Levenshtein1966} to describe the
amount of \intro{effort} that an author puts into making a
revision to an article; we use this as the basis for computing
edit longevity by estimating the amount of effort by an
author that brings the article text closer to some future
version of the article.

We evaluate these two measures using the PAN-WVC-10 dataset, which is
manually annotated to indicate which revisions are vandalism and which
are well-intentioned edits, and treat each as a predictor of vandalism.
We find that edit longevity performs much better than text longevity,
but even text longevity does better than chance at predicting vandalism.
Overall, these results are encouraging for using edit longevity and text
longevity as signals for inferring the community feedback of an author's
edit.  Knowing the quality of edits, we can build an author reputation
system upon these signals; we describe such a system in
Chapter~\ref{ch:reputation}.

Examining the performance of \textbf{ed2} in
Appendix~\ref{app:editlong-data}, we note that it was far from the best
performing definition of edit distance in terms of predicting vandalism
in the PAN-WVC-10 dataset.
Although using an edit distance formulation that satisfies the triangle
inequality is desirable, it is not sufficient to achieve good
performance.
The goal of an edit distance function in our context is to estimate
the amount of \intro{effort} that an author expends in creating an
edit from one revision to another.
We chose to follow a model which prefers the longest possible match
between source and target revisions, but another viable route is to
minimize the amount of text which is rearranged (\ie minimize the
number of transpositions)~\cite{Wagner1975}.
We leave as an open question how to best characterize the work
that an author does.

