\section{Related Work}

Some of the literature studying Wikipedia uses the notion of edit
quality as a basis for some other research goal, and so they use the
gross measure of detecting reverts to signal a poor quality
edit~\cite{Adler2007,Smets2008,Itakura2009,Belani2010}.
The WikiTrust project introduces the idea of text and edit
longevities~\cite{Adler2007}, which are finer grained than a
binary classification and are the subject of this chapter.
Similar to the notion of text longevity is the idea
of Persistent Word Revision per
word~\cite{Halfaker2009,Halfaker2011} (PWRpW),
which counts the number of revisions that added words survive
and normalizes that sum by the number of words added (that is,
it computes the average number of revisions that added words survive).
PWRpW differs from text longevity in that it is essentially computing
the area under the curve of text survival (such as that depicted by
Figure~\ref{fig:textsurvival}),
whereas text longevity is trying to model the curve as a
geometric decay (see Figure~\ref{fig:textlongevity}).

At a broader view, quantifying the quality of an edit is strongly
related to the problem of detecting vandalism.
Many machine learning models produce a probability that an edit
should be classified as vandalism, and this probability can be
directly taken as a quality score.
We refer the interested reader to Section~\ref{sec:vandalism-related}
for a discussion of the literature around this view of the problem.

