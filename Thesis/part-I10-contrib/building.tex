\section{Primitives}

In order to construct functions that compute the contributions
of authors, we need some primitives to work with.
Our view is that any sizing up of the work or productivity
of Wikipedia authors must take into account the fact that
not all contributions are positive; approximately 7\%
of edits are vandalism~\cite{Potthast2008,Potthast2010a}.
We propose that besides measuring the magnitude of work done
by an author, to guage the productivity of authors towards the
goal of producing a useful reference requires tempering that
magnitude with a quality measure.
One way to achieve this is to factor the two together
as ``\textit{quantity} $\cdot$ \textit{quality},'' with the
following desireable property for quality: it should be signed,
where positive values are towards a common goal and negative
values are away from that goal.

The quantity of a contribution can be measured in several ways.


We define various measures of author contribution, taking into
account the amount of text added or edits performed by the author
and the quality of those changes.
We would like to measure contributions both in absolute terms,
as the amount of text that was added by an author or the
amount of edits made by an author, and in relative terms, 
where we take into account the quality of the edits.
The contributions of all authors is cumulative over the 
entire revision history of the Wikipedia;
for our experiments, we picked revisions of all articles
previous to October~1, 2006.

For every article, $a \in \articles$, in the Wikipedia,
we consider each version $\version{i} \in \versions{a}$
to be edited by the author $A_i = \revauthor{\version{i}}$.
Each of the subsequent authors $A_{i+1}, A_{i+2}, \ldots$ can either
retain, or remove, the edits performed by $A_i$ in bring
version \version{i-1} to \version{i}.
These authors who edit article $a$ after \version{i}
are implicitly providing feedback on the content of \version{i}, and hence
act as \intro{judges} of the contribution made by author $A_i$.
We therefore define \judges{a}{n}{i} to return a set of up to $n$ versions
after \version{i} that will act as judges of \version{i}.
For this work, we take \judges{a}{n}{i} to be the \textbf{next} $n$
versions such that the author of each $\version{} \in \judges{a}{n}{i}$
is restricted to $\revauthor{\version{}} \ne \revauthor{\version{i}}$.
It should be noted that other formulations of \judges{a}{n}{i}
are possible, such as the one proposed in~\cite{Chatterjee2008}.

Recalling that
\begin{equation*}
    \versions{a} = [ \version{1}, \version{2}, \ldots, \version{n} ],
\end{equation*}
we define $\revpos{\version{i}} = i$;
note that $a$ is implicit in \version{i}, since a revision
is part of the history of some specific article.
Crucially, to define contribution formulas for each user,
we also define the map
\begin{equation*}
\authorrevs{} : \users \times \articles \to 2^\versions{},
\end{equation*}
which, given a user $u \in \users$ and an article $a \in \articles$, returns
the set of revisions that were created by user $u$ for article $a$.

\subsection{Quantity Measures.}

We would like to measure the \textit{size} of an author's contribution
when they create version \version{i} of some article $a \in \articles$.
The most obvious quantity that can be measured is counting how
many words the author added in version \version{i},
which can be computed from the text difference in going from
\version{i-1} to \version{i}.
If we let $I(\version{i-1}, \version{i})$ represent the number
of words that were inserted in going to version \version{i},
our first quantity measure to measure the size of a
\intro{text contribution} is:
\begin{equation*}
\txt{}{\version{i}} = \tsurv{a}{i,i} = I(\version{i-1},\version{i}),
\end{equation*}
which is a specific case of the more general
definition in Equation~\ref{eq:tsurv}.


As we have previously noted, counting only the words added in
an edit ignores the fact that some users of Wikipedia do maintenance
work in the form of rearranging text or removing vandalism.
To capture this extra behavior, we would like to also measure
what words were deleted in the edit, $\version{i-1} \goesto \version{i}$,
as well as how many words were rearranged.
This is known in the literature as the \intro{edit distance} between
versions \version{i-1} and \version{i}.
There are several ways to compute edit
distance~\cite{Levenshtein1966,Tichy1984},
usually based on insertions and deletions of
characters.\footnote{The careful reader will observe that
the definition of \txt{}{\version{i}} is actually an
edit distance as well, albeit one which ignores deletions
and text rearrangement.}
Our formulation is instead based on words as the fundamental unit,
to more closely approximate how people perceive edits.
We define the edit distance in terms of the following quantities:
\begin{itemize}
\item $I(\version{i-1}, \version{i})$ is the number of words that are inserted,
\item $D(\version{i-1}, \version{i})$ is the number of words that are deleted,
    and
\item $M(\version{i-1}, \version{i})$ is the number of words that are moved,
    times the fraction of the document that they move across,
\end{itemize}
which can be computed according to the text differencing
algorithm described in Chapter~\ref{ch:diff}.
The size of the \intro{edit contribution} for \version{i} is
then given by:
%
\begin{equation*}
\dist{}{\version{i}} = \dist{a}{i-1,i} = \max(I, D)
    - \frac{1}{2}\min(I, D) + M,
\end{equation*}
which is a specialization of the more general
edit distance in Equation~\ref{eq:dist}.



\subsection{Quality Measures.}

In addition to the quantity measures defined above, we need
multiple quality measures to choose from.
We start with the two quality measures derived in
Chapter~\ref{ch:editquality}:
\begin{itemize}
\item \quality{tdecay}{10}{i} is the value that best describes
    the text survival of text inserted in version \version{i}
    over the next ten revision as an exponential
    decay, defined in Section~\ref{sec:textquality}.
    This value ranges from 0 for completely removed text,
    to 1 for text which is completely preserved.
\item \quality{elong}{10}{i} is the average edit longevity of
    the contribution of version \version{i} as judged by up
    to ten judges in the immediate future.
    This value ranges from $-1$ for completely reverted edits,
    to $+1$ for completely preserved edits.
    See Section~\ref{sec:editquality} for the development of
    this measure.
\end{itemize}

A third way to measure the quality of a text contribution
is to simply sum the fraction of text that remains over the
succeeding ten revisions:
%
\begin{equation*}
    \quality{tsurv}{10}{i} = \frac{1}{\tsurv{}{i,i}} \cdot
    \left( \sum_{v \in \judges{}{10}{\version{i}}} \tsurv{}{i,\revpos{v}} \right)
\end{equation*}
%
This value generally ranges from $0$ for text which is immediately removed,
to $+10$ for text which completely survives all ten revisions.

