\section{Contribution Measures}

We now present several potential contribution measures in the form of
``\textit{quality} $\cdot$ \textit{quantity},''
using the building blocks described in the last section.

\subsection{Number of Edits}

\noindent
The simplest quantitative measure of contribution for 
authors is to compute the number of revisions they authored.
In previous works, this is referred to as the 
\intro{number of edits} made by an 
author~\cite{Wales2005,Wilkinson2007,Kittur2007,SteinHess2007}.
We follow the tradition, and define this precisely
for some user $u \in \users$ as:
%
\[
\numedits(u) = \sum_{a \in \articles} \sum_{v \in E(u, a)} 1 \cdot 1.
\]
%

\subsection{Text Only}

\noindent
Another very natural measure of author contribution
is to count up how many words were added by each author,
during the course of all their revisions.
Since there is no quality measure involved,
we refer to this measure as \textonly,
and define it for each $u \in \users$ as:
%
\[
\textonly(u) = \sum_{a \in \articles} \sum_{v \in E(u, a)} 1 \cdot \txt{}{v}.
\]
%
We refer to this measure as the {\em absolute\/} text contribution
measure.

\subsection{Edit Only}

\noindent
Correcting grammar, polishing the article structure,
and reverting vandalism are all chores~\cite{Burke2008}
which must be done to keep the Wikipedia presentable.
Counting the number of words added can partially account
for these chores if there is no text authorship tracking done;
without text authorship tracking, however, vandals could
easily subvert any system for measuring contributions.
Instead, we note that measuring the size of the change
in each revision is able to reward both authors who write
new text, as well as authors who polish existing text.
More formally, we measure the edit distance between the
the version of a page that was generated by revision $r_i$
and the version that immediately preceded it.
The \editonly measure is thus defined for all $u \in \users$ as:
%
\[
\editonly(u) = \sum_{a \in \articles} \sum_{v \in E(u, a)} 1 \cdot \dist{}{v}.
\]
%
We refer to this measure as the {\em absolute\/} edit contribution
measure.

\subsection{Text Longevity}

\noindent
The next level of sophistication is to incorporate
non-constant quality measures into the calculation of contribution.
We desire the text longevity of a revision to be
the amount of original text that was
added by the author $\revauthor{\version{i}}$
for a revision \version{i}, discounted by the
text quality measure \quality{tdecay}{}{\version{i}}, which describes
how the text decays over the next several revision.
%
\[
\textlong(u) = 
\sum_{a \in \articles} \sum_{v \in E(u, a)} \quality{tdecay}{}{v}
        \cdot \txt{}{v}
\]
%

\subsection{Edit Longevity}

\noindent
Similar to the text longevity measure, we define the edit longevity
of a revision \version{i} as the edit contribution, discounted by the 
average edit quality measure \quality{elong}{}{\version{i}}.
As with all the measures, we accumulate contributions based on 
edit longevity over all revisions edited by each user $u \in \users$:
%
\[
\editlong(u) = \sum_{a \in \articles} \sum_{v \in E(u, a)} 
\quality{elong}{}{v} \cdot \dist{}{v}
\]
%

\subsection{Ten Revisions}

\noindent
A simpler method for measuring how useful newly inserted
text is, is to simply add up how many words survive over
the next ten revisions.
Large contributions are thus richly rewarded, if they survive;
smaller contributions have a slightly better chance of surviving
for the entire ten revisions, thus encouraging change ---
but not too much change.

We consider the ten revisions that follow any revision $r_i$ of an 
article, and accumulate the amount of text contribution that was made 
in $r_i$ that remained in each of those ten subsequent revisions
of the article.
We call this measure $\tenrevs$ and define it for each
$u \in \users$ as follows:
%
\begin{gather*}
\tenrevs(u) = 
\sum_{a \in \articles} \sum_{v \in E(u, a)} \quality{tsurv}{10}{v}
        \cdot \txt{}{v}.
\end{gather*}

\subsection{Text Longevity with Penalty}

\noindent
A last variation that we propose is to combine text longevity with 
edit longevity in such a way that authors of new content are rewarded, 
but vandals are actively punished for both inserting and deleting text.
Text longevity, as we have defined it, already does not
reward vandals --- vandals either insert no text, or the
text they insert is immediately removed; both cases result
in a text longevity of zero for the revision.
Vandals are still able to accumulate positive contributions
from other revisions, however, while disrupting other
authors with their vandalism.
By only counting edit longevity when it is negative,
we are able to punish vandals for any kind of vandalism
which is reverted.
This leads to the following definition of our punishing 
measure for every $u \in \users$:
%
\begin{gather*}
\punish(u) = \\
\textlong(u) +
\sum_{a \in \articles} \sum_{v \in E(u, a)} \min(0, \quality{elong}{10}{v}) \cdot \dist{}{v}.
\end{gather*}
%

