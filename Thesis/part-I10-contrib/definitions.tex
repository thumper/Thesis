\section{Definitions}

The following notation will be used throughout the paper.
We consider the set $\pages$ of all articles in the main English 
Wikipedia.
We denote the set of authors of pages on the Wikipedia by
$\authors$.
We assume that we have $n > 0$ versions $v_0, v_1, v_2,
\ldots, v_n$ of a page $p$; version $v_0$ is empty, and version 
$v_i$, for $1 \le i \le n$, is obtained by an author performing 
a revision $r_i : v_{i-1} \leadsto v_i$.
Since each revision $r_i$ is performed by one author, we refer
to the author who edited revision $r_i$ as $a_i$.
We denote the set of all revisions of a page by $\revisions$.
We refer to the change set corresponding to 
$r_i : v_{i-1} \leadsto v_i$ as the edit performed at $r_i$ : the 
edit consists of the text insertions, deletions, displacements, 
and replacements that led from $v_{i-1}$ to $v_i$. 
We define the map $E : \authors \times \pages \to 2^\revisions$, 
which given an author $a \in \authors$ and a page $p \in \pages$, returns a
set of revisions that were created by author $a$ for page $p$.
When editing a versioned document, authors commonly save several 
versions in a short time frame.
We filter the versions, keeping only the last of consecutive 
versions by the same author; we assume thus that for $1 \le i \le n$
we have $a_{i-1} \ne a_i$.
Every version $v_i$ for $0 \le i \le n$, consists of a sequence
$[w_1^i,\ldots,w_{m_i}^i]$ of words, where $m_i$ is the number
of words in $v_i$; version $v_0$ consists of the empty sequence.

\smallskip

\noindent\textbf{Quantity Measures.}

Given a series of versions $v_0,\ldots,v_n$ of a page $p$, we assume
that we can compute the following quantity measures:
%
\begin{itemize}

\item $txt(v_i, v_j)$, for $0 < i \le j \le n$, is the amount of
	text (measured in number of words) that is introduced by
	$r_i$ in $v_i$, and that is still present (and due to the same
	author $a_i$) in $v_j$.
	$txt(v_i,v_i)$ is the amount of new text added by $a_i$ through
	$r_i$.
	We define $txt(r_i) = txt(v_i,v_i)$, and refer to this
	as the \intro{text contribution} of $r_i$.

\item $d(v_i, v_j)$, for $0 \le i < j \le n$, is the
	\textit{edit distance} between $v_i$ and $v_j$,
	and measures how much change
	(word additions, deletions, replacements, displacements, etc.)
	there has been in going from $v_i$ to $v_j$.
	We define $d(r_i) = d(v_{i-1}, v_i)$, for the \intro{edit contribution}
	made in a revision $r_i$.
\end{itemize}
%
There are several ways to compute edit distance~\cite{Levenshtein66,TichyEditDist},
usually based on insertions and deletions of characters.
Our formulation is instead based on words as the fundamental unit,
to more closely approximate how people perceive edits.
We define the edit distance in terms of the following quantities:
$I(v_i, v_j)$ is the number of words that are inserted;
$D(v_i, v_j)$ is the number of words that are deleted;
$M(v_i, v_j)$ is the number of words that are moved, times the fraction
of the document that they move across \cite{Adler2007}.
The edit distance between two versions, $v_i$ and $v_j$, is
then given by:
%
\[
d(v_i, v_j) = \max(I, D) - \frac{1}{2}\min(I, D) + M
\]
%
A more precise treatment of this definition is available
in~\cite{Adler2007}, along with reasoning for this particular
choice of edit distance and a discussion of text tracking
for authorship.


\smallskip

\noindent\textbf{Quality Measures.}

In addition to the quantity measures defined above, we define the
following quality measures.
We first consider the edits in a revision $r_i$ made by author
$a_i$.
Given $d(v_{i-1}, v_i)$, the edit distance between versions
$v_{i-1}$ and $v_i$, we would like to associate a higher edit
quality to revision $r_i$ if the edits made take the page
closer to subsequent versions of the page.
For example, if none of the edits made in $r_i$ are reverted
in subsequent revisions, then the edits made in $r_i$
have taken the page in the same {\em direction\/} as subsequent 
versions of the page and hence merit the highest quality measure.
We define $\editquality(v_i, v_j)$, the quality of the edits
performed in revision $r_i$ of a page (with respect to $v_j$) as follows:
%
\[
\editquality(v_i, v_j) = 
\frac{d(v_{i-1}, v_j) - d(v_i, v_j)}{d(v_{i-1}, v_i)}
\]
%
The triangle inequality generally holds, so that
$\editquality(v_i, v_j)$ typically varies from
$-1$ for revisions which are completely reverted,
to $+1$ for revisions which are completely preserved;
when the value falls outside this range, we
cap it to one of these two values.

Due to the occasional vandalism that happens, we prefer to
judge quality using several succeeding versions.
We define a map $J : \revisions \to 2^\revisions$,
such that if $r_i$ is a revision of a page $p$, 
$J(r_i)$ consists of the first ten revisions after $r_i$ that have
author different from that of $r_i$.
If there are fewer than ten revisions after $r_i$ that have author
different from that of $r_i$, then $J(r_i)$ returns all such
revisions. 
We use the versions in $J(r_i)$ as judges of the quality of $r_i$.
We define the {\em average edit quality} $\avgeditquality(r_i)$
of a revision $r_i$ with $|J(r_i)| \neq \emptyset$ as follows:
%
\[
\avgeditquality(r_i) = 
\frac{1}{|J(r_i)|} \cdot \left( \sum_{r_j \in J(r_i)} 
  \editquality(v_i, v_j) \right)
\]
%
Thus, $\avgeditquality(r_i)$ is the average of the \editquality values
determined by each of the judges of $r_i$.

The quality of a text contribution to a page is a function of
how much of the original text was edited out in subsequent
revisions of the page.
If none of the text was removed, then the text quality of revision
$r_i$ should be $1$.
One model for how text behaves over time is that it decays in
an approximately geometric fashion:
the largest chunk that is deleted is right after text is added,
and subsequent revisions remove smaller and smaller chunks
until what is left is the stable core which is generally
accepted by other authors.
We define the text quality measure 
$\textquality(r_i)$ as the solution to the following equation that
expresses text decay in this manner:
%
\[
\sum_{r_j \in J(r_i)} txt(i,j) = 
txt(i,i) \cdot \left( 1 +\sum_{r_j \in J(r_i)} (\textquality(r_i))^{j-i} \right)
\]
%
The resulting value for $\textquality(r_i)$ ranges from $0$
for text which is completely removed, to $+1$ for text which
is completely preserved.

A different way to measure the quality of a text contribution
is to simply sum the amount of text that remains, over the
succeeding revisions.
For a revision $r_i$, by considering the amount of original text 
introduced in $r_i$ that survives in the next ten revisions, we 
define the following additional quality measure for text 
contributions,
%
\[
\textlongevity(r_i) = \frac{1}{txt(i,i)} \cdot
\left( \sum_{r_j \in J(r_i)} txt(i,j) \right)
\]
%
This value generally ranges from $0$ for text which is immediately removed,
to $+10$ for text which completely survives all ten revisions.
(Due to how text tracking works, if a piece of text is copied within
an article, the original author might receive credit for more
words than she originally wrote.)

