\section{Implementation}

As part of our research into author reputation
and text trust~\cite{Adler2007,WikiTrust08}, we have
created a modular tool for processing XML dumps from
the Wikipedia.
It analyzes all the revisions of a page,
filtering down the revisions to remove consecutive edits
by the same author, and computing differences between
revisions to track the author of each word and measure
how the author might have rearranged the page.
These results can be passed to any of several
modules to do additional processing; we use the tool
to reduce the enormous collection of data down to
a much smaller \textit{statistics file}.
We process the statistics file with a second tool,
which we instrumented to calculate the various
contribution measures we have defined.

Our analysis is based on main namespace (\texttt{NS\_MAIN})
article revisions from
the Wikipedia dump of February~6, 2007,
which we process to create a reduced statistics file.
The statistics file contains information about every version,
including the amount of text added, the edit distance from
the previous version, and information about how the edit
persists for ten revisions into the future.
To ensure that each version we considered had revisions
after it, we consider only versions before October~1, 2006.
After further processing on the file,
we used R~\cite{R2007}, an open source statistics package,
to analyze the resulting data.

\textbf{Bots.}
During the course of our analysis, we found that
some authors were extraordinary outliers for multiple measures.
Some investigation into the most extreme cases revealed
that bots were making automated edits to the Wikipedia,
and that a few bots dwarfed manual labor in the edit based
measures $\editlong$ and $\editonly$.
We also found that there are bots that improve content,
and bots that vandalize it.
We chose to identify bots as those with a username which
ends in the string ``bot;''
While this does not include every bot (especially
the ones that vandalize), it is a useful first approximation.
We found $614$ bots in total as of October~1, 2006.

\textbf{Vandals.}
There is a similar problem in trying to define vandals,
since such authors don't register themselves as such.
For our purposes, we decided to define a vandal
as someone who, on average, makes an edit which is
completely reverted.
Precisely, we define a vandal who meets one of two criteria:
$\quality{decay}{10}{} < 0.05$, or $\quality{elong}{10}{} < -0.9$.
We justify this choice in the next section.

