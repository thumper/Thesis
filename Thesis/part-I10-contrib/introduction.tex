
\section{Introduction}

History is filled with examples of partnerships, people collaborating
to create something larger than they could achieve on their own,
but how do you value the contributions of individuals?
How do we say whether Apple benefited more from the technical
designs of Steve Wozniak or the design intuition of Steve Jobs?
The reality is that the very notion of what is a \intro{contribution}
depends on perspective and relative priorities of importance.
\mynote{Citation from economics?}
Was the coding work of Mark Zuckerberg more important than
the initial financial support of \mynote{who} to the eventual
success of Facebook?  \mynote{Citation?  Better example?}
When multiple authors work on a single book, how should the
revenue be split?

Today, online collaboration is growing by leaps and bounds,
fostered under the name \intro{Web~2.0}: ``the actvities of
users generating content (in the form of ideas, text,
videos, or pictures) could be \textit{harnessed} to
create value''~\cite{wiki:Web20}.
In Silicon Valley, we hear examples every day of companies
built atop the contributions of their users:
Google, Facebook, Twitter, Flickr, to name a few famous ones.
For the most part, these sites operate under a motto
of ``more is better,'' and only a few sites try to estimate
a quality of contributions (\eg Amazon and eBay).
These sites are distinct from the Wikipedia, because although
users generate content, and even collaborate in some sense,
they don't actually work on the same content.

Within the Wikipedia, there has been some discussion
about measuring contritbutions~\cite{Wales2005,Swartz2006},
in the context of whether some restrictions would improve
the overall quality of the Wikipedia.
Another motivation for understanding contributions by
users is for attribution purposes: the Creative Commons license
that the Wikipedia content is available under requires attribution
of all authors, which is currently taken to include spammers
and other vandals.
When wikis~\cite{Leuf2001} are used in a corporate setting,
measuring contributions can be a proxy for the productivity
of workers.

In this chapter, we examine multiple ways that contributions
can be defined within the Wikipedia.
We explore the distribution of users under these different metrics,
and make some observations.

