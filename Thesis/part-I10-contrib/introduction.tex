
\section{Introduction}

On-line collaboration is fast becoming one of the primary ways in
which content is created and shared.
From open-source software projects, to on-line encyclopedias such as
the Wikipedia, open on-line collaboration taps into the knowledge,
time, and resources of millions of people, and it enables content
creation on a speed and scale never seen before.

In such collaborative systems for content creation, a relevant 
problem is how to measure the contributions of individual authors. 
This problem arises in several contexts.
In the Wikipedia, it may be desirable to have a measure of how much
work various users have performed, as an aid in deciding how to
distribute promotions and honors (such as Barnstars). 
In a corporate setting, a measure of user contribution can be used to
promote the use of collaboration tools such as wikis, while ensuring
that the contributions of the individual employees can be duly
recognized. 
In wikis that generate revenue, measures of author contribution can be
used as a basis for deciding revenue sharing among authors. 
In this paper, we propose and analyze several quantitative measures of
author contribution for versioned bodies of textual information that
evolve via successive modifications, or {\em edits,} performed by
individual authors. 
We focus on wikis, and in particular on the Wikipedia, an on-line
encyclopedia built on wiki technology \cite{Wikis01}. 

On the Wikipedia, the problem of measuring author contributions has
so far been considered mainly in the context of gaining a better
understanding of the dynamics of how authors contribute to the
Wikipedia. 
In particular, measures of author contribution have been used to
discuss the issue of whether it is a large group of novice users, or a
small group of experienced editors, who contribute most of the
content of the Wikipedia \cite{Wales2005,Swartz2006,Bourgeoisie2007}. 
In these discussions, author contribution was measured via the number
of edits performed by authors (edit count) \cite{Wales2005,EditsEqQuality2007,
Bourgeoisie2007,WikiDashboard2008,OrtegaBarahona2007,SteinHess2007},
or by the total amount of text the authors introduced (text count)
\cite{Swartz2006}. 
We argue that neither of these two measures is robust, or fully
informative. 
Both edit count and text count can easily be gamed. 
In the case of edit count, an author can increase her edit count
simply by doing a small modification, then undoing it (perhaps with an
accompanying message of apology).  
These small changes can be spread across the millions of pages of the
English Wikipedia, to make detection harder. 
Fighting this kind of abuse requires time-consuming human
intervention, as it requires the examination of the edit log performed
by individual users. 
In addition, adopting edit count as a measure of contribution,
and basing important decisions on such a measure, would create a strong
incentive towards this kind of tampering, with negative consequences
for the stability and quality of Wikipedia content. 
Text count can be gamed in a similar, if not easier, fashion, as it is
possible to introduce in a single edit a very large amount of text,
which is then promptly removed in a subsequent edit. 
These measures also under-estimate the contributions that particular
groups of authors make to the Wikipedia. 
Edit count fails to adequately recognize authors who sporadically provide
large amounts of bulk text (thereby contributing to the basic information
build-up).
Text count fails to properly recognize the contributions of those
users who mostly re-arrange existing content, remove vandalism, and
perform other fundamental maintenance duties that involve little new
text creation. 

To provide a more precise, and more robust, measure of author
contribution, we introduce several measures that capture not only the 
{\em quantity\/} of contributions (how many edits, how much text),
but also the {\em quality\/}, analyzing how long the contributions last
as content evolves. 
We analyze the trade-offs involved in the different measures, and the
differences in the author rankings they produce. 
We emphasize that we make a distinction between
contribution and reputation; the first is meant to measure productivity,
and the second, reliability.
A reputation system generally incorporates some measure of productivity,
as we do in~\cite{Adler2007}.
This work explores different notions of productivity,
particularly as a tool for analyzing user behavior,
but does not evaluate the effectiveness of any measure
as a reputation system.

Of the measures we explore, our favorite measure is {\em edit longevity.\/}
The measure combines the amount of change performed by an author, with
how long the change lasts. 
To compute the edit longevity of an author, we consider all the edits
performed by the author. 
For each edit, we compute both the {\em edit size\/} and the 
{\em edit quality.}
The edit size is measured as the edit distance between the article
revision produced by the author, and the previous revision; the
problem of computing edit distances has been studied in 
\cite{EditDist74,TichyEditDist,EditDistanceMoves}. 
The edit quality is in the interval $[-1, 1]$, and measures how long
the change lasts in the system: it is close to $1$ for edits that are
preserved fully in subsequent revisions, and it is close to $-1$ for
edits that are reverted \cite{Adler2007}. 
The edit longevity of an author combines these quantities, and is
computed as the sum, over all the edits performed by the author, of
the edit size multiplied by the edit quality. 
We show that edit longevity has several desirable properties. 
It cannot be easily gamed, since the quality of an edit by author $A$
depends on how long authors {\em different from $A$\/} preserve the
edit done by $A$. 
Edit longevity is sensitive to the size of contributions, giving
appropriate reward to authors who sporadically contribute large
amounts of text. 
Edit longevity also rewards the authors who mostly engage in
improvement and maintenance work, as edit distance, measures not only
new text, but also text deletions and displacements \cite{Adler2007}. 
Furthermore, edit longevity successfully prevents vandals and spammers
from accruing contributions. 

In addition to edit longevity, some other measures we introduce may
serve a purpose in specific contexts. 

{\em Text longevity\/} is a measure of how much text has been
introduced by an author, in which each text contribution is weighed
according to how long the text lasts in subsequent revisions. 
This measure has the advantage of a very obvious definition: while the
estimation of edit quality requires agreement on a particular formula
based on edit distances, the estimation of text longevity requires 
simply, the ability of tracking text through revisions. 
In contexts where revenue must be divided, a simpler definition has
the appeal of being easier to define; for instance, in a legally
binding contract. 
Text longevity, however, has two drawbacks. 
The first is that text longevity fails to adequately reward authors
who mainly engage in maintenance edits, inserting little new text. 
The second is that text longevity fails to penalize spammers and
vandals, assigning to them the same low amount of positive
contribution reserved for novices. 

{\em Text longevity with penalty\/} rewards authors in proportion to
the text they insert, but in addition to this reward, authors of
reverted contributions (that is, edits with negative quality) also
accrue large negative ``fines''. 
Text longevity with penalty is particularly effective in
distinguishing productive members of the author community from vandals
and spammers. 
The drawback of text longevity with penalty, compared with edit
longevity, is that it fails to adequately reward authors who perform
large amounts of maintenance work. 

\subsection*{Related Work}

Measuring contributions of individuals in group collaborative
efforts have been studied for several decades, in the context of
software project management.
Most programmers are familiar with the KLOC (thousand lines of 
code) measurement~\cite{MITRE1988, SLOC1992} ---
it counts how many lines of software a programmer
writes per week, and is analogous to text count.
Although intended to measure the progress of a project, it also implicitly 
measures each programmers contribution to the project.
% Luca: cut
% To incorporate quality into the KLOC measure,
% it has been suggested to count the number of
% defects that occur per section of code.

The problem of measuring author contributions to the Wikipedia first
arose during a debate on which group of authors was chiefly
responsible for Wikipedia content. 
A count of edits indicated that a small group of dedicated authors
gave the largest contribution~\cite{Wales2005}; this result was later
disputed in \cite{Swartz2006}, where it was argued that the majority
of the content was due to the large number of authors who perform only
occasional contributions to the Wikipedia. 
Kittur et al., \cite{Bourgeoisie2007} discovered that the
percentage of edits made by the masses is larger and growing 
when compared to the authors who are either sysops or have a 
very large number of edits to their credit.
Burke and Kraut use a wide variety of edit counts to predict
the authors to be promoted to the status of Wikipedia 
administrators~\cite{AdministratorMop2008}.
There are many works that measure user contributions by the
number of edits which authors make to an 
article~\cite{Wales2005,EditsEqQuality2007,Bourgeoisie2007,
WikiDashboard2008,OrtegaBarahona2007,SteinHess2007}.
In \cite{Bourgeoisie2007}, they consider the number of edits
as well as the change in edits and state that their conclusion
remains unaffected with either of those measures.
Wilkinson and Huberman show that the quality of an article improves
with the number of edits and the number of distinct authors that 
revise that article~\cite{EditsEqQuality2007}.

Another approach consists in measuring contributions indirectly, by
noting citations~\cite{PageRank98,AckCounting2004,WikiMTWtrust06}.
While this approach enables one to judge the relevance of a complete
article, it cannot easily be used to ascribe the merit of the
article to its individual contributors. 
In~\cite{KPB2006}, the authors examine the sequence of edits
to an article and build a social network of authors based
on the amount of citation (how much text is removed or preserved)
by later authors; this social network is then analyzed to derive
authority values for authors. 
%Luca: cut
% This technique is similar to our proposed \textit{longevity}
% measure, but is used to ascribe authority values to authors
% and not to individual edits.

In~\cite{CognitiveSurplus2008}, the total amount of work that went
into the making of the complete Wikipedia is estimated to be about 
100 million hours of human thought. 
This measure is very different from the measures we
propose in this paper, as it is based on the effect on
the author (the amount of time required of the author to contribute),
rather than on the effect on the wiki (how large the
contributions are, and how long they last). 


