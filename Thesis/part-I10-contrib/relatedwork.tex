\section{Related Work}

From our point of view, measuring contributions to a collaborative
work seems most like the software engineering practice of counting
source lines of code to estimate programmer effort and
productivity~\cite{Schultz1988,Park1992}.
There are several other productivity measures that have
roots in the manufacturing process~\cite{Diewart2005},
such as measuring the number of defects or customer
satisfaction~\cite{Tennant2001}.

Within the Wikipedia, the problem of measuring contributions
seems to have first arisen in the context of trying to understand
the process by which knowledge is accumulated and organized
in such a large group collaboration (a discussion informed
by such works
as~\cite{Butler2002,Benkler2002,Surowiecki2004,Reagle2004}).
Wales conducts a survey in December~2004 which finds that half
the edits within the Wikipedia are made by only 2.5\% of logged
in users~\cite{Wales2005}.
Swartz challenges that \intro{edit counts} capture only one part of the
story; counting the size of edits presents a different picture,
and this has policy implications for the Wikipedia in how
it decides to encourage more contributions~\cite{Swartz2006}.

In trying to ascribe a source to the many contributions
that make up the Wikipedia, Anthony~\etal measure the
\textit{survivability} of an edit by looking at the percentage of
characters retained in later edits~\cite{Anthony2005}.
The authors use the entire content of the version being
evaluated (since the author could have made edits to any
part of the content), which distinguishes their metric
from those we developed in Chapter~\ref{ch:editquality}
that are designed to track only the changes done specifically
by the author being evaluated.
Given that caveat, they find that ``Good Samaritans'' (one-time
anonymous users) have the highest quality contributions overall.

Kittur~\etal take up the question of whether \textit{elite}
or \textit{common} users contribute more content, analyzing both
the number of edits made by authors and the total size of the
edit differences~\cite{Kittur2007}.
Their data suggests that both metrics point to the same
conclusion: that elites dominated content-generation in the early
history of the Wikipedia, but the workload had shifted to
the common users by mid-2006.
The opposite conclusion is reached by Ortega~\etal,
who revisit the question of how contributions (as measured by
edit counts) are distributed over the Wikipedia user base and use
Gini coefficients to quantify the concentration of
core contributors~\cite{Ortega2008}.

Many other works exist that are, in the abstract, considering
author contributions.  In practice, we find that most works
use edit count as their contribution
metric~\cite{Wilkinson2007,Burke2008,Suh2008,Ortega2007,Stein2007,Ortega2009}.
We suspect that this is due to the relative simplicity of
counting edits versus computing text differences.

Our own work differs from previous research in two important ways.
First, we propose that there are multiple measures that
can represent the ``size'' of a contribution.
Second, and more importantly, we observe that not all
contributions of the same size are equivalent.
As an example, consider the addition of the sentence ``UC Santa
Cruz rocks your socks'' to an article.
This contribution of six words is considered equivalent to other
legitimate contributions, but actually is a \intro{negative contribution}
in that it creates more work for some other author.
We explore variations of both these parameters.

