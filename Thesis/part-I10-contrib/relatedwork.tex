\subsection{Related Work}

Measuring contributions of individuals in group collaborative
efforts have been studied for several decades, in the context of
software project management.
Most programmers are familiar with the KLOC (thousand lines of 
code) measurement~\cite{MITRE1988,Park1992},
sometimes abused to measure productivity.
Although intended to measure the progress of an
entire project, it also implicitly 
measures each programmers contribution to the project.
What we propose in this paper can be thought of as combining
KLOC with defects/KLOC, so that merit is a function of both
amount of contribution and quality.

Shirky and Martin Wattenberg at IBM did a back of the
envelope calculation, and estimated that the complete Wikipedia
works amount to about 100 million hours of human
thought~\cite{CognitiveSurplus2008}.
This is interesting, because they are measuring work
in terms of the amount of human thought that goes into it.
(We interpret ``Human thought'' as being similar 
to ``man hours.'').

The debate that started it all!
Jimbo Wales said that most of Wikipedia is created by
a handful of people~\cite{Wales2005}.
We are not sure if he still believes this!
Aaron Swartz decided to count by characters,
and discovered that content is mostly added
by infrequent authors~\cite{Swartz2006}.
Kittur et al., \cite{Kittur2007} discovered that the
percentage of edits made by the masses is larger and growing 
when compared to the authors who are either sysops or have a 
very large number of edits to their credit.
Burke and Kraut use a wide variety of edit counts to predict
which authors be promoted to Wikipedia 
administrators~\cite{AdministratorMop2008}.
There are many works that measure user contributions by the
number of edits which authors make to an 
article~\cite{Wales2005,EditsEqQuality2007,Kittur2007,
WikiDashboard2008,OrtegaBarahona2007,SteinHess2007}.
In \cite{Kittur2007}, they consider the number of edits
as well as the change in edits and state that their conclusion
remains unaffected with either of those measures.

Wilkinson and Huberman show that the quality of an article improves
with the number of edits and the number of authors that revise
that article~\cite{EditsEqQuality2007}.

An interesting direction is the one where a group defines the 
\textit{impact} of an article as the number of times it is 
viewed~\cite{WikiImpact2007}. 
They stitch together several different data sources to build a model 
which estimates this number for any point in time.
This has implications on reputation systems and in systems such as
ours where we try to measure contribution.

Contributions (or maybe reputation) can sometimes be measured 
indirectly, by noting 
citations~\cite{PageRank98,Giles2004,WikiMTWtrust06}.
Our technique is content-driven, and uses text tracking
to determine the author of each individual 
word~\cite{RankingControversies2008,Adler2007}.
The basic unit of measure for the amount of text added or the
amount of edit performed is a word in our system.
Our objective is defining a set of contribution measures
that could be used to measure contributions quantitatively
and to compare them.

\mynote{This list was commented out.}
\begin{itemize}

\item Measuring contributions to group collaborations has been
	studied for several decades, in the context of
	software project management.
	Most programmers are familiar with the KLOC
	(thousand lines of code) measurement~\cite{MITRE1988,Park1992},
	sometimes abused to measure productivity.
	Although used to measure performance, it also
	implicitly measures each programmers contribution
	to the project. 
    Our inclusion of quality is akin
	to combining KLOC with defects/KLOC.

\item Shirky and Martin Wattenberg at IBM do a back of the
        envelope calculation, and estimate that the complete Wikipedia
	works amount to about 100 million hours of human
	thought~\cite{CognitiveSurplus2008}.
	This is interesting, because they are measuring work
	in terms of the amount of human thought that goes into it.
%	(``Human thought'' is probably the same as ``man hours.'')

\item The debate that started it all!
	Jimbo Wales says that most of Wikipedia is created by
	a handful of people~\cite{Wales2005}.
	Aaron Swartz decides to count by characters,
	and discovers that content is mostly added
	by infrequent authors~\cite{Swartz2006}.

\item Contributions (or maybe reputation) can sometimes be measured indirectly,
	by noting citations~\cite{PageRank98,Giles2004,WikiMTWtrust06}.

\item Burke and Kraut use a wide variety of edit counts to predict
	which authors be promoted to Wikipedia administrators~\cite{AdministratorMop2008}.

\item Some previous works measure user contributions by the number
	of edits which authors make to an
	article~\cite{Wales2005,EditsEqQuality2007,Kittur2007,
	WikiDashboard2008,OrtegaBarahona2007,SteinHess2007}.

\item Wilkinson and Huberman show that there is a relation between
	the number of edits and the quality of an article~\cite{EditsEqQuality2007}.

\item One group defines the \textit{impact} of an article as the number
	of times it is viewed~\cite{WikiImpact2007}.  They stitch together
	several different data sources to build a model which estimates this number
	for any point in time.  
	%Our reputation system should probably include
	%something like this.

%\item Our technique is content-driven, and uses text tracking
%	to determine the author of each individual word~\cite{RankingControversies2008,Adler2007}.

\item Cosley et al. concern themselves with how to increase contributions
	and maintain high quality through \textit{intelligent task routing},
	which attempts to improve user satisfaction with their
	participation~\cite{TaskRouting2006}.
	This is totally different than measuring contributions,
	but it could be argued that we are measuring to
	increase competition, and thereby increase the quantity and
	quality of contributions.  \mynote{Weak?}

\item We used R~\cite{R2007}.

\end{itemize}
