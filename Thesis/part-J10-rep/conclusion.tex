\section{Conclusions}

\mynote{Need to finish this section.}

We have proposed a reputation system that is
based on metric derived from the content evolution
of a versioned document.

The most significant challenge of this part of our work
was in developing evaluations to measure our performance.

As mentioned previously, our system has low precision because,
as part of our attempt to prevent vandals from avoiding a
negative reputation by reregistering,
we assume that new authors are equivalent to proven offenders.
On reflection, I think that an alternative is to assign an
initial (low) reputation based on information from vandalism
detection models such as those discussed in Chapter~\ref{ch:vandalism}.
This follows the idea that reputation is predictor of future
behavior based on past behavior; when there is no past behavior
of the individual, there is still past behavior of other users
that are somehow similar to the user in question.

Reputations should be allowed to go negative.
Although a motivated vandal can reregister for a fresh account,
they might not be sophisticated enough to do so.
Or, a bot might go bad, and a negative reputation
would identify it more quickly to administrators.

