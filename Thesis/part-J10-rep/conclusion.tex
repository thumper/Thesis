\section{Conclusions}

In this chapter, we propose a reputation system for authors to allow us
to make an educated guess at the quality of a revision when it is first
made.
This reputation system is build atop notions developed as quality
measures in Chapter~\ref{ch:editquality}, which are in turn built atop a
difference algorithm defined in Chapter~\ref{ch:diff}.

To validate the effectiveness of the WikiTrust reputation system, we
evaluate a variety of measures.
We find that short-lived text and short-lived edits are correlated with low-reputation,
and that manual examination of the edits by a small group of reviewers
has high agreement with the assessment by our reputation system.
We also compare favorably against a reputation based purely on the
number of edits made by authors (the so-called \intro{edit count}
reputation~\cite{Cross2006}), but without the same exposure to simple
reputation attacks such as breaking up a large edit into smaller edits.

