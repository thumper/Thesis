\section{Summary of Completed Research}

To measure the reputation of authors, the system analyzes
the revision history of all wiki articles, in
chronological order.
The evolution of an article is tracked using a granularity
of words (as opposed to characters or sentences),
with an author being assigned to each word.
Suppose that an author \editor{k} contributes to a Wikipedia article by
performing an edit $\revision{k}: \version{k-1} \goesto \version{k}$
from version \version{k-1} to
version \version{k}.
When another author \editor{j} subsequently performs an edit
$\revision{j}: \version{j-1} \goesto \version{j}$
of the same article ($j > k$), author \editor{j}
implicitly gives a judgment on the quality of \editor{k}'s contribution, by
preserving it or removing it.
The reputation system relies on these judgments, in two forms, to increase
or decrease the reputation of \editor{k}: \textit{text survival}
and \textit{edit survival}~\cite{Adler2007}.

\subsection{Reputation update}

  Updating the reputation of author \editor{k} for their editing
  work is based on this notion of edit longevity.
  When edit \revision{j} occurs, if
  $k < j \le k+3$ and $\editor{k} \ne \editor{j}$,
  we first compute the value:
  \begin{equation*}
  	q := \frac{ \slack \cdot \dist(\version{k-1},\version{j})
		- \dist(\version{k},\version{j}) }
		{ \dist(\version{k-1},\version{k}) }
  \end{equation*}
  If $q < 0$, we update this value to include a punishment factor,
  $q := q \cdot \coeffpunish$.
  The reputation for \editor{k} is then updated according to:
  \begin{equation*}
  	q \cdot \coeffrep \cdot (1 - \coefftext) \cdot
	        (\dist(\version{k-1},\version{k}))^\lengthexp
		\cdot \log(1 + \rep(\editor{j},\revision{j}))
  \end{equation*}
  where $\coeffpunish \geq 1$, $\slack \geq 1$, $\coeffrep > 0$,
  $\coefftext \in [0,1]$, and $\lengthexp \in [0,1]$ are parameters,
  and $\rep(\editor{j},\revision{j})$ is the reputation
  of~$\editor{j}$ at the time~$\revision{j}$ is performed.
  As with the text survival parameters, we used a search function
  to discover optimal parameter values, so that there was a high
  correlation between author reputation and the edit longevity
  of edit \revision{k}.


\subsubsection*{Quantitative Evaluation}

\begin{table}
\begin{center}
\begin{tabular}{|r|c|c|} \hline
Longevity & Judged bad & Judged good \\ \hline
\multicolumn{1}{|l|}{Short-lived edits: \qquad \quad} & & \\[1ex]
   Low [0.0--0.2]   &    66  \% &    19 \% \\
Normal [0.2--1.0]   &    16  \% &    68 \% \\ \hline
\multicolumn{1}{|l|}{Short-lived text: \qquad \quad} & & \\[1ex]
   Low [0.0--0.2]   &    74  \% &    13 \% \\
Normal [0.2--1.0]   &    14  \% &    85 \% \\ \hline
\end{tabular}
\end{center}
\caption{User ratings of short-lived edits and text from the
  Italian Wikipedia, as a function of author reputation.
  We presented edit differences to a test group of users;
  we selected edits that had
  low text longevity or low edit longevity,
  and asked users to rate whether the edit was good or bad.
  In square brackets, we give the
  interval where the normalized value $\log(1+r) / \log (1 +
  \coeffmaxrep)$ of a reputation $r$ falls.  The precentages do not
  add to 100\%, because users could also rank a change as ``neutral''.}
\label{tbl:human}
\end{table}

\begin{table}
\begin{center}
\begin{tabular}{|l||c|c||c|c|} \hline
 & \multicolumn{2}{|c||}{Precision}
 & \multicolumn{2}{|c|}{Recall} \\
 & Edit & Text & Edit & Text \\[0.5ex] \hline
\textbf{Italian Wikipedia:} & & & &    \\
\qquad Content-driven reputation & 14.15 &  3.94 & 19.39 & 38.69 \\
\qquad Edit count as reputation  & 11.50 &  3.32 & 19.09 & 39.52 \\ \hline
\textbf{French Wikipedia:} & & & & \\
\qquad Content-driven reputation & 23.92 &  5.85 & 32.24 & 37.80 \\
\qquad Edit count as reputation &  21.62 &  5.63 & 28.30 & 37.92 \\ \hline
\end{tabular}
\end{center}
\caption{Summary of the performance of content-driven reputation over
the Italian and French Wikipedias. All data are expressed as percentages.
Precision is the probability that the text or edit longevity is low,
given that the reputation is low.
Recall is the probability that the reputation is low, given that
the text or edit longevity is low.
}
\label{tbl:comparison-with-count}
\end{table}



We have evaluated the effectiveness of the
reputation system in two ways.
The first evaluation involved the human judgements of seven volunteers on
680 edits from the Italian Wikipedia;
each participant could rate
an edit as ``good,'' ``bad'' or ``neutral.''
We averaged the judgements of the participants, and
compared these results to the predictions made
by the reputation of the authors of the edits.
The results, presented in Table~\ref{tbl:human},
show that low reputation is a very good predictor
of bad short-lived text.
Using these results, we computed the approximate
recall factors on the Italian Wikipedia; what fraction
of the changes judged \textit{bad} by humans did our
content-driven reputation system identify by assigning low-reputation
to the author:
\begin{itemize}
\item The recall for short-lived edits that are judged to
    be bad is over 49\%.
\item The recall for short-lived text that is judged to
    be bad is over 79\%.
\end{itemize}

Our second evaluation was to measure the \textit{predictive}
power of our reputation system, and compare it to the
\textit{edit-count} reputation system.
It is commonly believed that, as Wikipedia authors gain
experience (through revision comments, talk pages,
and reading articles on Wikipedia standards), the quality
of their submissions goes up.
Hence, it is reasonable to take edit count --- the number of edits
performed by the author --- as a form of reputation.

We considered edits to the Italian and French Wikipedias, and
for each revision, we noted the reputation of the author at the
moment the change was made (which is based on past changes by the author),
and we computed the text and edit longevity of the change (which
is based on future edits to the same article).
Some results are presented in Table~\ref{tbl:comparison-with-count};
other metrics are additionally presented in~\cite{Adler2007}.
We believe that one reason the edit-count based reputation
performs well in our measurements is that authors, after
performing edits that are criticized and reverted,
commonly either replace their identity or stop contributing altogether.
The edit-count reputation system could never work in the live Wikipedia, however,
as authors would change their behavior to maximize their edit count.

