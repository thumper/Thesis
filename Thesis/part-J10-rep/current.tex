\subsubsection*{Quantitative Evaluation}

\mynote{Cross-check this wording with earlier wording.  Do
I prefer this?}

We have evaluated the effectiveness of the
reputation system in two ways.
The first evaluation involved the human judgements of seven volunteers on
680 edits from the Italian Wikipedia;
each participant could rate
an edit as ``good,'' ``bad'' or ``neutral.''
We averaged the judgements of the participants, and
compared these results to the predictions made
by the reputation of the authors of the edits.
The results, presented in Table~\ref{tbl:human},
show that low reputation is a very good predictor
of bad short-lived text.
Using these results, we computed the approximate
recall factors on the Italian Wikipedia; what fraction
of the changes judged \textit{bad} by humans did our
content-driven reputation system identify by assigning low-reputation
to the author:
\begin{itemize}
\item The recall for short-lived edits that are judged to
    be bad is over 49\%.
\item The recall for short-lived text that is judged to
    be bad is over 79\%.
\end{itemize}

Our second evaluation was to measure the \textit{predictive}
power of our reputation system, and compare it to the
\textit{edit-count} reputation system.
It is commonly believed that, as Wikipedia authors gain
experience (through revision comments, talk pages,
and reading articles on Wikipedia standards), the quality
of their submissions goes up.
Hence, it is reasonable to take edit count --- the number of edits
performed by the author --- as a form of reputation.

We considered edits to the Italian and French Wikipedias, and
for each revision, we noted the reputation of the author at the
moment the change was made (which is based on past changes by the author),
and we computed the text and edit longevity of the change (which
is based on future edits to the same article).
Some results are presented in Table~\ref{tbl:comparison-with-count};
other metrics are additionally presented in~\cite{Adler2007}.
We believe that one reason the edit-count based reputation
performs well in our measurements is that authors, after
performing edits that are criticized and reverted,
commonly either replace their identity or stop contributing altogether.
The edit-count reputation system could never work in the live Wikipedia, however,
as authors would change their behavior to maximize their edit count.

