\section{Evaluation Metrics}
\label{sec:rep-eval}

In developing a reputation system, one must ask
``what is it intended to signal?''
For WikiTrust, our hope was that a high reputation
would signal that edits made by the author were likely
to be of good quality, while a low reputation would
signal that the edit was of unknown quality.
Evaluation of the system becomes the crucial
factor, so that users can compare one system to another.

We evaluate our reputation system by using the quality
measures of Chapter~\ref{ch:editquality} to define two
sets of binary classifications, and then calculate
our reputation system's precision and recall of correctly
classifying each revision according to those classifications.
We observe that both text longevity and edit longevity
are computed based on the
evolution of the article text \textit{after} the time that
revision \version{i} is created, while $\rep(\version{i})$ is
computed based on events \textit{before} the time of \version{i},
so that a comparison between them isn't predisposed to showing
a correlation.

To formally define this framework, we take the view that revisions
are generated by a probabilistic process, with \versions{} as
the list of outcomes from that process.
We associate with each revision a probability mass (a weight)
proportional to the number of words that revision affects.
This compensates for our setting of \textit{filtered} revisions,
where we combine consecutive revisions made by the same author;
in such a setting, the unit of a ``revision'' is somewhat arbitrary,
while weighting scales with the net amount of work done by each author.
Each of our two quality measures has a weighting that is appropriate
to it; given $\version{i} \in \versions{a}$, for some
article $a \in \articles$, where $i = \revpos{\version{i}}$ as usual,
we define the probability mass to scale a revision by as:
\begin{align*}
\editmass(\version{i}) & = \dist{a}{i-1,i}, & \text{for edit longevity.} \\
\textmass(\version{i}) & = \tsurv{a}{i,i}, & \text{for text longevity.} \\
\end{align*}

We define our categories by choosing a partition for each measure,
and define three random variables $S_e, S_t, L: \versions{}
\mapsto \set{0,1}$ as follows,
for all $\version{i} \in \versions{}$ where $i = \revpos{v}$:
\begin{itemize} 
\item We say that the new text added in version \version{i}
  is \intro{short-lived text} if \quality{tdecay}{10}{i} is at
  the low end of the range.
  We define $S_t(\version{i})=1$ if $\quality{tdecay}{10}{i} \leq  0.2$,
  and $S_t(\version{i})=0$ otherwise.
  This indicates that at most 20\% of new text, on
  average, survives from one version to the next. 

\item We say that the edit performed in taking version
  \version{i-1} to version \version{i}
  is a \textit{short-lived edit} if
  \quality{elong}{3}{i} is low.
  Specifically, $S_e(\version{i})=1$ if $\quality{elong}{3}{i} \leq -0.8$,
  and $S_e(\version{i})=0$ otherwise.

\item We also partition revisions according to whether
  they are \intro{low-reputation} or not.
  We define low-reputation similarly to the quality measures,
  as $L(\version{i})=1$ if
  $\log (1+\rep(\version{i})) \leq \log(1+\coeffmaxrep) / 5$;
  $L(\version{i})=0$ otherwise, and again we have chosen a
  partition that represents the lowest 20\% of the range
  after logarithmic scaling.
  Note that the reputation of \revauthor{\version{i}} does not actually
  change at the time of version \version{i}'s creation; the reputation
  of the author of a revision
  is adjusted as judges become available.

\end{itemize}
%
The precision $\prect$ and recall $\recallt$
for short-lived text, and 
the precision $\prece$ and recall $\recalle$
for short-lived edits, are defined as:
%
\begin{align*}
    \prect & = \textstyle\Pr(S_t{=}1 \mid L{=}1) 
  & \recallt & = \textstyle\Pr(L{=}1 \mid S_t{=}1) \\
    \prece & = \textstyle\Pr(S_e{=}1 \mid L{=}1) 
  & \recalle & = \textstyle\Pr(L{=}1 \mid S_e{=}1).
\end{align*}
\begin{comment}
\ifshort
For short-lived text, the {\em precision\/} is 
$
  \textstyle \prect = \Pr(S_t=1 \mid L=1)
$,
and the {\em recall\/} is 
$
  \textstyle \recallt = \Pr(L=1 \mid S_t=1)
$.
Similarly, for short-lived edits, we define the 
precision is $\prece = \Pr(S_e=1 \mid L=1)$, 
and the recall is $\recalle = \Pr(L=1 \mid S_e=1)$.
\fi
\end{comment}
These quantities can be computed as usual; for instance, 
\begin{equation*}
  {\textstyle \Pr} (S_e=1 \mid L=1) = 
  \frac{\sum_{v \in \versions{}} S_e(v) \cdot L(v) \cdot \editmass(v)}{%
    \sum_{v \in \versions{}} L(v) \cdot \editmass(v)}.
\end{equation*}


We also define the \intro{boost} that knowing reputation gives
to predicting a short-lived revision: 
%
\begin{align*}
  \booste & = \frac{\Pr(S_e=1 \mid L=1)}{\Pr(S_e=1)}
            = \frac{\Pr(S_e=1 , L=1)}{\Pr(S_e=1)\cdot\Pr(L=1)} \\
  \boostt & = \frac{\Pr(S_t=1 \mid L=1)}{\Pr(S_t=1)}
            = \frac{\Pr(S_t=1 , L=1)}{\Pr(S_t=1)\cdot\Pr(L=1)}
\end{align*}
%
Intuitively, $\booste$ indicates how much more likely than average
it is that edits produced by low-reputation authors are short-lived.
The quantity $\boostt$ has a similar meaning. 

Our last indicators of quality are the
\textit{coefficients of constraint}~\cite{Coombs1970,Cover1991}:
\[ 
  \constrainte = I_e(S_e,L) / H_e(L)
  \qquad 
  \constraintt = I_t(S_t,L) / H_t(L),
\]
where $I_e$ is the {\em mutual information\/} of $S_e$ and $L$,
and $H_e$ is the entropy of $L$;
similarly for $I_t(S_t,L)$ and $H_t(L)$.
The quantity $\constrainte$ is the fraction of the entropy of the
edit longevity which can be explained by the reputation of the author; 
this is an information-theoretic measure of correlation. 
The quantity $\constraintt$ has an analogous meaning. 

To assign a value to the coefficients $\coeffrep$, $\slack$,
$\coeffpunish$, $\coefftext$, $\lengthexp$, and $\coeffmaxrep$, 
we implemented a search procedure, whose goal was to find values for
the parameters that maximized a given objective function. 
We applied the search procedure to the Italian Wikipedia, reserving
the French Wikipedia for validation once the coefficients were
determined. 
We experimented with $\constrainte$ and $\prece \cdot \recalle$
as objective functions, and they gave very similar results. 

