\section{Evaluation Metrics}
\label{sec:rep-eval}

In developing a reputation system, one must ask
``what is it intended to signal?''
For WikiTrust, our hope was that a high reputation
would signal that edits made the author were likely
to be of good quality, while a low reputation would
signal that the edit was of unknown quality.
Evaluation of the system becomes the crucial
factor, so that users can compare one system to another.


At the time that this work was conducted, there was
no annotated corpus of edits available to compare against.
\mynote{Look up what McGuinness did for evaluation in her papers.}

We propose several different metrics for
evaluating our reputation systems, by casting the interpretation
into binary classification problems parameterized by a
quality metric.
We use the two quality metrics explored in Chapter~\ref{ch:editquality}
to define the following categories:
\begin{itemize} 
\item We say that the new text added in version \version{i}
  is \intro{short-lived} if $\quality{tdecay}{10}{i} \leq 0.2$.
  This indicates that at most 20\% of it, on
  average, survives from one version to the next. 

\item We say that the edit performed in taking version
  \version{i-1} to version \version{i}
  is \textit{short-lived} if
  $\quality{elong}{3}{i} \leq -0.8$.

\item We say that a version \version{i} is \intro{low-reputation} if
  $\log (1+\rep(\version{i})) \leq \log(1+\coeffmaxrep) / 5$, indicating that
  the reputation, after logarithmic scaling, falls in the lowest 20\% of
  the range.
  Note that the reputation of \revauthor{\version{i}} does not actually
  change at the time of version \version{i}'s creation; the reputation
  of the author of a revision
  is adjusted as judges become available.

\end{itemize}
We observe that both quality metrics are computed based on the
evolution of the article text \textit{after} the time that
version \version{i} is created, while $\rep(\version{i})$ is
computed based on events \textit{before} the time of \version{i},
so that a comparison between them isn't predisposed to showing
a correlation.


We take the view that revisions are a probabilistic
process, with \versions{} as the set of outcomes across
all articles.
Since we keep only the last revision of several consecutive
versions by the same author, a ``version'' seems somewhat arbitrary.
We therefore weight each version $v \in \versions{}$
(and let $a \in \articles$ be such that $v \in \versions{a}$,
and $i = \revpos{v}$)
by a ``revision amount'' appropriate to the quality being studied:
\begin{itemize}
\item for edit longevity, we weight revisions by
        $\editmass(v) = \dist{a}{i-1,i}$.
\item for text decay, we weight revisions by
        $\textmass(v) = \tsurv{a}{i,i}$.
\end{itemize}


Using our categories as a basis,
we define three random variables $S_e, S_t, L: \versions{}
\mapsto \set{0,1}$ as follows, for all $v \in \versions{}$ where
$i = \revpos{v}$:
%
\begin{itemize} 

\item $S_e(v)=1$ if $\quality{elong}{3}{i} \leq -0.8$, and $S_e(v)=0$ otherwise.
\item $S_t(v)=1$ if $\quality{tdecay}{10}{i} \leq  0.2$, and $S_t(v)=0$ otherwise.
\item $L(v)=1$ if $\log (1+\rep(v)) \leq \log(1+\coeffmaxrep) / 5$,
  and $L(v)=0$ otherwise.

\end{itemize}
%
The {\em precision\/} $\prect$ and {\em recall\/} $\recallt$
for short-lived text, and 
the {\em precision\/} $\prece$ and {\em recall\/} $\recalle$
for short-lived edits, are defined as:
%
\begin{align*}
    \prect & = \textstyle\Pr_t(S_t{=}1 \mid L{=}1) 
  & \recallt & = \textstyle\Pr_t(L{=}1 \mid S_t{=}1) \\
    \prece & = \textstyle\Pr_e(S_e{=}1 \mid L{=}1) 
  & \recalle & = \textstyle\Pr_e(L{=}1 \mid S_e{=}1).
\end{align*}
\begin{comment}
\ifshort
For short-lived text, the {\em precision\/} is 
$
  \textstyle \prect = \Pr_t(S_t=1 \mid L=1)
$,
and the {\em recall\/} is 
$
  \textstyle \recallt = \Pr_t(L=1 \mid S_t=1)
$.
Similarly, for short-lived edits, we define the 
precision is $\prece = \Pr_e(S_e=1 \mid L=1)$, 
and the recall is $\recalle = \Pr_e(L=1 \mid S_e=1)$.
\fi
\end{comment}
These quantities can be computed as usual; for instance, 
\begin{equation*}
  {\textstyle \Pr_e} (S_e=1 \mid L=1) = 
  \frac{\sum_{v \in \versions{}} S_e(v) \cdot L(v) \cdot \editmass(v)}{%
    \sum_{v \in \versions{}} L(v) \cdot \editmass(v)}.
\end{equation*}


We also define: 
%
\begin{align*}
  \booste & = \frac{\Pr_e(S_e=1 \mid L=1)}{\Pr_e(S_e=1)}
            = \frac{\Pr_e(S_e=1 , L=1)}{\Pr_e(S_e=1)\cdot\Pr_e(L=1)} \\
  \boostt & = \frac{\Pr_t(S_t=1 \mid L=1)}{\Pr_t(S_t=1)}
            = \frac{\Pr_t(S_t=1 , L=1)}{\Pr_t(S_t=1)\cdot\Pr_t(L=1)}
\end{align*}
%
Intuitively, $\booste$ indicates how much more likely than average
it is that edits produced by low-reputation authors are short-lived.
The quantity $\boostt$ has a similar meaning. 
Our last indicator of quality are the {\em coefficients of constraint\/}
\[ 
  \constrainte = I_e(S_e,L) / H_e(L)
  \qquad 
  \constraintt = I_t(S_t,L) / H_t(L),
\]
where $I_e$ is the {\em mutual information\/} of $S_e$ and $L$,
computed with respect to $\Pr_e$, and $H_e$ is the entropy of $L$,
computed with respect to $\Pr_e$ \cite{Cover1991}; similarly for
$I_t(S_t,L)$ and $H_t(L)$.
The quantity $\constrainte$ is the fraction of the entropy of the
edit longevity which can be explained by the reputation of the author; 
this is an information-theoretic measure of correlation. 
The quantity $\constraintt$ has an analogous meaning. 

To assign a value to the coefficients $\coeffrep$, $\slack$,
$\coeffpunish$, $\coefftext$, $\lengthexp$, and $\coeffmaxrep$, 
we implemented a search procedure, whose goal was to find values for
the parameters that maximized a given objective function. 
We applied the search procedure to the Italian Wikipedia, reserving
the French Wikipedia for validation, once the coefficients were
determined. 
We experimented with $I_e(S_e,L)$ and $\prece \cdot \recalle$
as objective functions, and they gave very similar results. 

