
In trying to address the quality issues of the Wikipedia,
our primary guideline was to preserve the current user experience.
By relying on article version history, we can assign reputation
through examination of the content evolution:
authors who perform long-lived contributions gain reputation; authors
whose contributions are reverted, or are soon removed, lose reputation.
There are several possible applications of computing a reputation
value for authors (for example, to grant or deny editing rights to
crucial pages \cite{Blaze96}), and we use it to drive a trust system for
Wikipedia content.
The fact that authors can only comment on other authors by
making contributions themselves discourages users from attacking
each other in an unproductive way, because they risk their own reputation in the process.


The simplest idea for a content-driven reputation system would measure
how much text an author contributed.
During the course of our research, however, we realized that there are
two distinct ways that authors contribute to the Wikipedia: by adding
new content, and by revising existing content.
Both are important to consider, since several users will
adopt one contribution style and not the other.
We introduced the notions of
\intro{text survival}, which measures how much text survives
into later revisions, and \intro{edit survival}, which measures
how consistent an edit is with later revisions,
to account for these different methods of contribution~\cite{www07}.
We use the principle that every future author is
implicitly making an evaluation on the work of past authors:
edits and new text that are consistent with later revisions
are judged ``good'' by the community.
Constants in the model we develop are assigned so as to optimize
for the heuristic that author reputation
be correlated to edit longevity as much as possible.

