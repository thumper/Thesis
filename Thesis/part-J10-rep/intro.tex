One approach to dealing with the vandalism problem is to detect
it immediately and flag it to administrators.
We call this the ``immediate vandalism detection'' problem,
which we discuss further in Chapter~\ref{ch:vandalism}.
There were already bots making automated edits within Wikipedia
in~2005 when we began this research, and it was clear that
expert systems would soon be developed based on natural language processing
ideas~\cite{wiki:AntiVandalBot,wiki:MartinBot,wiki:ClueBot,Carter2007},
so we decided to strike out in the direction of building a reputation system
--- and accidentally, we stumbled onto the problem of
``historical vandalism detection.''


Reputation is founded on the notion that the past is a good
predictor of the future, but there is also a prescriptive aspect
as the formulation describes to the world what is important and
what is not.
We take it to be a score, and the higher your reputation the more
likely you are to do good work in the future; when you do bad
work, your reputation score goes down.

We wrestled for some time about negative reputations. We assign "bad"
reputations to people in real life, but can this translate to an online
system?  

\mynote{Cite Randy's book?}
An obvious way to build a reputation system would be to build one
modeled on eBay or Amazon's rating system: elicit votes from users
on the quality of a revision or the work of an author.
We had several concerns about this: lack of adoption, sybil attacks
through puppet votes, block voting by cliques of users, and
disrupting the current user experience of the Wikipedia community.
This led us to the idea of \intro{implicit voting} through
actions already taken by the community as part of their normal activities.
For example, reading an article and not making an edit to it is
an implicit vote (though perhaps not a strong one)
that the article is of good quality; the more page views an article
has, the more we can believe that there are no errors in it.
We settled using actual edits as the implicit vote: if a
user edits the text of a previous author, that would be a
negative vote for the previous work; but if the user only added
new text, then implicitly he is saying that the previous text
was already in good shape.

When we started, we debated using a fixpoint algortihm
such as the PageRank~\cite{Page1999} or HITS~\cite{Kleinberg1999}
algorithms.

In trying to address the quality issues of the Wikipedia,
our primary guideline was to preserve the current user experience.
By relying on article version history, we can assign reputation
through examination of the content evolution:
authors who perform long-lived contributions gain reputation; authors
whose contributions are reverted, or are soon removed, lose reputation.
There are several possible applications of computing a reputation
value for authors (for example, to grant or deny editing rights to
crucial pages~\cite{Blaze1996}); within the WikiTrust project,
we use reputation to drive a trust system for
Wikipedia content~\cite{Adler2008}.\footnote{This trust data
is publicly available through our Firefox plugin:
\url{https://addons.mozilla.org/en-US/firefox/addon/wikitrust/}}
The fact that authors can only comment on other authors by
making contributions themselves discourages users from attacking
each other in an unproductive way, because they risk their own reputation in the process.


The simplest idea for a content-driven reputation system would measure
how much text an author contributed.
During the course of our research, however, we realized that there are
two distinct ways that authors contribute to the Wikipedia: by adding
new content, and by revising existing content.
Both are important to consider, since several users will
adopt one contribution style and not the other.
We introduced the notions of
\intro{text survival}, which measures how much text survives
into later revisions, and \intro{edit survival}, which measures
how consistent an edit is with later revisions,
to account for these different methods of contribution~\cite{Adler2007}.
We use the principle that every future author is
implicitly making an evaluation on the work of past authors:
edits and new text that are consistent with later revisions
are judged ``good'' by the community.
Constants in the model we develop are assigned so as to optimize
for the heuristic that author reputation
be correlated to edit longevity as much as possible.

This most significant challenge of this part of our work
was in developing evaluations to measure our performance.

The main contribution of this paper is a content-driven reputation
system for the Wikipedia. 
In particular, we define the concepts of edit and text longevity in a
versioned document, and we show how these two concepts can be used to
compute a reputation for each contributor. 
A novel feature of the proposed reputation system is its emphasis on
the quantitative {\em predictive\/} value of reputation; this emphasis
enabled us both to tune the behavior of our reputation system, and to
measure its effectiveness on actual instances of the Wikipedia. 

In the course of developing the main results of the paper, we have
developed novel techniques which might be of interest in their own
right. 
These include algorithms for attributing text authorship in a
versioned document, along with algorithms for measuring the longevity
of text additions and edits performed in the various revisions of a
document. 
We note that, while we present our work in the context of the
Wikipedia, our algorithms can be used to compute the reputation, and
contribution, of users to any set of versioned documents, including
computer code. 


