\section{Related Work}

The work most closely related to ours is \cite{Zeng2006}, where the
revision history of a Wikipedia article is used to compute a trust
value for the article.
Dynamic Bayesian networks are used to model the
evolution of trust level over the versions. 
At each revision, the inputs to the network are a priori models of
trust of authors (determined by their Wikipedia ranks), 
and the amount of added and deleted text. 
The paper shows that this approach can be used to predict the quality
of an article; for instance, it can be used to predict when an article
in a test set can be used as a featured article.  
Differently from the current work, author trustworthiness is
taken as input; we compute author reputation as output. 
Several approaches for computing text trust are outlined
in~\cite{WikiMTWtrust06}. 
A simpler approach to text trust, based solely on text age, is
advocated in~\cite{Cross2006}. 

Document trust, and author reputation, have different applications. 
Document trust provides a measure of confidence in the accuracy of the
document. 
Author reputation is most useful as an indicator of quality
for fresh text, and can be used to manage author activity, for
instance, preventing low-reputation authors from editing some pages. 

Reputation systems in e-commerce and social networks has been extensively
studied~\cite{Kleinberg1999,Resnick2000,Dellarocas2003,Kamvar2003};
the reputation in those systems is generally user-driven, rather than
content-driven as in our case. 
Related is also work on trust in social networks~\cite{Guha2004,Golbeck2005}. 

The history flow of text contributed by Wikipedia authors has
been studied with flow visualization methods in~\cite{Viegas2004}; 
the results have been used to analyze a number of interesting patterns
in the content evolution of Wikipedia articles. 
Work on mining software revision logs~\cite{Livshits2005}
is similar in its emphasis of in-depth analysis of revision logs; the
aim, however, is to find revision patterns and indicators that point
to software defects, rather than to develop a notion of author
reputation. 

