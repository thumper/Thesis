\section{Content-Driven Reputation in a Versioned Document} 

We propose the following method for computing content-driven
reputation in a versioned document. 
Consider a revision $r_i: v_{i-1} \voom v_i$, performed by author $a_i$, 
for some $0 < i \leq n$. 
Each of the subsequent authors $a_{i+1}, a_{i+2}, \ldots$ can either
retain, or remove, the edits performed by $a_i$ at $r_i$.
Thus, we consider later versions $v_j$ of the document, for $i < j
\leq n$, and we increase, or decrease, the reputation of $a_i$
according to how much of the change performed at $r_i$ has survived
until $v_j$. 
The greater is the reputation of the author $a_j$ of $v_j$, 
the greater is the change in the reputation of $a_i$. 
This ensures that high-reputation authors risk little of 
their reputation when fighting revision wars against low
reputation authors, such as anonymous authors. 
If this were not the case, high-reputation authors would be wary of
undoing damage done by low-reputation authors, including spammers, for
fear of reprisal.

We use two criteria for awarding reputation: the survival of text, and
the survival of edits.
Text survival is the most natural of these two criteria. 
Adding new text to a Wikipedia article is a fundamental way of
contributing --- it is how knowledge is added to the Wikipedia. 
However, text survival alone fails to capture many ways in which authors
contribute to the Wikipedia. 
If a user rearranges the content of an article without introducing new
text, the contribution cannot be captured by text survival. 
Similarly, reverting an act of vandalism does not result in the
introduction of new text.\footnote{Our algorithms distinguish between
new text, and the reintroduction of previously deleted text.}
%% If we computed reputation on the basis of text survival alone, authors
%% who revert vandalism would not gain reputation from their actions,
%% even though the repair they perform is fundamental to the Wikipedia. 
Edit survival captures how long the re-arrangements performed by an
author last in the history of a Wikipedia article, and captures all of
the above ways of contributing. 

%% We use a parameter $\coefftext \in [0,1]$ to determine how much weight
%% should be given to text versus edit survival. 
%% We have determined the parameter $\coefftext$, along with other
%% parameters used in the calculation, via an optimization process aimed
%% at producing reputation functions of high predictive value; the
%% optimization process is described later in Section~\ref{sec-optimize}.


\subsection{Accounting for Text Survival} 

If text introduced by $r_i$ is still present in $v_j$, for 
$0 < i < j \leq n$, this indicates that author $a_j$, who performed the
revision $r_j: v_{j-1} \voom v_j$, agrees that the text is valuable. 
To reflect this, we increase the reputation of $a_i$ in a manner that
is proportional both to the amount of residual text, and to the
reputation of $a_i$. 
Thus, we propose the following rule.

\begin{regola} \textbf{(reputation update due to text survival)} \label{rule-text}
  When the revision $r_j$ occurs, for all $0 < i < j$ such that 
  $j - i \leq 10$ and $a_j \neq a_i$, we add to the reputation
  of $a_i$ the amount: 
  \[
    \coeffrep \cdot \coefftext \cdot \frac{\txt(i,j)}{\txt(i,i)} 
    \cdot (\txt (i,i))^\lengthexp \cdot \log(1 + \rep(a_j,r_j)),
  \]
  where $\coeffrep > 0$, $\coefftext \in [0,1]$, and $\lengthexp \in
  [0,1]$ are parameters, and where $\rep(a_j,r_j)$ is the reputation of $a_j$
  at the time $r_j$ is performed. 
\end{regola}

\noindent
In the rule, $\txt(i,j) / \txt(i,i)$ is the fraction of text
introduced at version $v_i$ that is still present in version $v_j$;
this is a measure of the ``quality'' of $r_i$. 
The quantity $\log(1 + \rep(a_j,r_j))$ is the ``weight'' of the
reputation of $a_j$, that is, how much the reputation of $a_j$ lends
credibility to the judgements of $a_j$. 
In our reputation system, the reputations of many regular contributors
soar to very high values; using a logarithmic weight for reputation
ensures that the feedback coming from new authors is not completely
overridden by the feedback coming from such regular contributors. 
The parameters $\coeffrep$, $\coefftext$ and $\lengthexp$ will be
determined experimentally via an optimization process, as described
later.
The parameter $\lengthexp \in [0,1]$ is an exponent that specifies how
to take into account the length of the original contribution: if
$\lengthexp = 1$, then the increment is proportional to the length of
the original contribution; if $\lengthexp = 0$, then the increment
does not depend on the length of the original contribution.
The parameter $\coeffrep$ specifies how much the reputation should
vary in response to an individual feedback. 
The parameter $\coefftext$ specifies how much the feedback should 
depends on residual text (Rule~\ref{rule-text}) or residual edit
(Rule~\ref{rule-edit}, presented later). 

To give feedback on a revision, the rule considers at most 10
successive versions. 
This ensures that contributors to early versions of articles do not
accumulate disproportionate amounts of reputation. 
%% We considered basing the limit on time, rather than on the number of
%% versions, but each Wikipedia article has its own rate of change:
%% using the number of versions ensures that fast and
%% slow-changing pages are treated in similar fashion.
A rule using exponential decay, rather than a sharp cutoff at 10,
would probably have been preferable, at a slightly larger
computational cost. 
% cancut above

\subsection{Accounting for Edit Survival}

\iflong
\begin{figure}
\begin{center}
\input{edist.eepic}
\end{center}
\vspace*{-3ex}
\caption{Distances involved in the computation of $\elong(i,j)$.}
\label{fig-triangle}
\end{figure}
\fi 

To judge how much of a revision $r_i: v_{i-1} \voom v_i$ is preserved
in a later version $v_j$, for $0 < i < j \leq n$, we reason as
follows. 
We wish to rate $r_i$ higher, if $r_i$ made the article more similar to
$v_j$. 
In particular, we wish to credit $a_i$ in proportion to how much of
the change $r_i$ is directed towards $v_j$. 
This suggests using the formula: 
%
\begin{equation} \label{eq-edit-long} 
  \elong(i,j) =
  \frac{\dist(v_{i-1},v_j) - \dist(v_{i},v_j)}{\dist(v_{i-1},v_i)} . 
\end{equation}
%
\iflong
Figure~\ref{fig-triangle} depicts the distances involved in the
computation of $\elong(i,j)$. 
\fi
If $\dist{}{}$ satisfies the triangular inequality (as our edit distance
does, to a high degree of accuracy), then $\elong(i,j) \in [-1,1]$.  
For two consecutive edits $r_i$, $r_{i+1}$,
if $r_i$ is completely undone in $r_{i+1}$ 
(as is common when $r_i$ consists in the introduction of spam or
inappropriate material), then $\elong(i,i+1) = -1$; 
if $r_{i+1}$ entirely preserves $r_i$, then $\elong(i,i+1) = +1$.
For more distant edits, $\elong(i,j)$ is a measure of how much of the
edit performed during $r_i$ is undone (value $-1$) or preserved (value
$+1$) before $r_j$. 

Note that $\elong(i,j) < 0$, i.e., $j$ votes to lower the reputation
of $i$, only when $\dist(v_{i-1},v_j) < \dist(v_{i},v_j)$, that is,
when $r_j$ is closer to the version $v_{i-1}$ {\em preceding\/}
$a_i$'s contribution, than to $v_i$.
Thus, the revision performed by $a_i$ needs to have been undone at
least in part by $a_{i+1}$, $a_{i+2}$, \ldots, $a_j$, for $a_i$'s
reputation to suffer.  
This is one of the two mechanisms we have to ensure that authors whose
contributions are rewritten, rather than rolled back, still receive
reputation in return for their efforts. 
The second mechanism is described in the next section.
We use the following rule for updating reputations. 

\begin{regola} \textbf{(reputation update due to edit survival)} \label{rule-edit}
  When the revision $r_j$ occurs, for all $0 < i < j$ such that 
  $j - i \leq 3$, we add to the reputation
  of $a_i$ an amount $q$ determined as follows. 
  If $a_j = a_i$ or $\dist(v_{i-1},v_i) = 0$, then $q=0$; otherwise,
  $q$ is determined by the following algorithm.
  %
  \begin{tabbing}
  $\displaystyle q := \frac{\slack \cdot \dist(v_{i-1}, v_j) 
    - \dist(v_i, v_j)}{\dist(v_{i-1}, v_i)}$ \\[1ex]
  \textbf{if} $q < 0$ \textbf{then} $q := q \cdot \coeffpunish$ \textbf{endif} \\[1ex]
  $q := q \cdot \coeffrep \cdot (1 - \coefftext) \cdot 
        (\dist(v_{i-1}, v_i))^\lengthexp \cdot \log(1 + \rep(a_j,r_j))$
  \end{tabbing}
  %
  In the algorithm, $\coeffpunish \geq 1$, $\slack \geq 1$, $\coeffrep > 0$, 
  $\coefftext \in [0,1]$, and $\lengthexp \in [0,1]$ are parameters,
  and $\rep(a_j,r_j)$ is the reputation of $a_j$ at the time
  $r_j$ is performed.
\end{regola}

\noindent
The rule adopts a modified version of (\ref{eq-edit-long}).
The parameter $\slack > 1$ is used to spare $a_i$ from punishment when
$r_i: v_{i-1} \voom v_i$ is only slightly
counterproductive.
On the other hand, when punishment is incurred, its magnitude is
magnified by the amount $\coeffpunish$, raising the reputation cost of
edits that are later undone. 
% cancut next 
%% Amplifying punishment is instrumental to making the threat a credible one. 
%% Without amplification, a rogue contributor could use the reputation
%% gained in one part of the Wikipedia to constantly destroy a small set
%% of articles elsewhere.
%% Amplification makes this harder to achieve. 
The parameters $\slack$ and $\coeffpunish$, as well as $\coeffrep$,
$\coefftext$ and $\lengthexp$, will be determined via an optimization
process. 
To assign edit feedback, we have chosen to consider only
the~3 previous versions of an article. 
This approach proved adequate for analyzing an already-existing wiki,
in which authors could not modify their behavior using knowledge of
this threshold. 
If the proposed content-driven reputation were to be used on a live
Wiki, it would be advisable to replace this hard threshold by a scheme
in which the feedback of $v_j$ on $r_i$ is weighed by a gradually
decreasing function of $j-i$ (such as $\exp(c \cdot (i-j))$ for some
$c > 0$). 


\subsection{Computing Content-Driven Reputation}

We compute the reputation for Wikipedia authors as follows. 
We examine all revisions in chronological order, thus simulating the
same order in which they were submitted to the Wikipedia servers. 
We initialize the reputations of all authors to the value~0.1;
the reputation of anonymous authors is fixed to~0.1.  
We choose a positive initial value to ensure that the weight
$\log (1+r)$ of an initial reputation $r = 0.1$ is non-zero, priming
the process of reputation computation.  
This choice of initial value is not particularly critical (the
parameter $\coeffrep$ may need to be adjusted for optimal performance,
if this initial value is changed). 
As the revisions are processed, we use Rules~\ref{rule-text}
and~\ref{rule-edit} to update the reputations of authors in the
system. 
When updating reputations, we ensure that they never become negative,
and that they never grow beyond a bound $\coeffmaxrep > 0$. 
The bound $\coeffmaxrep$ prevents frequent contributors from
accumulating unbounded amounts of reputation, and becoming essentially
immune to negative feedback. 
The value of $\coeffmaxrep$ will be once again determined via
optimization techniques. 

Wikipedia allows users to register, and create an {\em author\/}
identity, whenever they wish. 
As a consequence, we need to make the initial reputation of new authors
very low, close to the minimum possible (in our case,~0). 
If we made the initial reputation of new authors any higher,
then authors, after committing revisions that damage their reputation,
would simply re-register as new users to gain the higher value. 
An unfortunate side-effect of allowing people to obtain new identities
at will is that we cannot presume that people are innocent until
proven otherwise: we have to assign to newcomers the same reputation
as proven offenders. 
This is a contributing factor to our reputation having low 
{\em precision:\/} many authors who have low reputation still perform
very good quality revisions, as they are simply new authors, rather than
proven offenders. 
%% We conjecture that content-driven reputation systems for the Wikipedia
%% would have better predictive value if creating a new author identity
%% was not free, either monetarily, or in some other sense. 


