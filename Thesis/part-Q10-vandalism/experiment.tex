\section{Experiment}

To test our user reputation system's effectiveness, we
decided to evaluate its performance as part of a vandalism
detection system.
The PAN-WVC-10 corpus is a corpus of 32,439 edits manually
annotated by at least three people as part of an Amazon
Mechanical Turk task~\cite{Potthast2010a}.
Of these edits, 2,394 (that is, 7.97\%) were classified as vandalism.

Our goal is to include the WikiTrust user reputations as a
feature in building a model to predict vandalized edits in
the PAN-WVC-10 corpus.
We instrumented the batch processing mode of the WikiTrust
code base to output the reputation score of each author as
it was adjusted over time, creating a chronology of reputation scores.
This revised code was used to process the English Wikipedia
dump from 30-Jan-2010, which includes the time period of
edits from the PAN-WVC-10 corpus.

Using the timestamp of each edit in the PAN-WVC-10 corpus,
we locate its position in the chronology and then work backwards
to find the most recent reputation score of the author
that appears \textit{before} the edit was made.
These reputation stores are then merged with the collection
of features used in the PAN~2010 competition~\cite{Adler2010},
and used to build a model to predict whether an edit is vandalism.
We used Weka~\cite{Weka09} to construct the models using both
the Random Forest and Alternating Decision Tree algorithms.

