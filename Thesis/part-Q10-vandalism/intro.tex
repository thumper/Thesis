
The Wikipedia is a shared resource of the global community, but it depends on
the continued participation of that community to update it and keep it relevant.
Anyone can edit the Wikipedia: this is both its strength,\footnote{The Wikipedia
was initially a side-project of Nupedia enabling collaboration on content
before entering a more formal peer-review process~\cite{wiki:Nupedia}.}
and its weakness.
Multiple studies have found that roughly 7\% of edits are
vandalism~\cite{Potthast2008,Potthast2010a}.
To combat the vandalism, a group of volunteers scans the list of recent changes
to catch obvious vandalism quickly~\cite{wiki:RCPatrol}.

In the chapters leading to this one, we have detailed the technologies
necessary to build a content-driven reputation system for authors.
We first constructed a difference algorithm to compute the work done in a
revision, doing so in a way which models how users think about the units of
language while maintaining performance such that the entire English Wikipedia
could be evaluated in a tractable amount of time.
We then propose two methods for evaluating the quality of the work done by the
users in their revisioning: text longevity and edit longevity.
Finally, we use these quality measures as the basis for rules in a reputation
system.
The output of this reputation system is an estimate of the balance of past
positive contributions over past negative contributions, which we evaluate
as a predictor of the future quality of revisions by the same author.

As part of the PAN 2010 Workshop on vandalism
detection\footnote{\url{http://www.webis.de/research/events/pan-10}, Task 2},
a competition was organized to test out vandalism detection systems with a
single evaluation measure.
Our research group submitted a system based on features derived from
WikiTrust~\cite{Adler2010}, leaving out the actual reputation scores due
to our lack of historical reputation values for authors.
In this chapter, we revisit that work and extend it by including as a feature
the reputation score of authors at the time of their edits.

