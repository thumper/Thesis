\chapter{Vandalism Detection}
\label{ch:vandalism}

\begin{quote}
\textit{This chapter explores a variation of the ideas
published as~\cite{Adler2010}.}
\end{quote}

Multiple studies have found that roughly 7\% of edits are
vandalism~\cite{Potthast2008,Potthast2010a} \mynote{I think
that Luca cites another}.
Due to the preservation of the entire revision history,
anyone can revert an edit, and a group of volunteers scans the
list of recent changes to catch vandalism
quickly~\cite{wiki:RCPatrol}.


\mynote{Running batch process in redherring screen 6.
  Sat, Feb 5.}

\section{Related Work}
\label{sec:vandalism-related}

Wikipedia's official statement of vandalism defines it as
``a \textit{deliberate} attempt to compromise the integrity
of Wikipedia.''\footnote{
\url{http://en.wikipedia.org/wiki/Wikipedia:Vandalism}}
It is, of course, impossible to know the motivations of individuals,
so this definition relies on human intelligence to determine
vandalism on a case-by-case basis --- that is, ``I know it
when I see it,''\footnote{Justice Potter Stewart in
\underline{Jacobellis v. Ohio}, 378 U.S. 184 (1964)}
but there is no precise definition.
Some researchers have undertaken the task of more formally defining a
taxonomy of vandalism~\cite{Viegas2004,Priedhorsky2007,Chin2010},
but nearly all research on vandalism detection uses one of a small
number of (convenient) definitions for purposes of obtaining an
annotated corpus: \textbf{manual annotation} uses human intelligence
to infer the intentions of the
editor~\cite{Potthast2008,Chin2010,West2010,Potthast2010a},
\textbf{reverts} are notations by the community when it feels that
vandalism has taken place~\cite{Smets2008,Itakura2009,Belani2010},
\textbf{rollbacks} are disapprovals by Wikipedia
Administrators~\cite{West2010},
and \textbf{edit quality} generalizes the idea of measuring the
sentiment of the community~\cite{Adler2007,Druck2008}.
There is an obvious trend going from \textit{manual} to
\textit{automatic} annotation, but equally important is to observe
that there is a trend from external judgement, to internal explicit
judgement, to internal implicit judgement.
Ultimately, it is the community itself which decides what is
vandalism (\eg the stark contrast between the communities of
Slashdot\footnote{\url{http://slashdot.org}} and
Hacker News\footnote{\url{http://news.ycombinator.com}}),
and this community standard is likely to change over time
(often described as the ``signal-to-noise'' ratio of the community;
examples of changing communities include USENET and Slashdot).
This argues strongly in favor of automated methods for measuring
the reaction of the community, and highlights the idea that vandalism
detection is a specialized form of trying to measure the ``noise'' in
a community.

The earliest attempts at vandalism detection within the Wikipedia come
directly from the user community and try to encode a human intuition
of vandalism detection into an expert system (some examples
include~\cite{wiki:AntiVandalBot,wiki:MartinBot,wiki:ClueBot,Carter2007}).
The largest disadvantage to this class of solution is that building
an expert system requires extensive human labor to produce the manual
annotation and analysis required to derive custom rules from the
annotation.
Primarily, the rules developed are based on features of the actual
content of the edit rather than on metadata (\eg an edit containing
only capital letters is indicative of vandalism).

The idea that the content is the primary source of features that
reveal the intent of the author is a natural one, and is investigated
by several different research groups
(\eg~\cite{Potthast2008,Smets2008,Druck2008,Itakura2009,Chin2010}).
Casting the problem as a machine learning binary classification
problem, Potthast~\etal~\cite{Potthast2008} manually identify and
inspect 301~incidents of vandalism to generate a feature set based on
metadata and content-level features and build a classifier using
logistic regression.
Smets~\etal~\cite{Smets2008} applies the ``naive bayes'' machine
learning technique to a bag-of-words model of the edit text.
Chin~\etal~\cite{Chin2010} delve deeper into the field of
natural language processing by constructing statistical language
models of an article from its revision history.
(On the topic of manual annotation, they also describe how supervised
active learning can help the training process by requesting
annotations for examples which will make a significant difference to
the algorithm.)

A different way of looking at the content approach is the
realization that appropriate content somehow ``belongs together,'' and
one way to measure that is through compression of the successive
revisions of an article~\cite{Smets2008,Itakura2009}.
If inappropriate content is added to the article, then the compression
level is lower than it would be for text which is similar to text
already in the article.
This is much more powerful than the bag-of-words model, because
phrases are significant and lead to better compression; nonsensical
sentences that include some key words will not compress as well.
A significant drawback of these compression techniques is that they
require manipulation of the content of a large number of revisions
from the article being edited.

Content-based analysis has the burden of having to
inspect potentially large edits, but the alternative is to depend
on the paucity of information available in the metadata ---
many previous works have some small dependence on metadata
features~\cite{Potthast2008,Druck2008,Belani2010}, but only
as far as it encoded some aspect of human intuition about vandalism.
Drawing inspiration from other areas of research,
West~\etal~\cite{West2010} demonstrate astonishing results because
they are based entirely on metadata (some of which is processed into
\textit{reputations}) that indicate there is more relatedness between
vandals than is readily apparent to the human eye.
One particularly interesting result was that using IP geolocation
to cluster users leads to better predictions, presumably because
anti-US sentiment also tends to cluster by geographic region.

The first systematic review and organization of features appears
by Potthast~\etal~\cite{Potthast2010b} as part of the competition
associated with the PAN~2010 Workshop on vandalism
detection.
Belani~\cite{Belani2010} includes several metrics for evaluating
predictors, and Potthast~\etal take up the discussion with a thorough
comparison of nine competitors using both the area under the
precision-recall curve and the area under the receiver operating
characteristic curve.
Potthast~\etal conclude their analysis by building a meta-classifier
based on the nine entries and discover that the result performs
significantly better than any single entry.

User reputation systems~\cite{Zeng2006,WikiMTWtrust06,Adler2007}
have been previously proposed as an underlying technology for
vandalism prevention or detection, and the second place entry in
the PAN~2010 competition was a system based on the
WikiTrust project~\cite{Adler2010b}.
In that entry, the WikiTrust user reputation system was not
directly used as a feature due to not having a historical record
of the reputation values.
The work presented in this chapter extends~\cite{Adler2010b} by
tracking the historical user reputation values and using that
as an additional feature to the machine learning algorithms.

The winner of the PAN~2010 competition, by a notable margin, was
an entry by Mola-Velasco~\cite{Mola2010} that extended the features
originally proposed by Potthast~\etal~\cite{Potthast2008}.
This entry was composed of 21~features (the largest in the
competition) that comprehensively model the content of the edit,
including features that rated use of language, formatting of text,
compressibility with earlier text, spelling,
and the size of the edit.

The combining of features used by Mola-Velasco~\cite{Mola2010},
WikiTrust~\cite{Adler2010b} and West~\cite{West2010}
is explored in~\cite{Adler2011a}.
That work improves on earlier results, and categorizes features
according to the difficulty of analysis.



\section{Experiment}

To test our user reputation system's effectiveness, we
evaluate its performance as part of a vandalism
detection system.
The PAN-WVC-10 is a corpus of 32,439 edits manually
annotated by at least three people as part of an Amazon
Mechanical Turk task~\cite{Potthast2010a}.
Of these edits, 2,394 (7.97\%) were classified as vandalism.

Our goal is to incorporate the WikiTrust user reputations
in building a model to predict vandalized edits in
the PAN-WVC-10 corpus.
We modified the WikiTrust
code base to output the reputation score of each author
over time, creating a chronology of reputation scores.
This revised code was used to process the English Wikipedia
dump of 30-Jan-2010, which includes the time period of
edits from the PAN-WVC-10 corpus.

Using the timestamp of each edit in the PAN-WVC-10 corpus,
we locate its position in the chronology and then work backwards
to find the most recent reputation score of the author
that appears \textit{before} the edit was made.
These reputation scores are then merged with the collection
of features used by the WikiTrust submission in the PAN~2010
competition~\cite{Adler2010b}, and used to build a model to predict
whether an edit is vandalism.
The analysis we present here considers only \intro{zero-delay} vandalism
detection; that is, detecting vandalism using only features which are
available at the moment an edit is made.

\newcommand{\sign}{{{\textrm{sign}}}}

\subsection{Classifier and Features}

Our vandalism detection tool uses the open source machine learning
package Weka~\cite{Weka09} to build and evaluate a prediction model from
our features.
In the original PAN~2010 competition, we used an alternating decision
tree (ADTree) classifier~\cite{Adler2010b} because it performed well and
generates models that are easy to interpret.
As part of our collaboration in combining multiple vandalism detection
systems~\cite{Adler2011a}, we evaluated the performance of the
WikiTrust features\footnote{In addition to the zero-delay features
described in~\cite{Adler2010b}, a feature measuring the length
of the article for~\cite{Adler2011a}.}
using a random forest classifier and discovered that the performance was
increased.
Anonymous users have no reputation in the WikiTrust system, and receive
a reputation score of zero within the feature set.

As in~\cite{Adler2011a}, we use the random forest algorithm to build our
prediction model and the standard ten-fold cross-validation to evaluate
the predictive ability.
The Weka command line we used was:
%
{\small
\begin{verbatim}
    weka.classifiers.meta.FilteredClassifier
      -t combined-features.arff -p 1 -F
      "weka.filters.unsupervised.attribute.Remove -R 1,2,3"
      -W weka.classifiers.trees.RandomForest -- -I 500
\end{verbatim}
}

We select a set of features based on the information that is readily
available within the pre-existing WikiTrust system, and only those which
are available at the instant an edit is made.
The purpose of our experiment is to answer the question,
``does the WikiTrust reputation computation provide information about
whether an edit is vandalism?''
In order to make that judgement, we use the same features chosen
in~\cite{Adler2010b} and add the WikiTrust reputation score.
The total set of features we used are:
%
\begin{itemize}

\item \textbf{Author reputation [Reputation].}
Vandalism tends to be performed
predominantly by anonymous or novice users, both of which have
reputation zero in the system.

\item \textbf{Author is anonymous [Anon].}

\item \textbf{Time interval since the previous revision [Logtime\_prev].}
We compute the quantity $\log(1 + t)$, where $t$ is amount of time since
the preceding revision of the same article.

\item \textbf{Hour of day when revision was created [Hour\_of\_day].}
We expect that the time of day at which the revision was created
might have some influence on the frequency of vandalism.
This did not have much influence in our previous work~\cite{Adler2010b},
but a more sophisticated version proved to contain much
information~\cite{West2010}.

\item \textbf{Delta [Delta].}
This feature measures the edit distance $d(r, r^{-})$ between the
revision being examined and the previous revision in the same article.

\item \textbf{Revision comment length [Comment\_len].}
The length of the comment attached to the revision.
It seems unlikely that vandals would provide a comment,
so we included it as a trivial feature to compute.

\item \textbf{Previous text trust histogram [P\_prev\_hist0 \ldots P\_prev\_hist9].}
Whenever a revision is created, WikiTrust computes the reputation (which
we also call \intro{trust} when applied to text) of each word of the
revision, where the reputation is an
integer in the interval $0, \ldots, 9$~\cite{Adler2008b}.  The
reputation of a word indicates how much the word has been revised by
other reputable authors; in particular, words that have are inserted or
moved by authors without reputation (including both novice and
anonymous authors) are assigned a reputation of zero.  When the revision is
created, WikiTrust also computes a ten column histogram detailing how
many words of the revision have each of the ten possible reputation
values, and stores the histogram in the database in an entry associated
with the revision.
We normalize the histogram (so that
the columns sum to one) of the previous revision in the same article history.

\item \textbf{Current text trust histogram [Hist0 \ldots Hist9].}
The values of the text trust histogram for the current revision,
without any normalization.

\item \textbf{Histogram difference [L\_delta\_hist0 \ldots L\_delta\_hist9].}
For each possible text trust value $i \in \{0, \ldots, 9\}$, we also
computed the value of
\begin{equation}
\log(1 + |h(i) - h^{-}(i)|) \cdot \sign(h(i) -
h^{-}(i)),
\end{equation}
where $h$ is the text trust histogram for the current
revision, and $h^{-}$ is the text trust histogram for the previous
revision.

\end{itemize}
%





\section{Questions}


The minor question I had was, what is the performance of just Mola+WT?
That is, Potthast got PR-AUC=0.78 by using a random-forest(1000 trees,
4 features) -- can we beat 78% by using the raw features rather than
the output of the classifier?  (I expect not, but I suspect we are
close.)  Is the solution (M+WT) a "sufficient cover set" of the
features explored in PAN2010, and STiki brings a new dimension of
features?

Another minor question: is there a way for us to rank the importance
of features?  Santiago almost does this in his PAN submission, and the
answer is tantalizing only to social scientists: does it matter more
what you say, or who you are?

Had crazy idea while looking at graphs for joint vandalism paper:
\begin{quote}
BTW, I don't know if you looked at the graphs/data that Santiago added today.  He was right in pointing out that they are "interesting"; see Fig 4, historical sets.  L+M+T and L+M+R+T perform nearly identically, suggesting that R seems to capture the same notions as L+M+T.  Also, the R and M+T curves are nearly identical (which is kinda begging for a comparison with an L+R curve).  Probably this will change because of the 3 I reclassified, but it's amazing how closely they track each other.

To answer my first question (same notion, or just same predictive power), we'd have to compare which edits are predicted for each threshold, and see how well they correlate.  Which seems unlikely for us to do for this paper...

To answer the second question, we could do "leave one out" calculations for each feature, and see if the lines cluster or scatter.  Hmm, that's interesting, because I thought my "discrete steps" imagery was crazy and would be impossible to test.  Anyway, for sure this is not possible to do for this paper.
\end{quote}


    \section{Features}
        Is it possible to visualize the amount of information
        contained in each feature?
        Has previous work always been a blend of metadata, NLP,
        and reputation?
    \section{Edits}
        Which edits did we guess correctly, where others didn't?
        Why did we guess correctly?  Was it because they were edits
        by old users?

    \section{Future Work}
        Ultimately, Druck~\etal are right when they proclaim that
        vandalism detection must look at the content of the
        edit~\cite{Druck2008}.
        Although metadata methods for identifying bad edits are
        fairly capable, they can always be fooled by a sufficiently
        motivated vandal.
        Vandalism detection will eventually incorporate better
        language models such as those used at Google,
        and will also have specialized topic models (someone did
        something like this!  and compression techniques are
        related).


