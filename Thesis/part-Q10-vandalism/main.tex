\chapter{Vandalism Detection}
\label{ch:vandalism}

\begin{quote}
\textit{This chapter explores a variation of the ideas
published as~\cite{Adler2010}.}
\end{quote}

Multiple studies have found that roughly 7\% of edits are
vandalism~\cite{Potthast2008,Potthast2010a} \mynote{I think
that Luca cites another}.
Due to the preservation of the entire revision history,
anyone can revert an edit, and a group of volunteers scans the
list of recent changes to catch vandalism
quickly~\cite{wiki:RCPatrol}.


\mynote{Running batch process in redherring screen 6.
  Sat, Feb 5.}

\section{Related Work}

As the Wikipedia becomes a standard resource for the Internet
public, there is rising interest in quality measures.
A series of incidents show that the Wikipedia can be manipulated,
despite the ``many eyes'' reviewing the site:
a prank biography~\cite{Seigenthaler05,NewYorkTimes05a,NewYorkTimes05b},
congressional aides adjusting political
biographies~\cite{TheSun06,NewZelandHerald06,BBC06},
a user pretending to be a professor~\cite{BBC07},
and a slew of other self-interested parties making
inappropriate edits~\cite{Wired07,Wikiscanner07,NPR08}.
Articles have been written questioning the
general credibility of the Wikipedia~\cite{NewYorkTimes06,TheNewYorker06},
and a scientific study addressing the question
has been published~\cite{Giles05}.

Today, there are several lines of research pursuing
vandalism detection specific to the Wikipedia~\cite{Potthast2010b}.
These solutions all apply machine learning techniques
to annotated data sets, treating the task as a
``supervised learning'' problem.
At the time our research was started, there were no annotated
data sets available.


The idea of assigning trust to specific sections of text of Wikipedia
articles as a guide to readers has been previously proposed in the scientific
literature~\cite{WikiMTWtrust06,Cr06,McGuinness06}, as well as in white
papers~\cite{King07} and blogs~\cite{PaoloMassa07}; these papers also contain
the idea of using text background color to visualize trust values.


Other studies of Wikipedia quality have focused on trust as
article-level, rather than word-level, information.
These studies can be used to answer the question of whether an
article is of good quality, or reliable overall, but cannot be used to
locate within an article the portions of text which deserve the most
scrutiny, as our approach can.
In~\cite{WikiTrust06}, which inspired~\cite{McGuinness06}, the
revision history of a Wikipedia article is used to compute a trust
value for the entire article.
In~\cite{Emigh05b,Mingus07}, metrics derived via natural language processing
are used to classify articles according to their quality.
In~\cite{Lih04}, the number of edits and unique editors are used to
estimate article quality.
The use of revert times for quality estimation has been proposed
in~\cite{Viegas04}, where a visualization of the Wikipedia editing
process is presented; an approach based on edit frequency and dynamics
is discussed in~\cite{WilkinsonHuberman07}.
There is a fast-growing body of literature reporting on
statistical studies of the evolution of Wikipedia content,
including~\cite{Viegas04,Voss05,Ortega07}; we refer to~\cite{Ortega07} for an
insightful overview of this line of work.
The history flow of text contributed by Wikipedia authors has
been studied with flow visualization methods in~\cite{Viegas04}.


\section{Experiment}

To test our user reputation system's effectiveness, we
decided to evaluate its performance as part of a vandalism
detection system.
The PAN-WVC-10 corpus is a corpus of 32,439 edits manually
annotated by at least three people as part of an Amazon
Mechanical Turk task~\cite{Potthast2010a}.
Of these edits, 2,394 (that is, 7.97\%) were classified as vandalism.

Our goal is to include the WikiTrust user reputations as a
feature in building a model to predict vandalized edits in
the PAN-WVC-10 corpus.
We instrumented the batch processing mode of the WikiTrust
code base to output the reputation score of each author as
it was adjusted over time, creating a chronology of reputation scores.
This revised code was used to process the English Wikipedia
dump from 30-Jan-2010, which includes the time period of
edits from the PAN-WVC-10 corpus.

Using the timestamp of each edit in the PAN-WVC-10 corpus,
we locate its position in the chronology and then work backwards
to find the most recent reputation score of the author
that appears \textit{before} the edit was made.
These reputation stores are then merged with the collection
of features used in the PAN~2010 competition~\cite{Adler2010},
and used to build a model to predict whether an edit is vandalism.
We used Weka~\cite{Weka09} to construct the models using both
the Random Forest and Alternating Decision Tree algorithms.




\section{Questions}


The minor question I had was, what is the performance of just Mola+WT?
That is, Potthast got PR-AUC=0.78 by using a random-forest(1000 trees,
4 features) -- can we beat 78% by using the raw features rather than
the output of the classifier?  (I expect not, but I suspect we are
close.)  Is the solution (M+WT) a "sufficient cover set" of the
features explored in PAN2010, and STiki brings a new dimension of
features?

Another minor question: is there a way for us to rank the importance
of features?  Santiago almost does this in his PAN submission, and the
answer is tantalizing only to social scientists: does it matter more
what you say, or who you are?

Had crazy idea while looking at graphs for joint vandalism paper:
\begin{quote}
BTW, I don't know if you looked at the graphs/data that Santiago added today.  He was right in pointing out that they are "interesting"; see Fig 4, historical sets.  L+M+T and L+M+R+T perform nearly identically, suggesting that R seems to capture the same notions as L+M+T.  Also, the R and M+T curves are nearly identical (which is kinda begging for a comparison with an L+R curve).  Probably this will change because of the 3 I reclassified, but it's amazing how closely they track each other.

To answer my first question (same notion, or just same predictive power), we'd have to compare which edits are predicted for each threshold, and see how well they correlate.  Which seems unlikely for us to do for this paper...

To answer the second question, we could do "leave one out" calculations for each feature, and see if the lines cluster or scatter.  Hmm, that's interesting, because I thought my "discrete steps" imagery was crazy and would be impossible to test.  Anyway, for sure this is not possible to do for this paper.
\end{quote}


    \section{Features}
        Is it possible to visualize the amount of information
        contained in each feature?
        Has previous work always been a blend of metadata, NLP,
        and reputation?
    \section{Edits}
        Which edits did we guess correctly, where others didn't?
        Why did we guess correctly?  Was it because they were edits
        by old users?

    \section{Future Work}
        Ultimately, Druck~\etal are right when they proclaim that
        vandalism detection must look at the content of the
        edit~\cite{Druck2008}.
        Although metadata methods for identifying bad edits are
        fairly capable, they can always be fooled by a sufficiently
        motivated vandal.
        Vandalism detection will eventually incorporate better
        language models such as those used at Google,
        and will also have specialized topic models (someone did
        something like this!  and compression techniques are
        related).


