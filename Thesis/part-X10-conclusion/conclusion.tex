\section{Conclusion}

\begin{comment}
Used to filter too much information, as in page rank.
But also used to encourage desirable behavior
(and discourage undesirable behavior).

Does a vandal to Wikipedia care that his reputation is
tarnished?  What kind of punishment is this?

Applications: peer review, text trust, sexual offender registry, job references.
Can we use reputation to figure out whether people
fit into a company culture?

What if web search rankings were done incrementally
instead of globally?  Would this give higher reputation
to \textit{fresh} parts of the web, and gracefully
phase out crufty parts?  Crucial to this would be the
need to update reputation based on page views, or
people following links from Facebook/Google, so that
we know that a static page is still actually relevant.
\end{comment}

What is reputation?  We know that it is a value, because your reputation goes
up and down. It can be good or bad.

What is less obvious is that reputation is multidimensional.  Some writers have
a good reputation for writing engaging essays (\eg Malcolm Gladwell), and some
have a good reputation for writing good children's fiction (\eg
J.\thinspace K.\thinspace Rowling),
but most of us would be doubtful if Gladwell and Rowling were to swap careers.
Interestingly, we also relate reputation between
dimensions --- for example, someone
with a good reputation for community involvement and hard work we also presume
to have a high reputation in regards to his behavior at home (for example,
marital fidelity).

After so many years of working with reputation, I have come to believe that
internally we use reputation as something like a probability. That is, all
other things being equal, reputation is the chance that you will meet some
standard of performance in a situation that calls for it.  The trick is that
things are never ``equal'' from situation to situation, so that there are
unknown components of a reputation that we construct from what
we know of a person's background and culture.
The most direct example of this is racial stereotyping applied to
someone we have never met before.

Although reputation might lead us to jump to incorrect conclusions about
someone based on limited information, they also help us to filter and
sort through the overwhelming amount of information we are presented
with each day.
A modern application of reputation is in sorting search
results (\eg PageRank~\cite{Page1999}), and we see companies
like Google and Facebook incorporating viewer specific information
to provide a greater context (of past search history and of
the social graph) to the reputation function, providing a
\textit{personalized} experience.
Imagine if these same ideas could be applied to other problems~\cite{Adler2011b},
such as peer review of academic papers~\cite{Adler2010a}
or job candidate evaluation through references.

