\section{Introduction}

The Future is inevitable.

History abounds with examples when the time was ripe for a particular idea.
Some famous instances are the development of calculus (by Leibniz and Newton),
the theory of evolution (by Darwin and Wallace), and the invention of
the integrated circuit (by Kilby and Noyce).
In the marketplace of ideas, memes compete for our attention and
are eventually absorbed into the great cosmic unconscious until
they are just ``in the air.''

When our group began work on this research into reputation systems for
the Wikipedia, it was focused on the
question of how to tell ``good guys'' from ``bad guys'' on the Wikipedia.
Right away, the investigation expanded to edit quality, and the question of what
\textit{is} quality in a collaborative work where there is no single
guiding hand.
We then realized that our work was a
way to quantify the consensus of a group of people.
This is a powerful view, not only because of the potential
applications, but also because now we know that it is \textit{possible}
to make such a measurement.
Economists long ago discovered ``revealed preferences''
as a way to surreptitiously measure the internal world of the
mind~\cite{Samuelson1938,Varian2006}, and this work widens that doorway.

It was some years later that I began work at Fujitsu Labs of America,
using statistics for word usage models and medical trials,
when I came to a new realization.
In the flood of the data deluge (which Wikipedia analysis falls smack
into the middle of), statistical summaries are how we abstract that data
and come to an ``understanding'' of it.
Our reputation system is a manifestation of our
rough understanding of the kind of useful Wikipedia content we aim to
promote.
And so I came to think that \textit{statistics} will be the
most important field in the $21^{st}$ century.

Most recently, our work on vandalism detection and my work with the
emerging Quantified Self group\footnote{\url{http://quantifiedself.com}}
has led me to realize that
\textit{data mining} is the new face of statistics; simple summarization
of so much data is not useful, but data mining gives us the power to
pull at the different threads of individuality and cluster like with like.
We go from summaries to context-dependent probabilities.

Through a chance conversation with Bayesian statistician
Owen Martin, these thoughts gelled into the notion that reputation
was not just some number of stars by your name --- it is the probability
that you will do or say something, given the context of your demographics
(\ie the background culture that helped shape you),
your past history and the present moment.
That is, the models built by data mining algorithms are a form of
reputation constructed around the description of user
behavior.\footnote{Reputation systems are both descriptive and
prescriptive~\cite{Adler2007}.  Machine learning algorithms compute a
description of past behavior but don't appear to include a prescriptive
component; the key to reconciling these views is in the choice of what
to predict.  For example, predicting whether an edit will be ``long
lived'' naturally takes negative behaviors into account.}
This work is just the first fumbling footsteps in that line of thinking.

