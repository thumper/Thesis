\section{Introduction}

The Future is inevitable.

History abounds with examples where the time was ripe for a particular idea.
Two famous ones are the development of calculus (by Newton and Leibniz)
and the theory of evolution (by Darwin and Wallace).
In the marketplace of ideas, memes compete for our attention and
are eventually absorbed into the great cosmic unconscious until
they are just ``in the air.''

When we began work on this research, it seemed to be about a very focused
question trying to tell ``good guys'' from ``bad guys'' on the Wikipedia.
Right away, the investigation expanded to quality and the question of what
\textit{is} quality in a collaborative work where there is no single
guiding hand.
It was then that we realized that one way to view our work was as a
way to quantify the consensus of a group of people.
This is an incredibly powerful view, not only because of the potential
applications, but also because now we know that it is \textit{possible}
to make such a measurement.
Economists long ago discovered ``revealed preferences''
as a way to surreptitiously measure the internal world of the mind,
and this work widens that doorway.

It was some years later that I began work at Fujitsu Labs of America,
using statistics for word usage models and medical trials when I came to
a new realization.
In the flood of the data deluge (which Wikipedia analysis falls smack
into the middle of), statistical summaries are how we abstract that data
and come to an ``understanding'' of it.
A reputation system such as we have built is a manifestation of our
rough understanding of which work is useful in the Wikipedia and
which work we would like to encourage.
And so I came to think that \textit{statistics} will be the
most important field in the $21^{st}$ century.

Most recently, our work on vandalism detection and my work with the
emerging Quantified Self group\footnote{\url{http://quantifiedself.com}}
has led me to realize that
\textit{data-mining} is the new face of statistics; simple summarization
of so much data is not useful, but data-mining gives us the power to
pull at the different threads of individuality and cluster like with like.
We go from summaries to context-dependent probabilities.

And through a chance conversation with Bayesian statistician
Owen Martin, these thoughts gelled into the notion that reputation
was not just some number of stars by your name --- it is the probability
that you will do or say something, given the context of your demographics
(\ie the background culture that helps to shape who you are),
your past history and the present moment.
Data-mining today is the process of constructing probabilities and
trying to determine what contexts are the useful ones to discriminate by.
And reputation, I feel, is merely data-mining on streams of data
rather than static datasets.

