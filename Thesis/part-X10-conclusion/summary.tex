\section{Summary of this Work}

The creation of a collaborative repository of knowledge is
one of those inevitable ideas; we have been striving towards it
for all of civilization.
Our history is one of ever increasing communication and collaboration
in our ambition to master our environment, from books and libraries,
to the telephone and the Internet.
The Wikipedia seems a natural result of this history.
And perhaps just as inevitably, there are people who seek to
mar the collective goals of the Wikipedia for their own gain and amusement.

The goal of our research was to develop a reputation system
that could assist users by flagging content added by users who
did not have a good track record, making the task of understanding
the short-term history of an article easier to grasp with a glance.
In working towards that goal, this research makes the following contributions:
\begin{enumerate}

\item Two methods for determining the ``quality'' of a contribution.
  We evaluate these methods (and the underlying difference algorithms)
  using the PAN~2010 vandalism detection corpus, and find that both
  measures perform much better than a random vandalism detector.
  Our \intro{text longevity} measure uses authorship information to
  compute the amount of contributed text and how it survives over
  future revisions (its ``rate of decay'').
  Our \intro{edit longevity} measure uses edit distances to estimate
  how much ``work'' of an edit goes to making an article more like a
  future instance of the article.

\item A reputation system for authors,
  which adjusts the reputation of an author based on quality feedback
  from later authors using the ideas of text and edit longevity.
  Our evaluation shows that content by low-reputation authors is
  four times as likely as the average to be short-lived, and that better
  precision-recall can be achieved when used as part of a vandalism
  detection system.

\end{enumerate}

During the course of developing and studying these primary contributions,
we have also had several smaller accomplishements with enabling other
technologies based on our reputation data:
\begin{description}
\item \textbf{text reputation}
  Our group developed a reputation system for text~\cite{Adler2008b},
  and created a visualization tool that is available to users as a Firefox
  plugin.\footnote{\url{https://addons.mozilla.org/en-US/firefox/addon/wikitrust/}}

\item \textbf{vandalism detection}
  The work of Chapter~\ref{ch:vandalism} grew out of an entry into a
  vandalism detection competition.
  That effort led to a joint publication~\cite{Adler2011a} presenting
  results that outperformed previously published results.
  Our features have also been incorporated
  into the STiki vandalism detection tool~\cite{wiki:STiki}.

\item \textbf{revision selection}
  The Wikipedia Offline project published a CD-ROM of a selection of
  Wikipedia content, which is distributed to school children across
  America.
  Due to space constraints, the project only selects one revision for
  each article, but manually reviews each choice to check for vandalism
  or questionable changes.
  We developed a system based on our vandalism detection work (and
  coining the term ``historic vandalism detection'') to
  identify revisions that were likely to be vandalism, narrowing the
  list of revisions that needed manual review.

\end{description}

As part of this work, we also investigated text difference algorithms
and outlined several design issues in how author attribution is
determined in a collaborative document.
With respect to text difference algorithms, we ultimately discovered
that the difference algorithm did not have a very dramatic impact on the
performance of the edit longevity measure (shown in
Tables~\ref{tab:editlongbyed5} to~\ref{tab:editlongbyed1});
we would have achieved substantially similar results simply using the
fastest performing algorithm.


\subsection{Future Work}

A critical part of the training received in graduate school
is the ability to ask questions that creatively expand on your
existing work.
There are a great number of questions left to explore in the
context of WikiTrust, some of which are described here.

\begin{description}
\item[\textbf{category reputation}]
    This is probably the one idea that we hear the most from audiences.
    A person can be
    an expert in one area but not another, but some people don't
    recognize their lack of expertise.
    By keeping track of separate reputation scores for every category,
    the system should be more able to predict the longevity of an edit.
    It is worth noting that, within the Wikipedia community, there
    are those editors which specialize in grammar and style rather than
    subject areas, so that there is some extra work that must go
    into classifying the type of edit (\eg~\cite{Fong2010}).
    Note that the STiki project~\cite{West2010} includes ``category
    reputation'', but it tracks a reputation for each category globally;
    here we mean that reputation is multidimensional and an author can
    have differing reputations in different categories.

\item[\textbf{match quality}] Our greedy text differencing algorithm
    uses a match quality function to prioritize which matches are
    preferable according to the criteria described in Chapter~\ref{ch:diff}.
    The evaluation in Chapter~\ref{ch:editquality}
    suggests that as long as length
    is the primary discriminant, there is not too much difference
    between the different quality functions.
    The evaluation used is one based on the resulting predictive ability
    of edit quality, but is there some better way to evaluate a difference
    algorithm whose aim is to model the human view of a text edit?

\item[\textbf{reputation as probabilities}]  This view of reputation
    clarifies how to best interpret the results.
    It also introduces the significant question of how should probabilities
    be adjusted as new information becomes available.
    Owen Martin's work on estimating bug counts provides a nice example:
    you are trying to build a rocketship, and you run many
    tests to try to uncover bugs~\cite{Martin2011}.
    If the test succeeds, your estimate
    of the number of bugs (which is a form of reputation: ``what is
    the probability that the system will fail?'') goes down.
    If the test fails, how do revise your estimate of the number
    of bugs?  And once the engineers have fixed the problem, can
    you be sure they haven't introduced new problems?

\item[\textbf{authorship}]
    We devised an algorithm for determining the ``authorship'' of words in
    revisioned documents.\footnote{This work is being explored further
      by the German chapter of the Wikipedia: \\
        \url{http://de.wikipedia.org/wiki/Benutzer:NetAction/WikiTrust}}
    This algorithm is a refinement of existing difference algorithms
    already well-known in the literature, but we additionally outline
    some of the design considerations for this particular application.
    As described in the concluding remarks of
    Chapter~\ref{ch:diff}, there are further refinements to be
    considered in how to assign authorship in a collaborative work.
    Identifying common idioms in the language (\eg through the use of
    tf-idf~\cite{Jones1972}) and common phrases within a topic area
    are two cases where authorship needs to be more carefully considered.
    The greatest challenge here is
    determining a suitable evaluation for comparing solutions.

\end{description}

