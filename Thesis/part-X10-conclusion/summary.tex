\section{Summary}

\subsection{Measuring Group Consensus}

\subsection{Conclusion}

What is reputation?  We know that it is a value, because your reputation goes
up and down. It can be good or bad.

What is less obvious is that reputation is multidimensional.  Some writers have
a good reputation for writing engaging essays (\eg Malcolm Gladwell), and some
have a good reputation for writing good children's fiction (\eg J. K. Rowling),
but most of us would be doubtful if Gladwell and Rowling were to swap careers.
Interestingly, we also relate reputation dimensions -- for example, someone
with a good reputation for community involvement and hard work we also presume
to have a high reputation in regards to his behavior at home (for example,
marital fidelity).

After so many years of working with reputation, I have come to believe that
internally we use reputation as something like a probability. That is, all
other things being equal, reputation is the chance that you will meet some
standard of performance in a situation that calls for it.  The trick is that
things are never ``equal'' from situation to situation, so that reputation is
merely an input to some more complicated model of human behavior.  Our work on
vandalism detection takes this approach, using reputation as a feature in
machine learning, and might loosely be described as ``trust but verify.''
\mynote{Not trust but verify.}

