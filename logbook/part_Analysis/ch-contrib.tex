\chapter{Author Contributions}

\section{Reimplementation}

\subsection{31-Jan-2011}

I started to reimplement the author contribution patches that
Vishwa had written, when I discovered/remembered that there's a
new way to do it in the current WikiTrust
code.\footnote{\url{http://www.wikiwtrust.net/write-your-own-analysis}}

This led me to creating my own analysis file,
\texttt{Fullcontribution\_analysis.ml}.
I have implemented the EditCount analysis, but am
still trying to learn how to run the code.
\begin{verbatim}
$ mkdir tmp
$ export OCAMLRUNPARAM=b
$ ./evalwiki -d tmp -eval_fullcontrib ../remote/analysis/dump-SantaCruzBeachBoardwalk.xml
\end{verbatim}
This gives an exception in \texttt{timeconv.ml}, so I need
to investigate what is happening.
Ah, it turns out that the time format generated by
\program{downloadwp2xml} was incorrect; I've fixed that
and now it works as I expect.

\subsection{19-Mar-2011}

After talking to people and estimating a schedule, I decided
to abandon recreating any of our previous works.

\section{Survival Graphs}

\subsection{20-Mar-2011}

As part of trying to explain how text longevity was calculated,
I wanted to show some actual graphs of text survival times.
I wrote a program to take the dump of a single article and compute
the text survival times for each revision.
\begin{verbatim}
$ time nice ./computeSurvival.pl dump-GeorgeWBush-fixed.xml > track-GWB-3.txt
\end{verbatim}
This generates a file that lists, for every revision, how many words
are attributed to earlier revisions.

I wrote another program to help me find the best looking graph.
\begin{verbatim}
$ ./graphSurvival.pl track-GWB-3.txt
\end{verbatim}
This will display gnuplot output at 3~second intervals for each revision.

While looking at the output for the ``George W. Bush'' article, I realized
the flaw in our formulation of text longevity.
We use the actual survival values to compute a quality score,
but when you examine the graphs, the computed quality value
always ends up looking like a straight like through the data.
This is because the exponential we've defined always approaches
zero gradually, not any higher value.

I was really expecting to see a curve that \textit{looked} exponential,
but to do this we need to subtract the final value as a baseline.
Doing this will give curves that look exponential, but this changes
the meaning of the quality score: a score of zero means that the text
jumped to it's final state right away.
That is, this variation is a measure of how quickly the text stabilizes.
We don't really care about that, so be sure to keep the original definition.

