\chapter{Contribution Quality}

\section{Introduction}

\subsection{19-Mar-2011}

The basis for this chapter was developed in the original
WikiTrust paper, and then expanded in the AuthorContrib
paper.

Although the AuthorContrib paper shows that the different
proposed formulas can be used to partition the authors
(and thus are useful as features to a machine learning algorithm),
there is no attempt to perform any kind of evaluation.
In part, this is because we believe that each formula calculates
something different; another motivation was that there was
no benchmark of any kind to compare against.
The publication of the PAN2010 corpus, which annotates a set
of edits as either ``vandalism'' or ``not vandalism,''
now gives a benchmark which we can analyze the different
formulas against.


\section{Edit Distance}

\subsection{Feb-2010}

While speaking with Gary Wolf about Quantified Self materials,
we got to discussing wikipedia.
He told me a story about an article
(``The Battle of Genine'')
which he heard was being quietly manipulated by the Israelis.

\subsection{14-Apr-2010}

Erica has sent me a link to an article which has many edit
wars in it.

\url{http://www.doublex.com/blog/xxfactor/bieber-fans-vs-wikipedia}

\section{Edit Quality}

\subsection{19-Mar-2011}

The experiment that I'm proposing is to compute the edit longevity
and text longevity for each of the PAN2010 edits.
Then, for each metric, construct two histograms:
\begin{enumerate}
\item A histogram showing how many vandalism edits fall within
    each quality bucket.  Essentially, what is the distribution
    of vandalism over quality.
\item A histogram showing how many non-vandalism edits fall within
    each quality bucket.  Essentially, what is the distribution
    of good edits over quality.
\end{enumerate}
This is very like some histograms that we already used in
the AuthorContrib paper, but this time we have a ``golden annotation''
dataset.
I would also like to somehow compute a correlation, but that
feels odd since vandalism is a binary categorization;
maybe I can show the inverse histograms and compute the
average quality?

