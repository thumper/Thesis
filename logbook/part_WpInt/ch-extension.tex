\chapter{Mediawiki Extension}

\section{Introduction}

\subsection{7-May-2009}

Ian has been the steward of the \program{mediawiki}
extension, but with him on leave Luca and I have been
trying to recall the details of the architecture.

\section{A Meeting}

\subsection{7-May-2009}

Luca and I had a meeting, where much of the discussion was
in trying to understand the organization of Ian's code
for the extension.

In trying to reason out the functionality, we determined that
there are three use cases:
\index{local mode}\index{remote mode}\index{wmf mode}
\begin{enumerate}
\item[local] - the wiki data and coloring is completely local on
	a single machine.
\item[remote] - the extension must color data as a remote service.
	There is no notification when new revisions are added to
	the wiki.
\item[wmf] - the extension colors data as a remote service,
	but receives revision notifications via an \textit{edit hook}.
\end{enumerate}

\subsubsection{Local Mode}
\index{local mode}

Normal wikis will just use ``local mode,'' where the WikiTrust extension
is integrated into the wiki and can detect revisions as they added
into the database.
In this case, the extension code would directly trigger the
OCaml code to color the revisions as they are made.

\subsubsection{Remote Mode}
\index{remote mode}

Our original vision for the GeneWiki project was to run the
system in ``remote mode.''
In this use case, we would run a mediawiki instance at UCSC
with the WikiTrust extension in remote mode.
GeneWiki users of Wikipedia would have a special \textit{skin}
available which would allow them to query our instance for
colored revision information.

The key point of this mode is that is allows administrative separation
between the host wiki and the colorized wiki.
We would not rely on notifications from Wikipedia about new revisions
that needed coloring; instead, we would color articles
on demand as they were requested by the GeneWiki users.

On the UCSC side, a \program{mediawiki} instance is still required
because the colorized article needs to be converted into HTML.
The GeneWiki skin can only replace the HTML on the page it was
trying to show.

\subsubsection{WMF Mode}
\index{wmf mode}

This is a special case of our extension running at a Wiki Media Foundation site.
The extension is still expected to not be integrated into Wikipedia,
as the ``local'' mode expects.
Instead, the extension ties into the \textit{edit hook} of
\program{mediawiki} to make notifications to the coloring server
of new revisions which are available.

