\chapter{Mediawiki Extension}

\section{Introduction}

\subsection{7-May-2009}

Ian has been the steward of the \program{mediawiki}
extension, but with him on leave Luca and I have been
trying to recall the details of the architecture.

\section{A Meeting}

\subsection{7-May-2009}

Luca and I had a meeting, where much of the discussion was
in trying to understand the organization of Ian's code
for the extension.

In trying to reason out the functionality, we determined that
there are three use cases:
\index{local mode}\index{remote mode}\index{wmf mode}
\begin{enumerate}
\item[local] - the wiki data and coloring is completely local on
	a single machine.
\item[remote] - coloring data is a remote service.
	There is no notification when new revisions are added to
	the wiki.  There is no extension on the hosting wiki.
\item[wmf] - the extension colors data as a remote service,
	but receives revision notifications via an \textit{edit hook}.
\end{enumerate}

\subsection{20-Jun-2009}

Tracing through the git history, the file \file{RemoteTrustAjax.php}
first makes an appearance on 18-Feb (commit 4e9a597\ldots).
This coincides with my first email to Ian to collaborate on a Firefox demo.
Most likely, then, the ``remote'' version isn't really what Luca and
I speculated --- it's probably just an interface to the remote service
that Ian implemented as part of the ``wmf'' mode, which allowed the
Firefox demo to get back data as colored HTML.

\section{Local Mode}
\index{local mode}

Normal wikis will just use ``local mode,'' where the WikiTrust extension
is integrated into the wiki and can detect revisions as they added
into the database.
In this case, the extension code would directly trigger the
OCaml code to color the revisions as they are made.

\subsection{19-Jun-2009}

Trying to understand the organization of the extension files.  Found a few weirdnesses, which I marked with ``TODO(Bo)''.

File \file{LocalTrust.php} looks like it initializes all the hooks.
\begin{enumerate}
\item Autoloads \file{LocalTrustImpl.php} and \file{RemoteTrustUpdate.php}.
\item Adds \function{TextTrust::init} as an extension function.
\item Exports \function{TextTrustImpl::handleVote} as an AJAX function.
\item Register \function{TextTrustUpdate::updateDB} to handle schema updates.
\item Add \function{TextTrust::ucscTrustTemplate} to hook \function{SkinTemplateTabs}.
\item Add \function{TextTrust::ucscArticleSaveComplete} to hook \function{ArticleSaveComplete}.
\item If the trust tab is selected, then add \function{TextTrust::ucscOutputBeforeHTML} to hook \function{OutputPageBeforeHTML}.
\end{enumerate}

Of special note is the function \function{runEvalEdit} in
\file{LocalTrustImpl.php}.
This function knows how to run our ocaml code directly to evaluate revisions.


\section{Remote Mode}
\index{remote mode}

Our original vision for the GeneWiki project was to run the
system in ``remote mode.''
In this use case, we would run a mediawiki instance at UCSC
with the WikiTrust extension in remote mode.
GeneWiki users of Wikipedia would have a special \textit{skin}
available which would allow them to query our instance for
colored revision information.

The key point of this mode is that is allows administrative separation
between the host wiki and the colorized wiki.
We would not rely on notifications from Wikipedia about new revisions
that needed coloring; instead, we would color articles
on demand as they were requested by the GeneWiki users.

On the UCSC side, a \program{mediawiki} instance is still required
because the colorized article needs to be converted into HTML.
The GeneWiki skin can only replace the HTML on the page it was
trying to show.

To implement this, a \program{mediawiki} AJAX call is made
from the GeneWiki skin to our remote colorized wiki code.

\subsection{19-Jun-2009}

The remote mode depends on the file \file{RemoteTrustAjax.php}.
This file is organized similarly to \file{LocalTrust.php},
with the following differences:
\begin{enumerate}
\item Another AJAX function is available: \function{TextTrustImpl::getColoredText}.
\item A different function is tied to the \function{OutputPageBeforeHTML} hook.  This one just adds buttons to the top of the page.
\end{enumerate}

\section{WMF Mode}
\index{wmf mode}

This is a special case of our extension running at a Wiki Media Foundation site.
The extension is still expected to not be integrated into Wikipedia,
as the ``local'' mode expects.
Instead, the extension ties into the \textit{edit hook} of
\program{mediawiki} to make notifications to the coloring server
of new revisions which are available.

\subsection{19-Jun-2009}

This code seems the most like \file{LocalTrust.php}.
In fact, in the init file, the only difference is that the skin is not optional.
The implementations are different, because the WMF version posts
a request to a remote server.

