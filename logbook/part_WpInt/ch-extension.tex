\chapter{Mediawiki Extension}

\section{Introduction}

\subsection{7-May-2009}

Ian has been the steward of the \program{mediawiki}
extension, but with him on leave Luca and I have been
trying to recall the details of the architecture.

\section{A Meeting}

\subsection{7-May-2009}

Luca and I had a meeting, where much of the discussion was
in trying to understand the organization of Ian's code
for the extension.

In trying to reason out the functionality, we determined that
there are three use cases:
\index{local mode}\index{remote mode}\index{wmf mode}
\begin{enumerate}
\item[local] - the wiki data and coloring is completely local on
	a single machine.
\item[remote] - coloring data is a remote service.
	There is no notification when new revisions are added to
	the wiki.  There is no extension on the hosting wiki.
\item[wmf] - the extension colors data as a remote service,
	but receives revision notifications via an \textit{edit hook}.
\end{enumerate}

\subsection{20-Jun-2009}

Tracing through the git history, the file \file{RemoteTrustAjax.php}
first makes an appearance on 18-Feb (commit 4e9a597\ldots).
This coincides with my first email to Ian to collaborate on a Firefox demo.
Most likely, then, the ``remote'' version isn't really what Luca and
I speculated --- it's probably just an interface to the remote service
that Ian implemented as part of the ``wmf'' mode, which allowed the
Firefox demo to get back data as colored HTML.


\section{Local Mode}
\index{local mode}

Normal wikis will just use ``local mode,'' where the WikiTrust extension
is integrated into the wiki and can detect revisions as they added
into the database.
In this case, the extension code would directly trigger the
OCaml code to color the revisions as they are made.

\subsection{19-Jun-2009}

Trying to understand the organization of the extension files.  Found a few weirdnesses, which I marked with ``TODO(Bo)''.

Refactoring the code is harder than I expected.
Each mode is very similar, but there are small differences
as the code evolved separately for each mode,
in addition to the intended differences between the mode.


\section{Remote Mode}
\index{remote mode}

Our original vision for the GeneWiki project was to run the
system in ``remote mode.''
In this use case, we would run a mediawiki instance at UCSC
with the WikiTrust extension in remote mode.
GeneWiki users of Wikipedia would have a special \textit{skin}
available which would allow them to query our instance for
colored revision information.

The key point of this mode is that is allows administrative separation
between the host wiki and the colorized wiki.
We would not rely on notifications from Wikipedia about new revisions
that needed coloring; instead, we would color articles
on demand as they were requested by the GeneWiki users.

On the UCSC side, a \program{mediawiki} instance is still required
because the colorized article needs to be converted into HTML.
The GeneWiki skin can only replace the HTML on the page it was
trying to show.

To implement this, a \program{mediawiki} AJAX call is made
from the GeneWiki skin to our remote colorized wiki code.

\subsection{19-Jun-2009}

The remote mode depends on the file \file{RemoteTrustAjax.php}.
This file is organized similarly to \file{LocalTrust.php},
with the following differences:
\begin{enumerate}
\item Another AJAX function is available: \function{TextTrustImpl::getColoredText}.
\item A different function is tied to the \function{OutputPageBeforeHTML} hook.  This one just adds buttons to the top of the page.
\end{enumerate}

\section{WMF Mode}
\index{wmf mode}

This is a special case of our extension running at a Wiki Media Foundation site.
The extension is still expected to not be integrated into Wikipedia,
as the ``local'' mode expects.
Instead, the extension ties into the \textit{edit hook} of
\program{mediawiki} to make notifications to the coloring server
of new revisions which are available.

\subsection{19-Jun-2009}

This code seems the most like \file{LocalTrust.php}.
In fact, in the init file, the only difference is that the skin is not optional.
The implementations are different, because the WMF version posts
a request to a remote server.

\section{Testing: Local Mode (DB revcolors)}

\subsection{5-Jul-2009}

Finished refactoring the code.
Tested on my laptop installation of WikiTrust (as documented
in Section~\ref{sec:up2date-testenv}).
The new code now sets default configuration values which
you can override in \file{LocalSettings.php}.
My new \file{LocalSettings.php} looks like:
\begin{verbatim}
$wgUseTidy = true;
$wgUseAjax = true;
require_once( $IP . "/extensions/WikiTrust/WikiTrust.php" );
$wgWikiTrustGadget = NULL;              // enable for everyone
\end{verbatim}

\section{Testing: Local Mode (file revcolors)}

\subsection{8-Jul-2009}

At our meeting yesterday, Luca described his plans for testing
the new batch coloring process.
We realized that this means that the ``local mode'' needs
to support reading colored data from disk (as opposed to
reading from the database, which would be the normal use case).
Ian and I have added some new configuration variables to
support this functionality.

I finished adding the support for reading colored revisions from
disk, and did some testing using my existing installation.
That turned out to be annoying, because I had to delete the
color data I already had in the database, but I did this from
the command line which created all the directories as owned by
me --- so the web process couldn't write to the directories.
In the end, I got it working but I feel that I should double-check
from a fresh install since that is what Luca will be doing next.

The additions to \file{LocalSettings.php} that I ended up with
are:
\begin{verbatim}
$wgUseTidy = true;
$wgUseAjax = true;
require_once( $IP . "/extensions/WikiTrust/WikiTrust.php" );
$wgWikiTrustGadget = NULL;              // enable for everyone
$wgWikiTrustColorPath = "/home/thumper/research/tmp/colors";
$wgWikiTrustCmdExtraArgs = "-wt_db_sig_base_path /home/thumper/research/tmp/sigs -wt_db_colored_base_path /home/thumper/research/tmp/colors -n_events 10000";
\end{verbatim}

\subsection{8-Jul-2009}

For a fresh installation, I decided to start with the latest
Ubuntu distribution, 9.04.
Additional packages are described by our \file{README}:
\begin{verbatim}
aptitude install mediawiki
aptitude install php5-cli
aptitude install mysql-server
aptitude install mysql-client
aptitude install tidy
\end{verbatim}

Past that, I generally follow the steps described in
Section~\ref{sec:up2date-testenv}.

First fix the alias for \mediawiki, by editing
\file{/etc/mediawiki/apache.conf}.
Uncomment the line:
\begin{verbatim}
Alias /mediawiki /var/lib/mediawiki
\end{verbatim}
and restart \mediawiki.
\begin{verbatim}
# /etc/init.d/apache2 restart
\end{verbatim}

Now you can configure \mediawiki by launching
a browser and navigating to \url{http://localhost/mediawiki/}.

Configuring the \mediawiki installation has a
bunch of questions to answer, most of them intuitive.
The only important ones to remember are:
\begin{tabular}{|r|l|}
\hline
Wiki name & WikiTest \\
Admin username & WikiSysop \\
Admin password & wksysop \\
Database name & wikidb \\
DB username & wikuser \\
DB password & wkuser \\
Superuser name & debian-sys-maint \\
Superuser password & taken from \texttt{/etc/mysql/debian.cnf} \\
\hline
\end{tabular}
Once configured, move the config file to its proper place:
\begin{verbatim}
# cd /var/lib/mediawiki
# mv config/LocalSettings.php .
# chmod 0 config
# chmod 0640 LocalSettings.php
\end{verbatim}

To put some data into the database, I copied the
file \file{wiki-00114050.xml.gz} from Luca's collection
of files split out of the last dump
(\url{file://redherring.cse.ucsc.edu/~luca/wikitrust1/enwiki-20080103/114/wiki-00114050.xml.gz})
and placed it into \url{file://localhost/~thumper/research/dat/orig/}.
The actual loading of the data requires the use of \program{mwdumper},
and the Java JRE (which isn't installed by default):
\begin{verbatim}
$ cd ~/research/dat/
$ wget http://download.wikimedia.org/tools/mwdumper.jar
$ sudo aptitude install sun-java6-jre
\end{verbatim}

Loading the test data with \program{mwdumper}:
\begin{verbatim}
$ cd ~/research/data/
$ gunzip -c orig/wiki-00114050.xml.gz | uniq > tmp.tmp
$ cat tmp.tmp | java -Xmx600M -server -jar mwdumper.jar --format=sql:1.5 | mysql -u wikiuser --password=wkuser wikidb
52 pages (140.921/sec), 57 revs (154.472/sec)
\end{verbatim}
Note that there's a weird step for removing the last line of
the file, which is a duplicate closing tag.
Double-checking the Special:AllPages page shows that
the pages were successfully loaded into the database.

Now we need to load the WikiTrust extension.
\begin{verbatim}
$ sudo aptitude install git-core gitk
$ cd ~/research
$ git clone ssh://thumper@trust.cse.ucsc.edu/pub/git/WikiTrust.git
Initialized empty Git repository in /home/thumper/research/WikiTrust/.git/
The authenticity of host 'trust.cse.ucsc.edu (128.114.63.37)' can't be established.
RSA key fingerprint is ca:c2:91:32:4b:02:88:80:4c:c0:15:3e:c5:50:8a:2a.
Are you sure you want to continue connecting (yes/no)? yes
Warning: Permanently added 'trust.cse.ucsc.edu,128.114.63.37' (RSA) to the list of known hosts.
thumper@trust.cse.ucsc.edu's password: 
remote: Generating pack...
remote: Counting objects: 759
remote: Done counting 5695 objects.
remote: Deltifying 5695 objects...
remote:  100% (5695/5695) done
remote: Total 5695 (delta 3658), reused 925 (delta 582)
Receiving objects: 100% (5695/5695), 4.22 MiB | 1092 KiB/s, done.
Resolving deltas: 100% (3658/3658), done.
\end{verbatim}

Godi is the best way to get OCaml running on a new machine.
Visit \url{http://godi.camlcity.org/godi/index.html} to download
the latest version of it.
The steps I followed to get it are:
\begin{verbatim}
$ sudo aptitude install m4
$ cd ~/research/tmp
$ wget http://download.camlcity.org/download/godi-rocketboost-20080630.tar.gz
$ tar -xzvf godi-rocketboost-*.tar.gz
$ cd godi-rocketboost-20080630
$ ./bootstrap --prefix ~/godi
$ export PATH=~/godi/bin:~/godi/sbin:$PATH
$ ./bootstrap_stage2
\end{verbatim}
At this point, the compilation fails because of PCRE being missing.
Edit \file{~/godi/etc/godi.conf} to uncomment the line which
uses the GODI version of PCRE.
\begin{verbatim}
$ ./bootstrap_stage2
$ cd ..
$ sudo aptitude install libmysqlclient16-dev
$ godi_console
\end{verbatim}
We also need the \package{OcamlLdaLibs} package:
\begin{verbatim}
$ cd ~/research/
$ git clone git://trust.cse.ucsc.edu/share/git/OcamlLdaLibs.git
$ cd OcamlLdaLibs
$ make all-godi
$ cd ../WikiTrust
$ make clean ; make all ; make allopt
\end{verbatim}


We need to create the path for storing the wikitrust data on disk:
\begin{verbatim}
$ mkdir ~/research/dat/test-data
$ sudo chown www-data.www-data ~/research/dat/test-data
\end{verbatim}
And link in the extension:
\begin{verbatim}
$ sudo ln -s /home/thumper/research/WikiTrust /var/lib/mediawiki/extensions/
\end{verbatim}

Finally, we need to edit \file{LocalSettings.php} to add the following lines:
\begin{verbatim}
$wgUseTidy = true;
$wgUseAjax = true;
require_once( $IP . "/extensions/WikiTrust/WikiTrust.php" );
$wgWikiTrustGadget = NULL;              // enable for everyone
$wgWikiTrustColorPath = "/home/thumper/research/test-data/colors";
$wgWikiTrustCmdExtraArgs = "-wt_db_sig_base_path /home/thumper/research/test-data/sigs -wt_db_colored_base_path /home/thumper/research/test-data/colors -n_events 10000";
\end{verbatim}

