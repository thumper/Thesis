\chapter{Firefox Plugin}

\section{Security}

\subsection{29-Sep-2010}

The problem with the current architecture (of users sending back the
rendered HTML for us to cache) is that a malicious user can send back
HTML that contains bad information or even triggers a ``drive by
download'' that installs malicious software.
I had always figured that the way around this was to get renderings
from multiple users and verify that the renderings are equivalent.
This has two problems that I can think of:
\begin{enumerate}
\item different versions of the plugin might render differently, and
\item a malicious user can make multiple submissions.
\end{enumerate}
The first is somewhat solved by storing the plugin version with the
rendering, but care must be taken in throwing out ``old'' versions
because a malicious user might pretend to be a newer plugin.
The second issue can be alleviated by using a distance metric on
IP address space, and requiring that the renderings come from hosts
which are relatively far apart from each other.
This raises the bar for attackers, but there is still opportunity
for large scale botnets to subvert the system.

