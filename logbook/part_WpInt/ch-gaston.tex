\chapter{Loading onto Gaston}

\section{Introduction}

\subsection{29-Sep-2010}

Our ``production system'' has been running for several months now,
with the English wikipedia running on Redherring, and several
different languages running on Gaston.
We have recently run into a few bugs that led to a relatively large
overhaul of how revisions are being stored, but we don't have a good
way to test the system.
Luca suggested that I set up a small English installation on Gaston
and use that to test the new code for problems.

\section{Setup}

\subsection{29-Sep-2010}

I have been tracing through Ian's installation files and trying
to organize it to be a bit more portable.
The core definitions are in \file{/etc/wikitrust/},
There are executables and cron files scattered elsewhere, but I am
slowly finding them and noting their installation in
\file{WikiTrust/util/ucsc-install/Makefile}.

Before running a test, the database needs to be setup:
\begin{verbatim}
$ mysql --user root --password mysql
mysql> create database enwiki;
mysql> GRANT ALL PRIVILEGES ON enwikidb.* TO 'wikiuser'@'localhost';
mysql> quit
\end{verbatim}

A directory for the downloaded revisions is also necessary:
\begin{verbatim}
$ cd /raid
$ sudo mkdir enwiki-test
$ sudo chown wikitrust.wikitrust enwiki-test
\end{verbatim}

The database also needs to be initialized with tables.
You can either setup a fullblown mediawiki installation,
or just create the empty datables.
To create the empty tables, use the following:
\begin{verbatim}
$ cd WikiTrust/sql
$ mysql --user wikiuser --password enwikidb
mysql> \. wikitrust.sql
mysql> quit
\end{verbatim}


\section{Debugging}

\subsection{29-Sep-2010}

In order to test things out the way that they would get launched
by the startup scripts, I hacked up a new script that mirrors
the work done by \program{wikitrustd}.
It's called \program{testlaunch} and you use it like so:
\begin{verbatim}
$ cd WikiTrust/remote/admin
$ sudo ./testlaunch enwiki
\end{verbatim}
This is most effective if you edit
\file{/etc/wikitrust/wikis_available.d/enwiki}
so that the \texttt{DISPATCHER} variable is pointing to
the version of the dispatcher compiled in your directory.

While testing out my configuration, I found that the
\program{dispatcher} was immediately terminating because
of a DB exception.
Due to the way that the dispatcher is started, it's a little
difficult to get the stack trace.
To do it, I had to clean my build tree and rebuild the debugging
version of \wikitrust, then modify the \file{enwiki} configuration
to say:
\begin{verbatim}
DISPATCHER="env OCAMLRUNPARAM=b /store/thumper/research/WikiTrust/remote/analysis/dispatcher"
\end{verbatim}
This will generate a stack trace with the exception.

Since I was getting a DB exception, I hacked the config a little
further to add the ``\texttt{-dump_db_calls}'' option.
This led to the discovery that I also need to initialize all
the tables after creating the database.


To add a page to the dispatcher queue, I used:
\begin{verbatim}
env WT_DBNAME=DBI:mysql:database=enwikidb:host=localhost WT_DBUSER=wikiuser WT_DBPASS=wikiword perl -I. -MWikiTrust -e "WikiTrust::cmdline();" method=gettext pageid=3703446
\end{verbatim}
