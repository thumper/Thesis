\chapter{GeneWiki Project}

\section{Introduction}

\subsection{30-Apr-2009}

\index{GeneWiki}
The GeneWiki is a subset of Wikipedia pages which is actively
edited by a group of biologists using the Wikipedia as
a collaboration tool.
Our research group was contacted because some (several?)
biologists were interested in using WikiTrust to help them
understand the content evolution and spot potential errors
more quickly.

\section{A Progress Report}

\subsection{30-Apr-2009}

Received an email from Andrew Su today, about providing an
update on our Wikipedia integration as it relates to the GeneWiki project.
In particular, he suggests that some screen grabs of colored
GeneWiki pages would be useful for the progress report.

Andrew provides two links which provide a list of titles that
are part of the GeneWiki:
\begin{quote}
Good timing... I just set up this service:

\url{http://plugins.gnf.org/cgi-bin/what_links_PBB.cgi}

(which basically parses this link: \url{http://en.wikipedia.org/w/index.php?title=Special%3AWhatLinksHere&target=Template%3AGNF+Protein+box&namespace=0})
\end{quote}

After more discussion, Andrew suggests a few particular articles
that might lead to interesting colorization:
\begin{itemize}
\item Insulin
\item FOXP2
\item BRCA2
\item Reelin
\item Human chorionic gonadotropin
\item Cannabinoid receptor
\item Erythropoietin
\end{itemize}


\subsection{2-May-2009}

\label{sec:genewiki-eval}

For a decent looking demo, article text needed to be downloaded,
analyzed, and then rendered.
To simplify the process, I decided to use the \mediawiki installation
on my laptop (as documented in Section~\ref{sec:up2date-testing}).
Luca wisely suggested that I temporarily modify the code to include
the \texttt{rvexpandtemplates=1} parameter, so that the articles
don't look so choppy.
(Not all templates are expanded, it seems, but the articles
are readable.)

\indexprogram{downloadwp}
\indexprogram{eval\_online\_wiki}
I placed the sample titles provided by Andrew into temporary
file \file{titles.txt}.
Downloading and analyzing the articles followed this sequence:
\begin{verbatim}
./downloadwp -db_pass wkuser -log_file download.log < titles.txt 
./eval_online_wiki -n_events 1000000 -db_user wikiuser -db_pass wkuser -db_name wikidb -delete_all
\end{verbatim}

There was some problem with the coloring step, which Luca helped me clear up.
During the database initialization, there had been some errors due
to path changes in the code, and when I reran the initialization
not all the steps completed.
Clearing and recreating the \wikitrust tables fixed the problem.

\subsection{5-May-2009}

Used Firefox extension \package{Screengrab} to capture
the rendered coloring of a few pages.

After sending around screenshots for \article{Insulin},
\article{FOXP2}, and \article{Erythropoietin},
Andrew suggests using a specific revision of \article{Insulin}
which has been vandalized:
\url{http://en.wikipedia.org/w/index.php?title=Insulin&oldid=275210688}.

