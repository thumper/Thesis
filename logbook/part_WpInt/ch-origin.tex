\chapter{The Origin Bug}

\section{Introduction}

\subsection{20-May-2009}

Back in March, Ian and I noticed a problem with the origin information
while debugging the Firefox extension.
In a few articles, it seemed that the origin was being ascribed
mainly to a single revision/author.
We believe we saw this in the \article{Olwen} article
(but the code was still changing at this point),
and definitely confirmed it as being unexpected in
the \article{Rio Grande} article.
(Discussed in emails with Ian on 3-Mar-2009,
with subject ``Try your extension now,
on any page you like''.)

The problem has been on the back-burner since then,
while we work on other tasks.
The impending need to analyze the Wikipedia, which we expect
to take around 30 days, implies that it would be wise to
investigate this problem now.

At around the same time we noticed this bug, we
also noticed that the articles were not showing
paragraph breaks properly (see the discussion about
the XML parsing problem~\ref{sec:up2date-xml}).
This suggests that the problem might have been with
the download procedure that Ian was using at the time,
although we felt this to be unlikely.

My plan is to try to reproduce the bug using the
two data import methods we have: loading via XML files,
and downloading via the Wikipedia API.

\section{Reproduction via XML data}

\subsection{20-May-2009}

I want to end up with a small test database that contains
the error, so I decided to use the test environment that
I had previously setup in Section~\ref{sec:up2date-testenv}.

Article \article{Rio Grande} is known to be have had a problem,
so I looked up its page ID (43005) and searched through the
metadata to find the \program{splitwiki} output containing
the page data.
\begin{verbatim}
$ cd ~luca/wikitrust2/enwiki-20080103-metadata
$ find . -name "*.xml" \! -name "*.info.*" -exec grep -q "<id>43005</id>" {} \; -print
\end{verbatim}
This returns a short list of matching files; there
is more than one because userid 43005 also passes the \program{grep} test.
(Possibly could have included a second \program{grep} test
to search for ``Rio Grande'' in the title.)
Downloaded metadata and revision info to my test environment:
\begin{verbatim}
$ cd ~/research/dat/meta
$ scp 'redherring.cse.ucsc.edu:/export/notbackedup/wikitrust2/enwiki-20080103-metadata/000/wiki-*0374.*' .
$ cd ../unpack/000/000
$ mkdir -p 043/005
$ cd 043/005
$ rsync -av --progress 'redherring.cse.ucsc.edu:/export/notbackedup/wikitrust2/enwiki-20080103-revs/000/000/043/005/*' .
\end{verbatim}


Now the data needs to be loaded, and the evaluation run:
\begin{verbatim}
$ ./load_data.py --clear_db meta/wiki-00000374.xml
$ time ./eval_online_wiki -n_events 29000 -wt_db_rev_base_path unpack -wt_db_sig_base_path sigs -wt_db_colored_base_path colors -db_user wikiuser -db_pass wkuser -db_name wikidb -log_file eval.log -delete_all
\end{verbatim}
This places the colored revisions into the filesytem.


At this point, I ran into a little bit of trouble.
There doesn't seem to be any way to get our extension to
display colored articles from the filesystem.

Instead, I located the latest colored revision, and
inspected it manually:
\begin{verbatim}
$ cd colors/000/000/043/005
$ ls -1 */*.gz | sed -e 's/^....//' | sort | tail -1
\end{verbatim}
There doesn't appear to be anything wrong with the origin
information, so I went back to the Firefox extension to
try to recreate the problem.
Sure enough, the majority of the first paragraph of the ``Description''
section is attributed to author ``67.174.187.41'', when using
the Firefox extension in ``newapi'' mode.

I confirmed that this paragraph seems to have correct origin
information in the colored version I created.

\section{Reproduction via Wikipedia API}

\subsection{20-May-2009}

To test the Wikipedia API, I can reload the database with the
metadata from my original testing, and then download the
entire \article{Rio Grande} article.
\begin{verbatim}
$ ./load_data.py --clear_db meta/wiki-00114050.xml
$ echo "Rio Grande" > titles2.txt
$ ./downloadwp -db_pass wkuser -log_file dlwp.log < titles2.txt
$ ./eval_online_wiki -n_events 1000000 -db_user wikiuser -db_pass wkuser -db_name wikidb -delete_all
\end{verbatim}
To view the colored data, I used the WikiTrust extension
set in ``local'' mode.

This time, the problem definitely crops up.
I have been tracking a paragraph that begins with:
\begin{quote}
``The Rio Grande rises in high mountains and flows for
much of its length at high elevation\ldots''
\end{quote}
As of revision 65342218, the text was attributed to
user Zundark, the author of the initial revision.
The next revision, revid 65342717, ascribes the text
to user Hajor.
And the next revision, revid 65443458, ascribes the text
to user 69.154.41.250 (who has not edited the article for
many revisions before).

Just to verify that the problem isn't with the tooltips,
I logged into mysql to check directly.
\begin{verbatim}
$ mysql -u wikiuser -p wikidb
mysql> select revision_text from wikitrust_colored_markup where revision_id = 65443458;
\end{verbatim}
Reviewing the result, I discovered that the attribution
data appears to be correct.
It seems that the tooltip code is to blame here.

Just to verify, I sent the data to a file so that I
could compare it with the colored version computed
in my previous test.

At first glance, they are a good match, but running \program{diff}
reveals differences in the first line.
Looking more closely, I see that the XML based import
has entities in the text; for example, angle brackets
are expanded as the typical \texttt{\&lt;} and \texttt{\&gt;}.
And this, it turns out, is causing a problem -- albeit a
different problem than I was initially investigating.

In the downloaded text is the sequence:
\begin{quote}
$<$ref name="usgs"$>$\{\{cite paper\ldots
\end{quote}
which seems to get handled properly by the coloring code.
The XML version of this revision shows the text as:
\begin{quote}
\&lt;ref name=\&quot;usgs\&quot;\&gt;\{\{cite paper\ldots
\end{quote}
which the coloring code sees as separate text segments
and inserts coloring commands aplenty.


As I see it, there are a few outstanding issues:
\begin{enumerate}
\item The XML extraction code needs to convert entities to their
    proper values.
\item The tooltip code needs to show the correct origin information
    for text.
\item The coloring code needs to be aware of markup sections
    that should not be interrupted.
\item The first trust value of each new line seems to be
    duplicated unnecessarily.
\end{enumerate}


\section{Fixing the Unpacked XML}

\subsection{21-May-2009}
\label{sec:origin-fixxml}

I met with Luca today, and we decided that the most expedient
way to fix the entities was to write a shell script that could
be used in conjuntion with \program{find} to process every file.
A first step was to inventory what entities were being used and
make sure we handled all of them.
The script to do this survey looks like:
\begin{verbatim}
#!/bin/sh
gunzip -c $1 | grep \& | sed -e 's/.*\&/\&/' -e 's/;.*/;/
\end{verbatim}
and running the script becomes:
\begin{verbatim}
$ find ~luca/wikitrust2/enwiki-20080103-revs -name "*.gz" -exec /tmp/scan.sh | uniq > tmp.tmp
$ sort tmp.tmp | uniq
\end{verbatim}
I interrupted the run after a reasonable amount of time (since this
was only a survey, and got back only the standard entities defined by
XML: \texttt{\&amp;}, \texttt{\&lt;}, \texttt{\&gt;}, and \texttt{\&quot;}.
(XML also defines \texttt{\&apos;}.)

While fiddling around with the next step, I ran into
article \article{Sucre}, which includes the numerical
entities (\eg, \texttt{\&\#1595;}) that XML also happens to allow.
The question is whether Wikipedia expects these to be
encoded as UTF-8 characters, or to leave them in their
entity encoding.
I don't think that it matters very much, since both methods
are visually the same, but I ran a JSON download of the
\article{Sucre} article and found that the numerical entities
are present in the actual text.
Thus, it is safe to translate only the five basic XML entities
to their character representations.

To do the conversion, I improved \file{scan.sh} to allow
me to stop the translation and to recompress to a new directory.
The new file, \file{fix.sh}, is presented here:
\begin{verbatim}
#!/bin/sh
set -e
if test -e "/tmp/find.stop"; then
    exit 0;
fi
dir=`dirname $1`
dest=/export/notbackedup/wikitrust1/enwiki-20080103-revs-thumper
mkdir -p ${dest}/${dir}
gunzip -c $1 | sed -e 's/\&amp;/\&/g' -e 's/\&lt;/</g' -e 's/\&gt;/>/g' -e 's/\&quot;/"/g' -e "s/\&apos;/\'/g" | gzip --best > ${dest}/$1
exit 0
\end{verbatim}
Note that I tell the shell to abort on any error, which we
can use in the \program{find} command to print out the file
causing the error.
And since Redherring has 8 CPUs, we can run multiple \program{find}
commands to convert the files:
\begin{verbatim}
$ cd ~luca/wikitrust2/enwiki-20080103-revs
$ find . -name "*[01].gz" \! -exec /tmp/fix.sh {} \; -print &
$ find . -name "*[23].gz" \! -exec /tmp/fix.sh {} \; -print &
$ find . -name "*[45].gz" \! -exec /tmp/fix.sh {} \; -print &
$ find . -name "*[67].gz" \! -exec /tmp/fix.sh {} \; -print &
$ find . -name "*[89].gz" \! -exec /tmp/fix.sh {} \; -print &
\end{verbatim}

I am a little surprised, because the process seems to be
disk I/O bound.
Monitoring just after starting the process shows the
disk reading a maximum of 3MB/sec, and writing up to 8MB/sec;
the CPU is spending up to 50\% in the WAIT state, but seems to be
settling down at about 15\%.

