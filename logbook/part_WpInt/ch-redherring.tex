\chapter{Loading onto Redherring}

\section{Introduction}

\subsection{28-Apr-2009}

Luca has unpacked the 20080103 dump of the Wikipedia into
a set of flat files, separating the text from the metadata.
Now the metadata needs to be loaded into a mysql database.

\section{Initialization}

\subsection{28-Apr-2009}

To begin, a \textbf{mediawiki} installation must be created and
initialized so that the database tables are created.
Ian documents how to do this in an email on 9-Apr-2009,
subject ``\textit{Directions to set up a wiki on redherring}.''
There are several errors in this email, and points that are
left to the imagination, but it conveys enough to figure
out the actual commands.

The particular instructions I followed were:
\begin{enumerate}
\item Download \textbf{mediawiki} and rename the directory 
	to \url{/export/notbackedup/wikitrust1/hosted\_wikis/wikipedia-2008}
\item Configure apache by doing the following:
\begin{verbatim}
$ sudo /usr/local/bin/wikitrust-ownwebconf
$ vi /etc/httpd/conf.d/wikitrust.conf
$ sudo /usr/local/bin/wikitrust-restartweb
\end{verbatim}
	and placed the following text into the file at the end:
\begin{verbatim}
Listen 10304
NameVirtualHost *:10304
<VirtualHost *:10304>
  ServerAdmin thumper@cs.ucsc.edu
  ServerName wikitrust.soe.ucsc.edu

  DocumentRoot /export/notbackedup/wikitrust1/hosted_wikis/wikipedia-2008
  <Directory />
    Options FollowSymLinks
    AllowOverride None
    Order allow,deny
    allow from all
  </Directory>
</VirtualHost>
\end{verbatim}

\item Then goto the mediawiki configuration page, at
    \url{http://redherring.cse.ucsc.edu:10304}.
    The configuration is standard, but a few items are worth noting:
\begin{table}[h]
\begin{tabular}{r l}
Wiki name & Wikipedia \\
Admin username & WikiSysop \\
Admin password & wksysop \\
Database name & wikidb-thumper \\
DB username & wikuser \\
DB password & wikiword \\
Superuser name & wikidba \\
Superuser password & taken from \texttt{/etc/.mypass} \\
\end{tabular}
\end{table}


\end{enumerate}

\section{Configuring the Ocaml Environment}

\subsection{28-Apr-2009}

The simplest way to get a working \textbf{Ocaml} environment is
to use the \textbf{godi} package management system.

Start by visiting the website:
\url{http://godi.camlcity.org/godi/index.html}.
Unpack the \texttt{rocketboost} package and run as:
\begin{verbatim}
$ cd /tmp/thumper/godi-rocketboost-20080630
$ ./bootstrap --prefix /export/notbackedup/wikitrust1/thumper/godi
$ set path= ( /export/notbackedup/wikitrust1/thumper/godi/sbin /export/notbackedup/wikitrust1/thumper/godi/bin /usr/local/bin /usr/sbin /usr/bin /sbin /bin )
$ vi /export/notbackedup/wikitrust1/thumper/godi/etc/godi.conf
$ ./bootstrap_stage2
\end{verbatim}
When editing the \texttt{godi.conf} file, uncomment the line
for PCRE.


\section{Configure Mediawiki with WikiTrust Extension}

\subsection{28-Apr-2009}

First, load the wikitrust extension into \textbf{mediawiki}:
\begin{verbatim}
$ cd /export/notbackedup/wikitrust1/hosted\_wikis/wikipedia-2008/extensions
$ git clone ssh://thumper@trust.cse.ucsc.edu/pub/git/WikiTrust.git
\end{verbatim}

Then we run Ian's script to build the wikitrust mysql tables:
\begin{verbatim}
$ cd sql
$ ./create_db.php /export/notbackedup/wikitrust1/hosted\_wikis/wikipedia-2008 wikidba
\end{verbatim}

\section{Loading the data}

\subsection{28-Apr-2009}

To load the data, we need to use the \textbf{load\_data.py} script,
which needs a little configuration:
\begin{verbatim}
$ cd ../test-scripts
$ cp db_access_data.ini.sample db_access_data.ini
$ vi db_access_data.ini
\end{verbatim}
Fill the INI file with the information we configured above.
Also put a recent copy of \texttt{mwdumper.jar} into the directory.

As a performance tip, it is recommended to remove all indices
before loading data.  This can be done with the following set
of commands:
\begin{verbatim}
$ ./load_data.py --clear_db
$ mysqldump -u wikidba -p wikidb-thumper > wikidb-thumper.sql
$ mysql -u wikidba -p wikidb-thumper
mysql> alter table page drop index name_title;
mysql> alter table page drop index page_random;
mysql> alter table page drop index page_len;
mysql> alter table revision drop index rev_id;
mysql> alter table revision drop index rev_timestamp;
mysql> alter table revision drop index page_timestamp;
mysql> alter table revision drop index user_timestamp;
mysql> alter table revision drop index usertext_timestamp;
mysql> alter table revision drop primary key;
mysql> alter table revision change `rev_id` `rev_id` int(10) unsigned NOT NULL;
mysql> alter table page change `page_id` `page_id` int(10) unsigned NOT NULL;
mysql> alter table text change `old_id` `old_id` int(10) unsigned NOT NULL;
mysql> alter table page drop primary key;
mysql> alter table text drop primary key;

Now we can process all the metadata files to load them into the DB:
\begin{verbatim}
$ ./load_data.py --clear_db
$ find ~luca/wikitrust2/enwiki-20080103-metadata -name "*[0-9].xml" -print0 | xargs -0 ./load_data.py 
\end{verbatim}

