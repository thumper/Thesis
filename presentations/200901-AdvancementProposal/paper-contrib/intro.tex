
A good reputation function has the basic property
of describing which behaviors are desirable
and which are undesirable.
To define the reputation of an author for the Wikipedia, we relied upon
intuition and insight to derive the \textit{text survival}
and \textit{edit survival} measures which we base our reputation upon.
Both measures compute the ``quantity'' of the contribution
that an author makes, and factor
in a notion of ``quality'' as judged by later revisions
to the author's work.
That is, we desire the reputation of an author to go up
as the amount of work she contributes goes up,
with the proviso that the work be of good quality;
vandalism and spam should instead be punished within
the reputation system.

Although text survival and edit longevity are just components of our reputation
system, their purpose is to estimate the \textit{amount of contribution}
that each author makes to the Wikipedia --- and this is an
application that has interest beyond reputation
systems~\cite{Wales2005,EditsEqQuality2007,
Bourgeoisie2007,WikiDashboard2008,OrtegaBarahona2007,SteinHess2007,
Swartz2006}.
Previous research about contributions have used various
``quantity'' based measures, with no consideration to ``quality,''
which might substantially change the analyses of user behavior.
On the other hand, our own measures for contribution are
not the only possible formulations, which raises the question
of what other measures are possible and how do they differ?
We now present some results from completed research on this topic~\cite{AuthorContrib2008}.

