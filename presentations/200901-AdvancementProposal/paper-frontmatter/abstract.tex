\begin{abstract}
The Wikipedia is an experiment in group collaboration to
build an encyclopedia in an open way.
Perennial criticism centers around the lack of accountability
for authors, and the potential for misinformation.
The large community around the Wikipedia project actively
fixes errors as they are discovered, but an unending
stream of vandals and spammers chip
away at the good will of volunteers who
maintain the project for the collective good.
We suggest that an author reputation system and
an article text trust system can be used to focus
the volunteer effort on edits more likely to be a problem,
making more efficient use of the project's human resources.

In order to minimize the human effort diverted towards a reputation system,
we choose to preserve the current user experience as much as
possible.
To that end, we propose \textit{content-driven} reputation:
using only the version history, the content evolution of an
article is used to measure how well the authors
contribute to the Wikipedia.
When authors make contributions that are preserved over many
revisions, their reputation goes up.
We desire that author reputation be correlated with the
stability of the text they contribute ---
low reputation should be a predictor of author
contributions which will be edited or deleted.

Author reputation and article history are used as
inputs to our content-driven trust system.
Trust is computed on a per-word basis, with the
goal of highlighting changing portions of article text.
We desire that low trust correlate with text that
is deleted on the next revision, but we also don't want
to habituate users to our flagging.
These goals are somewhat contradictory, so we describe
how to measure and combine them.

We propose automated methods for evaluating the quality of
authors, and (separately) the text they create.
By relying on the history of edits, we can draw inferences about
the quality of the text that has come earlier in the history.
Instead of measuring ``truth,'' our reputation ideas
measure the ``group consensus'' in a piece of text.
As the article text stabilizes over time, we conclude that
it has reached a form which most members of the community can
reasonably agree on.
As group collaboration increases in prominence on the Internet,
we feel that this research will open the door on new applications
and quality measures.

\end{abstract}
