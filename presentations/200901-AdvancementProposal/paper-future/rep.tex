\subsection{Unproposed}

Our initial works in author reputation and trust within
the Wikipedia present very positive results
based on a data-driven evaluation of trying
to predict future deletions of text~\cite{www07,WikiTrust2008}.
Although we are not proposing to investigate these directions,
some additional topics under our consideration would strengthen
the original results:

\subsubsection{Algorithm Comparison}

Our original work made some comparisons with
common suggestions for reputation, such as edit count
and text age~\cite{www07}.
We did not attempt to make comparisons with other
related works~\cite{McGuinness06,WikiMTWtrust06},
due to the complexity of their implementations.
It may be possible to compare against~\cite{WikiMTWtrust06}
by duplicating one of the evaluation criteria
and evaluating our own work in the same manner:
derive a single trust value for an article,
and predict whether it is a featured article or not.
We feel that this is a weak comparison that would
not illuminate the differences between the two approaches.

\subsubsection{Evaluation Stabilization}

Another direction to pursue is in stabilizing
the precision and recall of our content-driven reputation.
More recent tests have revealed that there can
be substantial variation in the precision and recall over time,
which we believe is due to bulk edits performed by bots.
For the sake of completeness, this should be investigated.
The most likely result will be to identify bots by the
pattern of their edits and separate them from human users.

\subsubsection{Attribution Evaluations}

As mentioned in Section~\ref{sec-attribution-eval},
the question of how to evaluate author attribution
systems is open.
We suggested that an indirect evaluation would be
to measure the predictive power of our reputation system,
which should theoretically increase as the attribution more closely
assigns text to the correct author.
Given more time, there are two concrete steps that can also be taken.
The first is to analyze several documents by hand
and create a test corpus with known attributions;
this is limited by the time and effort required to
perform the analysis.
The second step towards building an evaluation metric
is to conduct a user study:
choose limited sections of text from Wikipedia
articles, show sections of potentially copied text,
and ask users to select their best guess at the author
of the text.
User studies are often limited by the number of
people that participate, but we propose that
the Amazon Mechanical Turk\footnote{\url{http://www.mturk.com}}
service might offer easy access to motivated users.
By collecting a large enough sample of user opinions
on how text should be attributed, the statistics will
give us a good approximation of what decisions people
make in each case; the statistics may also be able to
highlight unusual cases where there is significant disagreement
between people.

\subsubsection{User Dynamics}

Besides how users risk their own reputation in
unfamiliar subject areas, we can also ask questions
about the lifecycle of authors.
For instance, there are many anecdotal stories
of academics that contribute to Wikipedia in their
area of expertise, but are reverted by established users.
A user study to evaluate anonymous contributions,
in the same vein as conducted in~\cite{www07},
can be used to reveal these contributions.
We hypothesize that reverts of good material
are primarily due to authors who view themselves
as caretakers of specific articles.
This raises another question about user beginnings:
if an anonymous user is not reverted early on,
does he go on to become a full Wikipedian?
Similarly, can we classify how users stop participating?
For instance, does participating in an edit war
increase the likelihood of leaving the project?


Looking at the how the progression of reputation
changes over time can help identify unusual users,
automating some of the work of the Wikidetective.
In~\cite{Durova2007}, a pattern of small edits early in
the editing history is used as a sockpuppet indicator;
we can use reputation increments to identify such accounts.
We can also build a social network (with positive and negative
relations) between authors, to expose cliques of authors.
These same tools can also be used to scrutinize the
work of self-organized groups within the Wikipedia project.

\subsection{Attack Resistance}

The primary threat concern in the Wikipedia project is the
\intro{Sybil attack}~\cite{Douceur02,Sybil05,TrustSybil05,SybilOverview} ---
multiple accounts (also known as \intro{sockpuppets} within
the Wikipedia community)
which are controlled by a single bad actor,
but cannot be distinguished from ordinary users.
In the current Internet environment,
a dedicated bad actor can easily have hundreds of sockpuppets.
Coordinating these accounts can enable the bad actor
to manipulate the reputation and trust systems,
while masking the fact that a single person is
responsible for the edits.

For the reputation system, we want to ensure that each
user is judged by other community members.
The existance of sockpuppets enables the bad actor
to make a sizable contribution (not necessarily useful!),
then use the sockpuppets to give positive feedback on
this contribution up to the time horizon which the reputation
system evaluates.
One possibility of repairing this scenario is to track
the social networks that arise from users editing each other
and using the social network to identify collusion~\cite{HMS02,MSI02}.
Each time a user makes an edit to an article, they are implicitly
making positive and negative judgements about previous users,
which we can take to be links in a social network.
The bad actor is at a distinct advantage here, though,
since they can create ``links'' between the sockpuppets
and unrelated users at their own whim.

Another variation, which we believe can thwart a Sybil attack,
uses the reputation of the author as a threshold on who
can make judgements.
For example, consider the reputation increase due to
edit survival (explained in Section~\ref{sec-reputation}
and Figure~\ref{fig-editlong}).
The author \editor{k} of version \version{k} is punished or
rewarded by comparing \version{k} against two reference
versions: \version{k-1} and \version{j}, for $j > k$.
By requiring that each reference version author have a higher
reputation than author \editor{k}, we ensure that a bad
actor cannot raise his reputation via Sybil attack
to a value higher than the highest he has already attained.
In order for regular users to gain reputation, we propose
to lower the threshold for reference reputations if all feedback
is positive for many revisions.


For the trust system, our current assessment is that there are two
main types of attack.
One type of attack tries to raise the trust in a portion of text by
repeatedly performing minor revisions of the page.
The reputation system provides a first line of defense against this
attack: authors can raise text trust only up to their reputation
value.
To further strengthen the system, we plan to develop a randomized
signature algorithm, which tags each word of the text of the most
recent revision of each article with a signature.
The signature is computed from the IDs of the authors who raise the
trust of the word, using a hashing scheme derived from Bloom filters
\cite{Bloom70,Mitzenmacher02}
or approximate membership techniques \cite{ApproxMemb06}.
The goal is to remember the authors who have recently raised the trust
of a word, and prevent them from doing so again until a sufficient
number of unrelated authors has also affected the trust of the word.
This signature problem is related to problems in network design, where
network flows must be tracked, and ``forgotten'' after some time
\cite{BroderMitzen04,ApproxMemb06}.

The second type of attack consists in repeatedly re-arranging the text
of an article, causing it to lose trust.
We propose developing algorithms that will enable us to quickly find the
past article revision most similar to the current one, so that once
the article is fixed, its old trust values can be reinstated.
The problem is related to finding closely-similar page pairs on the Web
\cite{Gusfield97,Lopresti99,Bilenko03}.



